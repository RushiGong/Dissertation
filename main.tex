\documentclass[letterpaper, 12pt]{report}
\usepackage[margin=0.75in]{geometry}
%\usepackage[english]{babel} 
\usepackage[T1]{fontenc}
\usepackage{graphicx, lipsum, textcomp, float} %figure formatting
\usepackage{hyperref} %referancing package
%\usepackage[version=4]{mhchem}  %chemical equations and formulas
\usepackage{chemformula}
%\usepackage{advdate, datenumber} %date packages
\usepackage{amsmath}
\usepackage{amssymb}
%\usepackage{amsfonts}
\usepackage{multirow}
%\usepackage{comment}
\usepackage{longtable}
\usepackage{xcolor}
\usepackage{listings}
\usepackage{adjustbox}
\usepackage{array, booktabs}
\usepackage{nameref}
\usepackage{etaremune}
\DeclareUnicodeCharacter{2212}{-}
%Listings
\usepackage{minted}
%\usepackage[finalizecache,cachedir=.]{minted}
%\usepackage[frozencache,cachedir=.]{minted}

%Font
\renewcommand{\thelinenumber}{\raisebox{1pt}{\textcolor[RGB]{200,200,200}{\arabic{linenumber}}}}
%\usepackage[scaled]{beramono}
%\usepackage{tgpagella}
%\renewcommand{\familydefault}{\sfdefault} 
\definecolor{darkgreen}{rgb}{0.05, 0.3, 0.1}

%\usepackage[htt]{hyphenat} %texttt hyphenation breaks

\let\oldtexttt\texttt
\renewcommand{\texttt}[1]{\oldtexttt{\textcolor{darkgreen}{#1}}}


% Line spacing (double for title page, single for TOC, and 1.5 for body)
\usepackage{setspace}

% Bibliography
\usepackage[
    style=ieee, 
    isbn=false, 
    url=true, 
    natbib=true, 
    backend=bibtex,
    maxcitenames=1,
    mincitenames=1
    ]{biblatex}
\addbibresource{references.bib}
\renewcommand*{\bibfont}{\footnotesize}

% Smaller figure captions 
\usepackage{caption}
\captionsetup[figure]{font=normalsize, labelfont=normalsize}

% Section titles
\usepackage{titlesec}
\setcounter{secnumdepth}{4}

% Chapter
\titleformat{\chapter}[display]
{\Large\bfseries\centering}
{Chapter \thechapter}{0.5em}{}[\vspace{2ex}\titlerule]
\titlespacing*{\chapter}{0pt}{0pt}{30pt}

% Section
\titleformat{\section}[hang]
{\large\bfseries}
{\thesection}{0.5em}{}

% Subsection
\titleformat{\subsection}[hang]
{\large\bfseries}
{\thesubsection}{0.5em}{}

% Subsubection
\titleformat{\subsubsection}[hang]
{\large\bfseries}
{\thesubsubsection}{0.5em}{}

% less section skip in the table of contents
\usepackage{tocbasic}

% Table
\renewcommand{\arraystretch}{1.2}

% Section
\DeclareTOCStyleEntry[
  beforeskip=.2em plus 1pt,% default is 1em plus 1pt
  pagenumberformat=\textbf
]{tocline}{section}

% Chapter
\DeclareTOCStyleEntry[
  entrynumberformat=\entrywithprefix{\chaptername},
  dynnumwidth
]{tocline}{chapter}
\newcommand*\entrywithprefix[2]{#1~#2}

% Appendix
\usepackage[toc]{appendix}
\newcommand{\mypart}[1]{\thispagestyle{empty}\part*{#1}}%\addtocounter{page}{-1}}

% MISC
\setlength\parindent{6pt} %paragraph indentation
\setlength{\parskip}{6pt} %paragraph spacing

%set hyperlinks colors
\definecolor{mypurple}{RGB}{140,54,140}
\definecolor{homered}{RGB}{127, 0, 10}
\definecolor{officeorange}{RGB}{204, 75, 0}
\definecolor{mauroblue}{RGB}{53, 48, 217}
\definecolor{citegreen}{RGB}{15, 133, 13}
\definecolor{hyperlinkpurple}{RGB}{42, 0, 163}
\definecolor{subtlegray}{gray}{0.98}
\definecolor{subduedgray}{gray}{0.75}
\hypersetup{
    colorlinks=true,
    linkcolor=hyperlinkpurple,
    filecolor=mypurple,      
    urlcolor=teal,
    citecolor=citegreen
}

% Macros:
% Full number and reference name hyperlinking
\newcommand*{\fullref}[1]{\hyperref[{#1}]{\ref*{#1} on \nameref*{#1}}}
% Acknoledgments on per-chapter basis
\newcommand{\acknowledge}[1]{\textit{
\small
Acknowledgment: #1
}}
% TODO markers.
\newcommand{\todo}{
\begin{center}
\textcolor{mauroblue}{
\textit{
This section is currently under preparation.
}}
\end{center}
}



%%%%%%%%%%%%%%%%%%%%%%%%%%%%%%%%%%%%%%%%%%%%%%%%%%%%%%%%%%%%%%%%%%%%%%%%%%%%
%%%%%%%%%%%%%%%%%%%%%%%%%%%%%   DOCUMENT   %%%%%%%%%%%%%%%%%%%%%%%%%%%%%%%%%

\begin{document}

% Front matter manually formatted according to the rules 
\pagenumbering{roman}
\thispagestyle{empty}
\setstretch{1}

% Title page body
{
\centering
The Pennsylvania State University\\
The Graduate School\\
\vfill
\setstretch{2}
{
\fontsize{14}{16}\selectfont
\textbf{INVESTIGATION OF ATOMIC ENVIRONMENTS FROM COMPUTATIONAL THERMODYNAMICS: APPLICATIONS IN INTERMETALLIC CATALYSTS AND MOLTEN SALTS}\\
}
\vfill
A Dissertation in\\
Materials Science and Engineering\\
by\\
Rushi Gong\\
\vfill
© 2024 Rushi Gong\\
\setstretch{1}
\vfill
Submitted in Partial Fulfillment\\
of the Requirements\\
for the Degree of\\

\vfill
Doctor of Philosophy\\
\vfill
December 2024\\
\vfill
}

% Committee page
\newpage
\setstretch{1.5}
\setlength\parindent{0pt} %no paragraph indentation

The dissertation of Rushi Gong was reviewed and approved by the following:\\

\textbf{Zi-Kui Liu}\\
Dorothy Pate Enright Professor at the Department of Materials Science and Engineering\\
Director of the Phases Research Laboratory\\
Dissertation Advisor and Chair of the Committee\\

\textbf{Michael Janik}\\
Professor of Chemical Engineering\\

\textbf{Hojong Kim}\\
Associate Professor of Materials Science and Engineering\\

\textbf{John Mauro}\\
Dorothy Pate Enright Professor and Associate Head for Graduate Education\\
Chair, Intercollege Graduate Degree Program in Materials Science and Engineering\\
Program Head

\vfill


\newpage
\chapter*{Abstract}
Understanding atomic environments is essential for optimizing the performance and properties of materials by providing insights into structure-property relationships. For complex multi-component solution phases, advanced thermodynamic models are required to capture inherent complexities, such as short-range ordering. This work utilizes first-principles calculations and CALPHAD modeling to investigate atomic environments in materials with industrial applications in catalysts and molten salts. Open-source software tools, PyCalphad and ESPEI, facilitate high-throughput CALPHAD modeling, uncertainty quantification, and model selection through Bayesian parameter estimation within the Markov Chain Monte Carlo approach. This work achieves atomic control of active-site ensembles in intermetallic catalysts for tailoring hydrogenation reactions, enables solution model selection and predictive modeling of critical characteristics in molten salts. Additionally, a template generator has been developed to allow users to customize thermodynamic models within PyCalphad. These advancements provide the community with extensive opportunities to comprehensively evaluate thermodynamic modeling with uncertainty quantification, accelerating materials design and discovery.

\setstretch{1}
\newpage
\setcounter{tocdepth}{3}
\tableofcontents

\newpage
\renewcommand{\listfigurename}{List of Figures}
\addcontentsline{toc}{chapter}{\listfigurename}
\listoffigures


\newpage
\renewcommand{\listtablename}{List of Tables}
\addcontentsline{toc}{chapter}{\listtablename}
\listoftables

\newpage
\chapter*{Acknowledgments}
\label{acknowledgments}
\addcontentsline{toc}{chapter}{\nameref{acknowledgments}}

I would like to express heartfelt thanks to my advisor, Dr. Zi-Kui Liu, for guiding me over the last five years. His passion for research has deeply influenced me. I will always be grateful for his continuous guidance and encouragement in my professional and personal development.

I would like to thank the current and former colleagues from Phases Research Lab, Dr. Shun-Li Shang, Dr. Yi Wang, Dr. Brandon Bocklund, Dr. Jorge Paz Soldan Palma, Dr. John Shimanek, Dr. Hui Sun, Dr. Adam Krajewski, Dr. Nigel Hew, Shuang Lin, Alexander Richter, Zhening Yang, Luke Myers, and Ricardo Amaral. I am so fortunate to have their guidance and have this opportunity to grow with them.

I would like to thank Dr. Michael Janik for supporting the intermetallics work and guiding my early development during my PhD journey. I am also grateful for the opportunity to complete two internships at Argonne National Laboratory. I would like to thank my advisor at Argonne National Laboratory Nuclear Science and Engineering Division, Dr. Shayan Shahbazi, who prompted me to deepen my understanding of molten salts. I would like to thank the support from Dr. Hojong Kim and Dr. Xiaofeng Guo for collaborations on molten salts projects.

I would like to thank my friends at State College and all the warm people I met at SCCAC. Thanks to my boyfriend Jiayang Wang for coming on this journey with me. I am so grateful to have Ori in my life, who lights up my world. Finally, I would like to express my deepest thanks to my family: my Mom, Mengjun Hu, and my Dad, Wei Gong, whose unwavering support will always be my greatest strength. I truly would not be where I am today without my parents.

This work was made possible by the financial support and training provided by the US Department of Energy (DOE) Office of Science, Office of Basic Energy Sciences, Catalysis Division via Award No. DE-SC0020147, DOE Nuclear Energy University Program (NEUP) via Award Nos. DE-NE0008945 and DE-NE0009288, and the Nuclear Energy Advanced Modeling and Simulation (NEAMS) program. Simulations were performed on the Roar supercomputer at the Pennsylvania State University's Institute for Computational and Data Sciences (ICDS) and the ACCESS supported by the National Science Foundation (NSF) with Grant No. ACI-1548562. Any opinions, findings, conclusions, or recommendations expressed in this publication are those of the author and do not necessarily reflect the views of the funding agencies.

%%%%%%%%%%%%%%%%%%%%%%%%%%%%%%%%%%%%%%%%

\newpage
\setlength\parindent{2em} %paragraph indentation
\setstretch{1.5}
\pagenumbering{arabic}

\chapter{Introduction} \label{sec:Introduction}

\section{CALPHAD modeling with model selection} \label{intro:sec:calphad}
The CALPHAD (Calculation of Phase Diagrams) approach \cite{liu2020computational, lukas2007computational} is a powerful computational thermodynamics methodology to predict the thermodynamic properties and phase behaviors of multi-component systems. By leveraging a combination of experimental data and simulation data, CALPHAD enables the construction of comprehensive thermodynamic databases that describe the Gibbs energy for each phase. This systematic approach facilitates the development of phase diagrams and other critical thermodynamic information essential for materials design, processing, and optimization. CALPHAD modeling relies on the accurate models of Gibbs energy functions for different phases. The selection of a thermodynamic model depends on the atomic environments within the phase, including factors such as chemical ordering and short-range interactions. Different phases exhibit varying degrees of atomic order and interaction complexities, requiring tailored modeling approaches to capture their unique behaviors accurately.  Various models, including the compound energy formalism (CEF) \cite{hillert1970regular}, the associate model \cite{sommer1982association}, the two-sublattice ionic model \cite{hillert1985two}, and the modified quasichemical model (MQM) \cite{pelton2018phase}, are employed to capture the complexities of solid and liquid phases. The selection of appropriate models to describe Gibbs energy functions of these phases is crucial, as it directly impacts the predicted thermodynamic properties and phase equilibrium.  However, systematically comparing different models remains a challenge due to the difficulty in quantifying model performance and the lack of tools that support the implementation of all thermodynamic models.

Recent advancements in computational tools and open-source software, such as PyCalphad \cite{otis2017pycalphad} and ESPEI \cite{bocklund2019espei}, have significantly enhanced the capabilities for CALPHAD modeling. The incorporation of Bayesian parameter estimation in thermodynamic modeling has enabled uncertainty quantification and statistical evaluation of model performance. These tools provide robust platforms for developing, validating, and implementing thermodynamic models, providing opportunities for high-throughput computational thermodynamics in the broad community.

\section{Challenges in intermetallic catalysts design} \label{intro:sec:catalysts}
Precise synthetic control of active site ensembles enables significant advancements in the design of selective heterogeneous catalysts. The active site can be thought of as the ensemble of atoms that directly catalyzes a reaction \cite{greeley2012active}. Intermetallic compounds (IMCs), characterized by their precise local atomic composition and structure (i.e., site occupancy), allow for systematic manipulation of the arrangement of multiple metals at active sites, provided the surface composition is consistent. The combination of active late transition metals, such as Pd, with a less catalytic second component, such as Zn, can be used to manipulate the active site ensemble, tuning the active site arrangement and electronic structure to facilitate the desired catalytic transformation while avoiding non-selective reactions. Several bimetallic IMCs have been identified that exhibit distinct catalytic properties compared to monometallic catalysts. For instance, Pd-Ga IMCs have been reported to show enhanced selectivity for acetylene semi-hydrogenation \cite{kovnir2007new, prinz2014adsorption}. Additionally, MgO-supported Ni-Ga IMCs have been investigated and demonstrated significantly higher selectivity for the semi-hydrogenation of phenylacetylene compared to pure Ni \cite{li2014nickel}.

Designing intermetallic catalysts involves addressing challenges related to thermodynamic stability and surface configuration of candidate IMCs. Ensuring the stability of these catalysts is crucial for both their design and processing, as variations in factors such as the composition of active metals can lead to phase transformations or decomposition, potentially undermining catalyst performance and longevity. Additionally, achieving a stable and well-defined surface configuration that retains the desired catalytic properties under operational conditions poses a significant challenge. A thorough understanding of the interplay between bulk thermodynamics and surface phenomena is essential for optimizing intermetallic catalysts. This requires comprehensive knowledge of phase diagrams and the ability to precisely control surface structures to ensure consistent performance and durability.

Determining phase stability and its variation with external conditions necessitates modeling the thermodynamic properties of all individual phases as functions of variables such as temperature and composition. The CALPHAD method is employed to model the Gibbs energies of both stable and metastable phases, using parameterized functions of temperature, composition, pressure, and internal degrees of freedom. This approach integrates experimental data with theoretical insights from density functional theory (DFT) calculations. By global minimization of Gibbs energy, this method allows for the determination of the distribution of active and non-active components, which in turn helps to identify stable bulk and surface configurations.

\section{Challenges in complex molten salts liquid modeling } \label{intro:sec:moltensalts}
Molten Salt Reactor (MSR) is one of the few game-changing concepts that use molten salts as solvents for dissolving nuclear fuels \cite{blander1964molten, abram2008generation, cottrell1955operation} for achieving high levels of reliability and efficiency as the nuclear reactor. The MSR utilizes a molten salt mixture, such as LiF-BeF$_2$-UF$_4$, in which fissile and fertile isotopes (e.g., $^{233}$U, $^{235}$U, $^{238}$U, and/or $^{239}$Pu) are dissolved. This mixture circulates continuously from the reactor core to the heat exchanger \cite{blander1964molten, leblanc2010molten}. A critical aspect of this system is its safety, which underscores the importance of carefully selecting appropriate molten salts \cite{benevs2013thermodynamic}.

Alkali and alkaline-earth metal fluorides, which can dissolve actinide fluorides like UF$_4$ and PuF$_3$, are central to MSR fuel salts. For instance, the $^7$LiF-BeF$_2$-ZrF$_4$-UF$_4$ fuel salt, with $^{235}$U, $^{233}$U, and/or $^{239}$Pu as fissile drivers, was used in the Molten Salt Reactor Experiment (MSRE) at Oak Ridge National Laboratory (ORNL) from 1965 to 1969 \cite{blander1964molten}. The coolant salt in the secondary loop was $^7$Li$_2$BeF$_4$. To date, substantial experimental data exist for simple coolant salt systems such as FLiNaK (the LiF-NaF-KF eutectics with its mole fraction around 0.465-0.115-0.420) and LiCl-NaCl-MgCl$_2$. However, data for chloride fuel salts, such as NaCl-UCl$_3$, NaCl-UCl$_3$-(Pu, TRU)Cl$_3$, and NaCl-MgCl$_2$-UCl$_3$, are limited \cite{mourogov2006potentialities}, positioning these chlorides as emerging model salts. Electrochemical pyroprocessing of used nuclear fuel with chloride melt matrices (e.g., LiCl-KCl) offers a promising option for the proliferation-resistant separation and recovery of fissile materials, particularly for high burn-up or fast reactor fuels where traditional solvent extraction methods may not be applicable \cite{blander1964molten}. The integration of pyroprocessing with reactor technology highlighted chloride-based molten salts as key materials for next-generation MSRs.

The CALPHAD method is extensively employed to predict the thermodynamic properties of molten salts. The primary thermodynamic databases for molten salts include the FactSage database \cite{bale2002factsage} and the open Molten Salt Thermodynamic Database (MSTDB-TC) \cite{ard2022development}. Despite these advancements, several challenges persist within the community: efficient high-throughput modeling of multicomponent molten salt systems, integrating modeling results from different database formats, ensuring the reliability and addressing the uncertainty of CALPHAD predictions, and selecting appropriate thermodynamic models for describing atomic environments in molten salts. The implementation of several thermodynamic models including the modified quasichemical model with quadrupled approximation (MQMQA) into the PyCalphad and ESPEI, has significantly enhanced high-throughput CALPHAD modeling by enabling uncertainty quantification and improved model selection. These developments are expected to provide more accurate and reliable predictions of critical molten salt properties.

\section{Executive Summary} \label{intro:sec:summary}
%%%%%%%%%%
First, Chapter \fullref{chap:method} introduces the first-principles calculations and CALPHAD method. First-principles calculations predict thermodynamic properties at both 0 K and finite temperatures, providing critical data to enhance the accuracy of CALPHAD modeling. Various thermodynamic models for the Gibbs energy function are presented, including the CEF model \cite{hillert1970regular}, the associate model \cite{sommer1982association}, the two-sublattice ionic model \cite{hillert1985two}, and the MQMQA model \cite{pelton2001modified}. The open-source software PyCalphad and ESPEI are utilized for computational thermodynamics. Additionally, this chapter discusses Bayesian parameter estimation used in the parameter optimization process, highlighting its role in uncertainty quantification and model selection.

Next, Chapter \fullref{chap:intermetallics} explores the application of CALPHAD modeling to intermetallic catalysts. It demonstrates how selecting the appropriate sublattice model for the $\gamma$-brass phase in the binary Pd-Zn system, as well as the ternary Pd-Zn-M (M=Cu, Ag, Au) systems, facilitates the investigation of site occupancy for active metals Pd, Cu, Ag, and Au in the $\gamma$-brass phase. The chapter also provides predictions regarding surface structure and active sites in intermetallic catalysts, aimed at optimizing the selectivity of hydrogenation reactions.

Chapter \fullref{chap:moltensalts} focuses on describing short-range ordering in complex molten salts using the CALPHAD method. The CALPHAD modeling aided by first-principles calculations are used in molten salts systems such as (LiF, NaF, KF, CrF$_2$)-CrF$_3$ and LiCl-KCl-LaCl$_3$. This chapter includes uncertainty quantification and propagation to assess the reliability of the modeling, as well as a detailed discussion of comparing various liquid models for the molten salts. Bayesian statistics are employed for model selection, providing insights into quantifying the performance of the models.

Chapter \fullref{chap:models} presents the enhancement of the applicability of PyCalphad through the integration of additional thermodynamic models. The universal quasichemical model (UNIQUAC) \cite{abrams1975statistical} is introduced and successfully implemented in PyCalphad, accompanied by thorough validation and demonstration. Additionally, a custom model template generator is developed to facilitate the efficient implementation of various thermodynamic models. This chapter also includes a demonstration of the application of this template generator in implementing the Peng-Robinson equation of state (PR EOS) \cite{peng1976new}.

Lastly, Chapter \fullref{chap:conclusion} provides a comprehensive summary of the current research on atomic environments in intermetallic catalysts and molten salts and future work directions. This chapter emphasizes the importance of selecting suitable thermodynamic models for an appropriate description of atomic environments in the phase and accurate predictions of thermodynamic properties. Bayesian statistics are employed for the modeling process, which enables model comparison and selection, ensuring robust and reliable results. The chapter also highlights the development of software features, including a custom model template generator, which is made available to the broader community to improve the efficiency and accessibility of computational thermodynamics.

\chapter{Computational methodology} \label{chap:method}
To investigate atomic environments in materials, theoretical simulations are essential for gaining insights into local structures and properties. This chapter will discuss two computational methods: first-principles calculations and CALPHAD modeling. First-principles calculations based on density functional theory (DFT) are used to predict stable structures and the thermochemical properties of materials. CALPHAD modeling integrates experimental data and simulation strengths to develop comprehensive thermodynamic descriptions of materials and further predict phase relations for materials design.

\section{First-principles calculations} \label{method:sec:firstprinciples}
DFT-based first-principles calculations are used to obtain thermodynamic properties at 0 K at finite temperatures through quasi-harmonic approximation (QHA). The Helmholtz energy $F(V,T)$ as a function of volume ($V$) and temperature ($T$) in terms of QHA can be determined by \cite{shang2010first}:
\begin{equation} \label{method:eq:qhaF}
    F\left(V,T\right)=E\left(V\right)+F_{el}\left(V,T\right)+F_{vib}(V,T)
\end{equation}
where the first term $E\left(V\right)$ is static energy at 0 K without the zero-point vibrational energy. In the present work, a four-parameter Birch-Murnaghan (BM4) equation of state (EOS) \cite{shang2010first} as shown in (\ref{method:eq:EOS}) is used to obtain equilibrium properties at zero external pressure (P = 0 GPa), including the static energy E$_0$, volume (V$_0$), bulk modulus (B$_0$) and its pressure derivate (B$^\prime$).
\begin{equation} \label{method:eq:EOS}
    E\left(V\right)=a_1+a_2V^{-2/3}+a_3V^{-4/3}+a_4V^{-2}
\end{equation}
where $a_1$, $a_2$, $a_3$, and $a_4$ are fitting parameters. The second term in (\ref{method:eq:qhaF}), $F_{el}\left(V,T\right)$, represents the temperature-dependent thermal electronic contribution \cite{wang2004thermodynamic}:
\begin{equation} \label{method:eq:Fel}
    F_{el}(V,T)=E_{el}(V,T)-T\:S_{el}(V,T)
\end{equation}
where $E_{el}$ and $S_{el}$ are the internal energy and entropy of thermal electron excitations, respectively, which can be obtained by the electronic density of states (DOS). Note that the thermal electronic contribution to Helmholtz free energy is negligible for non-metal, considering the Fermi level lies in the band gap. The third term in (\ref{method:eq:qhaF}), $F_{vib}(V,T)$, represents the vibrational contribution \cite{wang2004thermodynamic, van2002effect} given by:
\begin{equation} \label{method:eq:Fvib}
    F_{vib}(V,T)=k_BT\sum_{q}\sum_{j}\ln{\left\{2\sinh{\left[\frac{\hbar\omega_j(q,V)}{2k_BT}\right]}\right\}}
\end{equation}
where $\omega_j\left(q,V\right)$ represents the frequency of the $j_{th}$ phonon mode at wave vector $q$ and volume $V$, and $\hbar$ the reduced Plank constant. 

The Vienna Ab initio Simulation Package (VASP) \cite{kresse1996efficient} is used for all DFT-based calculations in the present work. Detailed settings of first-principles calculations for intermetallic catalysts and molten salts are discussed in Section \ref{intermetallics:ssec:PdZnmodel} and \ref{moltensalts:ssec:FLiNaKCrmodel}.

\section{CALPHAD modeling} \label{method:sec:calphad}
In the CALPHAD method, the general form of Gibbs energy of a phase can be expressed as:
\begin{equation} \label{method:eq:Gm}
    G_m=\:^{srf}G_m-T\:^{cnf}S_m +\:^{phys}G_m +\:^{xs}G_m
\end{equation}
where $^{srf}G_m$ represents the surface of reference Gibbs energy, $T\:^{cnf}S_m$ is the ideal configurational entropy contribution to the Gibbs energy, $^{phys}G_m$ represents the contribution of physical models to the Gibbs energy, such as magnetic transitions, and $^{xs}G_m$ is the excess Gibbs energy describing the remaining part of the real Gibbs energy \cite{lukas2007computational}. Considering the atomic environments of each phase, various models are employed to describe these contributions to the Gibbs energy in the present work.

\subsection{Compound energy formalism} \label{method:ssec:CEF}
%Introducting paragraph
A crystalline solid phase may have crystallographically different sublattices and the constituents may prefer different sublattices. This is a form of long-range order (LRO) and must be included in the modeling \cite{lukas2007computational}. For the solution phase, where phase can vary with composition, compound energy formalism (CEF) \cite{hillert1970regular, sundman1981regular} is used to describe the Gibbs energy. In the CEF, the selection of sublattices typically corresponds to the crystallographic sublattices associated with different Wyckoff positions. For complex phases requiring many sublattices, simplifications are often made to reduce the number of sublattices. Each sublattice may be described with a stoichiometric coefficient and contain any number of species, i.e. atoms, molecules, ions, or vacancies. 

For example, a phase with two sublattices can be represented by the formula:
\begin{equation} \label{method:eq:CEFphasemodel}
    ({\rm A,B})_k({\rm C,D})_l
\end{equation}
where A and B are mixed on the first sublattice (I), and C and D are mixed on the second sublattice (II). $k$ and $l$ are the stoichiometric coefficients. The constitution of the phase is described by site fraction, $y$, e.g. $y_{J}^{(s)}$ represents the mole fraction of species $J$ on the $s$ sublattice. Thus, the summation of site fraction $y_{J}$ of species over each sublattice equals 1. In the limit, there will be only one type of species on each sublattice, this configuration is called endmember. Endmembers of the phase with model (\ref{method:eq:CEFphasemodel}) are (A)${_k}$(C)${_l}$, (A)${_k}$(D)${_l}$, (B)${_k}$(C)${_l}$, and (B)${_k}$(D)${_l}$. The surface of reference energy term in (\ref{method:eq:Gm}) can be expressed as:
\begin{equation} \label{method:eq:CEFGsrf}
    ^{srf}G_m=\sum y_i^{({\rm I})}y_j^{({\rm II})}\:^{o}G_{i:j}
\end{equation}
where $^{o}G_{i:j}$ is the Gibbs energy of the endmember compound ($i$)${_k}$($j$)${_l}$ (here $i$ = A and B, $j$ = C and D). The ideal configurational entropy contribution to the Gibbs energy is described as:
\begin{equation} \label{method:eq:CEFScnf}
    \begin{aligned}
        -T\:^{cnf}S_m&=RT\sum n_s \sum y_J^{(s)} \ln (y_J^{(s)})\\
        &=RT(k\sum_{i={\rm A,B}} y_i^{({\rm I})}\ln (y_i^{({\rm I})})+l\sum_{j={\rm C,D}} y_j^{({\rm II})}\ln (y_j^{({\rm II})}))
    \end{aligned}
\end{equation}
where $R$ is the ideal gas constant, $n_s$ is the stoichiometric coefficient of the sublattice $s$ (here $n_s$ = $k$ and $l$). The excess Gibbs energy is described as:
\begin{equation} \label{method:eq:CEFGxs}
    ^{xs}G_m=y_{\rm A}^{({\rm I})}y_{\rm B}^{({\rm I})}\sum_{j={\rm C,D}} y_j^{({\rm II})} L_{{\rm A,B}:j}+y_{\rm C}^{({\rm II})}y_{\rm D}^{({\rm II})}\sum_{i={\rm A,B}} y_i^{({\rm I})} L_{i:{\rm C,D}}+y_{\rm A}^{({\rm I})}y_{\rm B}^{({\rm I})}y_{\rm C}^{({\rm II})}y_{\rm D}^{({\rm II})}L_{{\rm A,B}:{\rm C,D}}
\end{equation}
where $L$ is the interaction parameters, which can be expanded in a Redlich-Kister polynomial \cite{redlich1948algebraic}. For example, binary interaction term $L_{{\rm A,B}: {\rm C}}$ in (\ref{method:eq:CEFGxs}) can be expressed as:
\begin{equation} \label{method:eq:CEFLabc}
    L_{{\rm A,B}: {\rm C}} = \sum_{v=0}(y_{\rm A}^{({\rm I})}-y_{\rm B}^{({\rm I})})^v\:^vL_{{\rm A, B:C}}
\end{equation}
forming the polynomial basis with increasing orders of $v$. The reciprocal interaction parameters are described by:
\begin{equation} \label{method:eq:CEFLabcd}
    L_{{\rm A,B}:{\rm C,D}}=^0L_{{\rm A,B}:{\rm C,D}}+(y_{\rm A}^{({\rm I})}-y_{\rm B}^{({\rm I})})\:^1L_{{\rm A,B}:{\rm C,D}}+(y_{\rm C}^{({\rm II})}-y_{\rm D}^{({\rm II})})\:^2L_{{\rm A,B}:{\rm C,D}}
\end{equation}
The temperature dependence of these excess parameters is often modeled by:
\begin{equation} \label{method:eq:CEFLT}
    ^vL= b_1 + b_2\:T +b_3T\ln T
\end{equation}
where $b_1$, $b_2$, and $b_3$ are adjustable parameters.

There are several cases of the CEF worth mentioning. A sublattice model
that has only one constituent in every sublattice will have no internal degrees of freedom
with a fixed composition. It is used as a model for the stoichiometric compound (or the line compound). In addition, the constituents on a sublattice may be a complex species instead of a pure element, for example, using an \textit{associate} to represent clusters of short-range ordering in the liquid, commonly referred to as the associate model \cite{sommer1982association} and will be discussed in Section \ref{method:sssec:assm}.

\subsection{Thermodynamic models for liquid} \label{method:ssec:liqmodels}
%Introducting paragraph
In the liquid phase, the constituents have no fixed environment and the number of nearest neighbors can vary. For the simple case, one sublattice model as shown in \ref{method:ssec:CEF} can be used to describe the random mixing of constituents in the liquid. However, for liquids with complex phenomena, such as short-range ordering (SRO) or ionic characteristics, alternative models are used to describe the Gibbs energy.

\subsubsection{Associate model} \label{method:sssec:assm}
%Introducting paragraph
To account for SRO, fictitious constituents, or \textit{associates}, are introduced in the model. For an experimentally determined sharp minimum in the enthalpy curve, the selection of stoichiometry of the associate usually corresponds to the composition of the minimum or is based on experimental observation. For example, for a binary A-B liquid with AB associate introduced, the surface of reference Gibbs energy can be described as:
\begin{equation} \label{method:eq:assmGsrf}
    ^{srf}G_m=\sum_{i={\rm A,AB,B}} y_i\:^{o}G_{i}
\end{equation}
where the site fraction $y_i$ is used here to denote that the constituent fractions (A, AB, and B) are not the same as the mole fractions of the components (A and B). $^{o}G_{i}$ is the Gibbs energy of the constituent $i$. The ideal configurational entropy contribution can be expressed as:
\begin{equation} \label{method:eq:assmScnf}
    -T\:^{cnf}S_m=RT\sum_{i={\rm A,AB,B}}y_i\ln y_i
\end{equation}
where the summation of $i$ is over all the constituents (here $i$ = A, AB, and B). The excess Gibbs energy can be modeled as:
\begin{equation} \label{method:eq:assmGex}
    ^{xs}G_m=y_{\rm A}y_{\rm AB}L_{\rm A,AB}+y_{\rm AB}y_{\rm B}L_{\rm AB,B}+y_{\rm A}y_{\rm B}L_{\rm A,B}+y_{\rm A}y_{\rm AB}y_{\rm B}L_{\rm A,AB,B}
\end{equation}
where the interaction terms $L$ can also be expanded with Redlich-Kister polynomial \cite{redlich1948algebraic} as in (\ref{method:eq:CEFLT}).

This model is regarded as a convenient way of formally representing the thermodynamic effects of SRO \cite{sommer1982association} and it has been extensively applied to binary metallic melts. However, sometimes there are no physical indications of real associates and this may result in failure predictions when extrapolation into the higher-order system.

\subsubsection{Two-sublattice ionic model} \label{method:sssec:ionic}
%Introducting paragraph
A different model for treating SRO in liquid solutions was developed
by Hillert et al. \cite{hillert1985two}. In a liquid, there are no distinguished sites for anions or cations, but the mathematical formalism using mixing on two different sites gives good agreement with experimental information \cite{lukas2007computational}. In addition to cations and anions in the liquid, hypothetical vacancies are introduced on the anion sublattice for a liquid with only cations, i.e., a metallic liquid. Neutral species can also be included in this model for non-metallic systems. The model can be written as:
\begin{equation} \label{method:eq:ionicM}
    ({\rm C}_i^{v_i+})_P({\rm A}_j^{v_j-}, {\rm Va}, {\rm B}_k^0)_Q
\end{equation}
where C represents the cation, A is the anion, Va is the vacancy, B is the neutral species, $v$ denotes the charge of the ion, $i$, $j$, and $k$ are constituents. $P$ and $Q$ are the number of sites that will vary with the composition in order to maintain charge neutrality. $P$ and $Q$ can be calculated as:
\begin{equation} \label{method:eq:ionicPQ}
    \begin{aligned}
        & P=\sum_jv_jy_{{\rm A}_j}+Qy_{\rm Va}\\
        & Q=\sum_iv_iy_{{\rm C}_i}
    \end{aligned}
\end{equation}
where $y_i$ represents the constituent fraction of constituent $i$. The Gibbs energy in this model can be expressed as:
\begin{equation} \label{method:eq:ionicGsrf}
    ^{srf}G_m=\sum_i\sum_jy_{{\rm C}_i}y_{{\rm A}_j}\:^{o}G_{{\rm C}_i:{\rm A}_j}+Qy_{\rm Va}\sum_iy_{{\rm C}_i}\:^{o}G_{{\rm C}_i}+Q\sum_ky_{{\rm B}_k}\:^{o}G_{{\rm B}_k}
\end{equation}
\begin{equation} \label{method:eq:ionicScnf}
    -T\:^{cnf}S_m=RT(P\sum_iy_{{\rm C}_i}\ln (y_{{\rm C}_i})+Q(\sum_jy_{{\rm A}_j}\ln (y_{{\rm A}_j})+y_{\rm Va}\ln (y_{\rm Va})+\sum_ky_{{\rm B}_k}\ln (y_{{\rm B}_k})))
\end{equation}
\begin{equation} \label{method:eq:ionicGxs}
    \begin{aligned}
        ^{xs}G_m=&\sum_{i_1}\sum_{i_2}\sum_jy_{i_1}y_{i_2}y_{j}L_{i_1i_2:j}+\sum_{i_1}\sum_{i_2}y_{i_1}y_{i_2}y_{\rm Va}L_{i_1i_2:{\rm Va}}\\&+\sum_{i}\sum_{j_1}\sum_{j_2}y_{i}y_{j_1}y_{j_2}L_{i:j_1j_2}+\sum_{i}\sum_{j}y_iy_jy_{\rm Va}L_{i:j{\rm Va}}\\&+\sum_{i}\sum_{j}\sum_{k}y_{i}y_{j}y_{k}L_{i:jk}+\sum_{i}\sum_{k}y_{i}y_{k}y_{\rm Va}L_{i:{\rm Va}k}+\sum_{k_1}\sum_{k_2}y_{k_1}y_{k_2}L_{k_1k_2}
    \end{aligned}
\end{equation}
where in (\ref{method:eq:ionicGsrf}), $^{o}G_{{\rm C}_i:{\rm A}_j}$ is the Gibbs energy of formation for $v_i+v_j$ moles of atoms of liquid ${\rm C}_i{\rm A}_j$. $^{o}G_{{\rm C}_i}$ and $^{o}G_{{\rm B}_k}$ are the Gibbs energies of formation per mole of atoms of liquids ${\rm C}_i$ and ${\rm B}_k$. Interaction terms $L$ in (\ref{method:eq:ionicGxs}) can be expanded as in (\ref{method:eq:CEFLT}).

\subsubsection{Modified quasichemical model with quadruplets approximation} \label{method:ssec:mqmqa}
The mathematical expression of the modified quasichemical model (MQM), which only considers pairs of elements, will be discussed first before introducing the modified quasichemical model with quadruplets approximation (MQMQA), which is based on the quadruplet structures. If one considers a hypothetical A-B system, the model can be mathematically expressed by distributing A and B atoms over lattice sites in a quasi-lattice. An important distinction of this model is that its internal degrees of freedom are based on the number of pairs (bonds) between atoms or species rather than atoms occupying a sublattice. In this quasi-lattice scenario, the A-B pair interaction energy is described as:
\begin{equation} \label{method:eq:mqmpair}
    \left({\rm A-A}\right)+\left({\rm B-B}\right)=2\left({\rm A-B}\right);\:\:\:\mathrm{\Delta}g_{\rm AB}
\end{equation}
where $\mathrm{\Delta}g_{AB}$ is the Gibbs energy of the formation of two A-B pair bonds in the system. This energy formation physically represents the energy favorability for atoms of element A and atoms of element B to be next to each other in the lattice. 

The composition of the overall system is linked to the number of pairs in the system through the coordination number, which is mathematically expressed as:
\begin{equation} \label{method:eq:mqmZa}
    n_{\rm A}Z_{\rm A}={2n}_{\rm AA}+n_{\rm AB}
\end{equation}
where $n_{\rm A}$, $Z_{\rm A}$, $n_{\rm AA}$, and $n_{\rm AB}$ are the number of moles of element A, first-nearest-neighbor (FNN) coordination number of element A, and the number of moles of A-A and A-B pairs, respectively. In the MQM model, the coordination number of the elements are values used to denominate the composition of maximum short-range ordering (SRO) of the solution being described. Another important value in the quasichemical model is the site equivalent fraction, defined as follows:
\begin{equation} \label{method:eq:mqmYa}
    Y_A=\frac{n_{\rm A}Z_{\rm A}}{(Z_{\rm A}n_{\rm A}+Z_{\rm B}n_{\rm B})}=\frac{X_{\rm A}Z_{\rm A}}{(Z_{\rm A}X_{\rm A}+Z_{\rm B}X_{\rm B})}=X_{\rm AA}+\frac{1}{2}X_{\rm AB}
\end{equation}
where $X_{\rm A}$ is the mole fraction of A, $X_{\rm AA}$ is the pair fraction of A-A bonds, and $X_{\rm AB}$ is the pair fraction of A-B bonds. Pair fractions in this expression are simply the number of moles of a specific pair divided by the total number of moles of pairs considered in the system. These values are used to describe the configurational entropy of the model, which for the hypothetical A-B system is expressed as:
\begin{equation} \label{method:eq:mqmScnf}
    ^{cnf}S_m=-R(n_{\rm A}\ln X_{\rm A}+n_{\rm B}\ln X_{\rm B}+n_{\rm AA}\ln \frac{X_{\rm AA}}{Y_{\rm A}^2}+n_{\rm BB}\ln \frac{X_{\rm BB}}{Y_{\rm B}^2}{+n}_{\rm AB}\ln \frac{X_{\rm AB}}{2Y_{\rm A}Y_{\rm B}})
\end{equation}
When the $\mathrm{\Delta}g_{AB}=0$, the expressions that are based on $X_{ii}$ and $Y_i$ cancel each other, and (\ref{method:eq:mqmScnf}) reduces to the expression for ideal configurational entropy. 

Pelton later developed a quadruplet approximation in which 2 sublattices would be used to simultaneously describe SRO for FNN and second-nearest-neighbors (SNN) \cite{pelton2018phase, poschmann2021recent}. The two sublattices from MQMQA separate the cations from the anions (FNN) with two cations and two anions in each (SNN). For further discussion of the MQMQA model, a hypothetical A-B-X-Y system where A and B are cations and X, and Y are anions will be considered. Three different categories of quadruplets are formed, i.e., the unary quadruplets [A$_2$X$_2$, B$_2$X$_2$, A$_2$Y$_2$, and B$_2$Y$_2$], the binary quadruplets [ABX$_2$ ABY$_2$, A$_2$XY, and B$_2$XY], and the reciprocal quadruplet [ABXY]. 

In MQMQA, the internal degree of freedom is the quadruplet rather than the pair, and therefore adjustments to previous mathematical expressions were employed. For instance, the relationship between quadruplet fractions, coordination numbers, and moles of pure elements of the total system is represented as:
\begin{equation} \label{method:eq:mqmna}
    n_{\rm A}=\frac{{2n}_{{\rm A}_2:{\rm X}_2}}{Z_{{\rm A}_2:{\rm X}_2}^{\rm A}}+\frac{{2n}_{{\rm A}_2:{\rm Y}_2}}{Z_{{\rm A}_2:{\rm Y}_2}^{\rm A}}+\frac{{2n}_{{\rm A}_2:{\rm XY}}}{Z_{{\rm A_2:XY}}^{\rm A}}+\frac{n_{\rm AB:X_2}}{Z_{\rm AB:X_2}^{\rm A}}+\frac{n_{\rm AB:Y_2}}{Z_{\rm AB:Y_2}^{\rm A}}+\frac{n_{\rm AB:XY}}{Z_{\rm AB:XY}^{\rm A}}
\end{equation}
where $n_{\rm quadruplet}$ and $Z_{\rm quadruplet}^{\rm A}$ are the molar fraction and coordination number of element A of any specified quadruplet (SNN coordination number), respectively. Additionally, the relationship between equivalent site fractions and quadruplet fractions can be written as:
\begin{equation} \label{method:eq:mqmqaya}
    Y_{\rm A}=n_{\rm A_2:X_2}+n_{\rm A_2:Y_2}+n_{\rm A_2:YX}+{0.5n}_{\rm AB:X_2}+0.5n_{\rm AB:Y_2}+{0.5n}_{\rm AB:XY}
\end{equation}

The SNN coordination number is a model parameter used to describe SRO \cite{pelton2018phase} in the liquid phase. With the SNN coordination number of A in the [ABXY] reciprocal quadruplet denoted by $Z_{\rm AB:XY}^{\rm A}$, the condition of charge neutrality for this quadruplet is maintained as follows:
\begin{equation}\label{method:eq:mqmqaZ}
    \frac{q_{\rm A}}{Z_{\rm AB:XY}^{\rm A}}+\frac{q_{\rm B}}{Z_{\rm AB:XY}^{\rm B}}=\frac{q_{\rm X}}{Z_{\rm AB:XY}^{\rm X}}+\frac{q_{\rm Y}}{Z_{\rm AB:XY}^{\rm Y}}
\end{equation}
where $q_i$ is the charges of ion $i$ (=A, B, X, or Y) and $Z_i$ are ion’s corresponding SNN coordination number.

The surface of reference Gibbs energy can be expressed as:
\begin{equation} \label{method:eq:mqmqaGsrf}
    ^{srf}G=(\sum_{i={\rm A,B}}\:\sum_{j={\rm A,B};\:j\geq i}\:\sum_{m={\rm X,Y}}\:\sum_{n={\rm X,Y};\:n\geq m}X_{ij:mn}\:^oG_{ij:mn})/N_{tot}
\end{equation}
where $X_{ij:mn}$, $^oG_{ij:mn}$, and $N_{tot}$ are the quadruplet mole fraction, the standard Gibbs energy per mole of the quadruplet [$ijmn$], and the total number moles of elements of the system, respectively \cite{pelton2018phase}. The configurational entropy per mol-atom is approximately represented by,
\begin{equation}\label{method:eq:mqmqaScnf}
    \begin{aligned}
        -^{cnf}S/R=&\sum_{i={\rm A,B}}{X_i\ln X_i}+\sum_{m={\rm X,Y}}{X_m\ln X_m}+\sum_{i={\rm A,B}}\sum_{m={\rm X,Y}}{X_{i:m}^{tot}\ln{\left(\frac{X_{i:m}}{Y_iY_m}\right)}}\\&+\sum_{i={\rm A,B}}\:\sum_{j={\rm A,B};\:j\geq i}\:\sum_{m={\rm X,Y}}\:\sum_{n={\rm X,Y};\:n\geq m}X_{ij:mn}^{tot}\ln{\left(\frac{X_{ij:mn}}{\frac{fX_{i:m}^{e_1}X_{j:m}^{e_1}X_{i:n}^{e_1}X_{j:n}^{e_1}}{Y_i^{e_2}Y_j^{e_2}Y_m^{e_2}Y_n^{e_2}}}\right)}
    \end{aligned}
\end{equation}
where $X_i$ is the mole fraction of species i, $X_{i:m}$ the mole fraction of the pair $i:m$, $X_{i:m}^{tot}$ the number of moles pair i:m normalized to total number of moles of elements in the system, $X_{ij:mn}^{tot}$ the number of moles quadruplet $ij:mn$ normalized to total number of moles of elements in the system. The third term on the right side of (\ref{method:eq:mqmqaScnf}) represents the contribution from FNN, where $Y_i$ is the coordination equivalent site fraction and can be calculated as follows:
\begin{equation} \label{method:eq:mqmqaYi}
    Y_i=\frac{Z_iX_i}{\sum_{j}{Z_jX_j}}
\end{equation}

Note that the third and the fourth terms in (\ref{method:eq:mqmqaScnf}) are based on two different derivations from Pelton \cite{pelton2018phase}. The original formalism \cite{pelton2001modified}, which is labeled as SUBG in the software FactSage, kept $\zeta$ (a parameter that relates how many quadruplets emanate from a specific pair) as a constant value. However, the new formalism derived by Lambotte and Chartrand \cite{lambotte2011thermodynamic}, denoted as SUBQ in FactSage, allows $\zeta$ to be a function of composition. This change results in a modified coordination equivalent site fraction, $F_i$ as:
\begin{equation} \label{method:eq:mqmqaFi}
    F_i=\sum_{m} X_{i:m}
\end{equation}
and replaces the $Y_i$ term in (\ref{method:eq:mqmqaScnf}). These two different formalisms affect the last term in the configurational entropy expression in (\ref{method:eq:mqmqaScnf}) as well. The last term in (\ref{method:eq:mqmqaScnf}) represents the contribution from SNN, where $f$ is a factor that equals 1 for unary quadruplets, 2 for binary quadruplets, and 4 for reciprocal quadruplets. Note that the exponents $e_1$ and $e_2$ in the denominator of the fourth term in (\ref{method:eq:mqmqaScnf}) correspond to the symbols SUBG and SUBQ used by FactSage \cite{bale2002factsage}, with $e_1$ = 1 and $e_2$ = 1 for the case of SUBG; and $e_1$ = 0.75 and $e_2$ = 0.5 for SUBQ.

The excess Gibbs energy $^{xs}G$ relates to the Gibbs energy of the quadruplet formation. For example, the following reaction forms quadruplets:
\begin{equation} \label{method:eq:mqmqareac}
    \left({\rm A_2X_2}\right)+\left({\rm B_2X_2}\right)=2\left({\rm ABX_2}\right);\:\:\:\mathrm{\Delta}g_{\rm AB:X_2}^{ex}
\end{equation}
where ${\Delta}g_{\rm AB:X_2}^{ex}$ of this equation indicates the Gibbs energy change when forming the quadruplet [ABX$_2$]. If ${\Delta}g_{\rm AB:X_2}^{ex}=0$, the ideal random mixing occurs. If ${\Delta}g_{\rm AB:X_2}^{ex}<0$, the reaction of (\ref{method:eq:mqmqareac}) moves to the right, and SRO with the [ABX$_2$] SNN pairs is promoted \cite{pelton2018phase}. Pelton \cite{pelton2018phase} indicated the necessity to describe multicomponent systems with appropriate interaction parameters and take into account what chemical groups the elements of interest when extrapolating into higher-order composition space. This was done by incorporating the Kohler-Toop extrapolation model, taking into consideration different extrapolations when elements within the same sublattice are not in the same chemical group. In this case the variables $\xi_{ij:{\rm X}_2}$ and $\chi_{12:{\rm X}_2}$ are introduced to the excess Gibbs energy expression, where the $\xi_{ij:{\rm X}_2}$ is defined as follows:
\begin{equation} \label{method:eq:mqmqaxi}
    \xi_{ij:{\rm X}_2}=Y_i+\sum_{k}Y_k
\end{equation}
where $k$ is the component in the $(i-j-k):{\rm X}_2$ pseudoternary system with $j$ being the asymmetrical component. The $\chi_{12:{\rm X}_2}$is defined as follows:
\begin{equation} \label{method:eq:mqmqachi}
    \Delta \chi_{12:{\rm X}_2}=\frac{\sum_{{\rm A}=1,k}\sum_{{\rm B}=1,k}{(X_{\rm AB:X_2}+\sum_{\rm Y\neq X}{0.5X_{\rm AB:XY})}}}{\sum_{{\rm A}=1,2,k,l}\sum_{{\rm B}=1,2,k,l}{(X_{\rm AB:X_2}+\sum_{\rm Y\neq X}{0.5X_{\rm AB:XY})}}}
\end{equation}
where $k$ represents all elements in the asymmetrical $(1-2-k):{\rm X}_2$ pseudoternary system with 2 being the asymmetrical component, and $l$ represents all asymmetrical $(1-2-l):{\rm X}_2$ pseudoternary systems with 1 being the asymmetrical component. By considering these variables, the expression of ${\Delta}g_{\rm AB:X_2}^{ex}$ is obtained as follows:
\begin{equation}
    \begin{aligned}
        {\Delta}g_{\rm AB:X_2}^{ex}=&(\Delta g^o+\sum_{\left(i+j\geq1\right)}{\chi_{\rm AB:X_2}^i\chi_{\rm BA:X_2}^jg_{\rm AB:X_2}^{ij}}\\&+\sum_{i\geq 0, j\geq 0, k\geq 1}\chi_{\rm AB:X_2}^i\chi_{\rm BA:X_2}^j(\sum_{l}{\frac{g_{{\rm AB}\left(l\right){:\rm X}_2}^{ijk}X_{l:{\rm X}}}{Y_{\rm X}}\left(1-\xi_{\rm AB:X_2}-\xi_{\rm BA:X_2}\right)^{k-1}}\\&+\sum_m\frac{g_{{\rm AB}\left(m\right):{\rm X}_2}^{ijk}X_{m:{\rm X}}}{Y_{\rm X}\xi_{\rm BA:X_2}}\left(1-\frac{X_{\rm B:X}}{Y_{\rm X}\xi_{\rm BA:X_2}}\right)^{k-1}\\&+\sum_{n}{\frac{g_{{\rm AB}\left(n\right):{\rm X}_2}^{ijk}X_{n:X}}{Y_{\rm X}\xi_{\rm AB:X_2}}\left(1-\frac{X_{\rm A:X}}{Y_{\rm X}\xi_{\rm AB:X_2}}\right)^{k-1}+\sum_{\rm Y\neq X}{Y_{\rm Y}\left(1-Y_{\rm X}\right)^{k-1}}})/N_{tot}
    \end{aligned}
\end{equation}
where the first two terms on the right-hand side are binary expressions, and the rest terms describe ternary interactions with symmetrical and asymmetrical considerations for the components being mixed. The first of the three ternary interactions represents the scenario when either component $k$ is the asymmetrical component or when all components are symmetrical. The second term is when species B is the asymmetrical component in the system. The third term is when species A is the asymmetrical component in the system. The final term represents a reciprocal ternary interaction parameter to describe the interaction between [AB:X$_2$] quadruplet when component Y is present. The whole expression is then normalized to Joules per mol-atom by dividing with the total number of elements in the system, $N_{tot}$. 

\subsection{Open-source software} \label{method:ssec:tools}
The tools used for computational thermodynamics are open-source software PyCalphad \cite{otis2017pycalphad} and ESPEI \cite{bocklund2019espei}. PyCalphad \cite{otis2017pycalphad} is a Python library for designing thermodynamic models, calculating phase diagrams, and investigating phase equilibria using the CALPHAD method. ESPEI \cite{bocklund2019espei} is designed for evaluating model parameters with PyCalphad as the computational engine. Model parameters are evaluated in two steps using ESPEI. The first step is parameter generation. In this step, the thermochemical data from DFT-based first-principles calculations with all internal degrees of freedom specified, such as site fractions in each sublattice, are used to select the number of parameters and evaluate their values. The experimental thermochemical data can also be used in the first step if their internal degrees of freedom are specified, such as stoichiometric compounds or fully random solutions. This is because the minimization of Gibbs energy for the internal variables is not performed in the first step. In the second step, all model parameters are simultaneously optimized through the Bayesian parameter estimation using a Markov Chain Monte Carlo (MCMC) method \cite{bocklund2019espei} (see Section \ref{method:ssec:Bayesian}). The input data for the second step refining parameters are primarily the experimental phase equilibrium information including two or more co-existing phases and equilibrium thermochemical data. The statistical distributions of model parameters are evaluated from the samples during MCMC optimization based on the Metropolis criteria \cite{bocklund2019espei}. In addition to parameter optimization, uncertainty quantification (UQ) of model parameters and propagation (UP) to calculate thermodynamic properties and phase stability from the models are enabled by PDUQ (Phase Diagram Uncertainty Quantification) \cite{paulson2019quantified}. PDUQ relies on the PyCalphad for predicting thermodynamic properties of interest and ESPEI for Bayesian samples to leverage the distribution of model parameters and estimate uncertainties based on the estimated Gaussian distribution of input data uncertainty \cite{paulson2019quantified}. 

In order to model complex liquids such as molten salts, MQMQA has been implemented into PyCalphad for thermodynamic calculations. This endeavor facilitates the development of an MQMQA-based thermodynamic database with UQ and UP using ESPEI. A new database structure based on Extensible Markup Language (XML) is proposed for ESPEI evaluation of MQMQA model parameters \cite{palma2023thermodynamic}. This implementation hence offers an open-source capability for performing CALPHAD modeling for complex liquids with SRO using various models including associate model, two-sublattice ionic model, and MQMQA plus a new XML database structure.

\section{Bayesian parameter estimation} \label{method:ssec:Bayesian}
ESPEI \cite{bocklund2019espei} uses Bayesian parameter estimation to optimize model parameters \cite{gelman1995bayesian}:
\begin{equation}\label{method:eq:Bayes}
    p\left(\theta\middle| D,\ M\right)=\frac{p\left(D\middle|\theta,\ M\right)p\left(\theta\middle| M\right)}{p\left(D\middle| M\right)}
\end{equation}
where $\theta$ are the model parameters, $M$ the model, and $D$ the input data for parameter optimization. In (\ref{method:eq:Bayes}), the posterior $p\left(\theta\middle| D,\ M\right)$ is the probability of model parameters conditioned on data, the likelihood $p\left(D\middle|\theta,\ M\right)$ is the probability that the data are described by parameters, the prior $p\left(\theta\middle| M\right)$ contains the domain knowledge in the probability distribution of each parameter, and the marginal likelihood (or evidence) $p\left(D\middle| M\right)$ is the probability of data being generated by the model. The evaluation of evidence $p\left(D\middle| M\right)$ is often usually difficult and computationally expensive. Hence, Bayes’ theorem can be expressed proportionally:
\begin{equation} \label{method:eq:Bayespropotion}
    p\left(\theta\middle| D,\ M\right)\propto p\left(D\middle|\theta,\ M\right)p\left(\theta\middle| M\right)
\end{equation}
ESPEI optimize parameters by obtain posterior probability $p\left(\theta\middle| D,\ M\right)$ numerically using MCMC \cite{bocklund2019espei}. Naturally, it is desired to minimize the residual between the expected data and the current value. This is equivalent to maximum likelihood estimation, which means under the current assumed statistical model, the expected data is most probable. 

The Metropolis-Hasting criteria \cite{metropolis1953equation} is applied to determine whether to accept newly proposed parameters:
\begin{equation} \label{method:eq:BayesMH}
    p_{accept}=min\left(\frac{p_{proposed}}{p_{current}},\ 1\right)
\end{equation}
where $p_{accept}$ is the probability of accepting newly proposed parameters, $p_{current}$ is the posterior probability for the current set of parameters and $p_{proposed}$ is the posterior probability for the newly proposed parameters. It accepts parameters that increase probability while still accepting parameters that decrease the probability with a chance $\frac{p_{proposed}}{p_{current}}$. Acceptance of parameters that decrease the posterior probability systematically constructs the set of parameters that describe the posterior distribution.

Bayesian statistics employed in parameter optimization provide a strategy for model selection for CALPHAD modeling \cite{paulson2019bayesian, honarmandi2019bayesian}. Bayes factor usually suggests which model is more favorite by the data, which can be evaluated from the ratio of marginal likelihoods for two competing models:
\begin{equation} \label{method:eq:BayesFactor}
    K=\frac{p\left(D\middle| M_1\right)}{p\left(D\middle| M_2\right)}
\end{equation}
The marginal likelihood has the desirable qualities of rewarding models that match the data well and penalizing models that are overly complex (i.e., have too many degrees of freedom or parameters). The marginal likelihood is determined by:
\begin{equation} \label{method:eq:BayesEvd}
    p\left(D\middle| M\right)=\int_{\Omega_\theta}^{\ }{p\left(D\middle|\theta,\ M\right)p\left(\theta\middle| M\right)d\theta}
\end{equation}
where $\Omega_\theta$ represents the complete parameter space. The evaluation of marginal likelihood requires computation of an integral with dimension given by the number of parameters, which is typically high-dimensional. The evaluation of the marginal likelihood $p\left(D\middle| M\right)$ is usually difficult and computationally expensive. The harmonic mean estimator was hence proposed by Newton and Raftery \cite{newton1994approximate} to estimate the marginal likelihood: 
\begin{equation} \label{method:eq:BayesHME}
    p\left(D\middle| M\right)\approx(\frac{1}{N}\sum_{i=1}^{N}{p\left(D\middle|\theta_i,\ M\right)}^{-1})^{-1}
\end{equation}
where $\theta_i$ are samples from the parameters’ prior $p\left(\theta\middle| M\right)$. Likelihood values can be obtained from the ESPEI MCMC output, which provides a statistical comparison of liquid models through the Bayes factor. 

\section{Summary} \label{method:ssec:summary}
%Introducting paragraph
In this work, first-principles calculations are utilized to provide critical thermochemical data for various phases. CALPHAD modeling is performed using PyCalphad and ESPEI, offering several significant advantages: i) the extensive scientific Python language ecosystem, ii) high-throughput CALPHAD modeling based on MCMC optimization, iii) UQ and UP for both model parameters and thermodynamic properties, and iv) statistical model comparison and selection among diverse solution models. This highly accessible software further enables users to tailor functionalities and models to specific requirements, thereby enriching the capabilities and contributions within the CALPHAD community.

\chapter{Thermodynamic modeling of the M-Pd-Zn system for intermetallic catalysts design} \label{chap:intermetallics}

\section{Introduction} \label{intermetallics:sec:intro}
Catalytic applications of intermetallic compounds (IMCs) are quickly expanding due to the precise design of binary and ternary metal active sites \cite{armbruster2014intermetallic, dasgupta2019intermetallics, furukawa2017intermetallic, armbruster2020intermetallic, yang2020intermetallic}. Atomic arrangement in IMCs provides an opportunity to isolate active metal atoms in a majority of an inactive host metal to tune the nuclearity of the active site, defined as the number of catalytically active metal atoms in a contiguous unit. Palladium (Pd) alloys are known for their excellent catalytic properties in a variety of reactions such as hydrogenation of alkynes (including acetylene) and butadiene \cite{teschner2006alkyne, zhou2016pdzn, sarkany1993hydrogenation}. By alloying with inert components such as zinc (Zn), catalytic ensembles on the surface of Pd can be controlled to enhance selectivity for semi/partial-hydrogenation reactions \cite{zhou2016pdzn, Dasgupta2022}. The Pd-Zn catalyst is also effective for other reactions, such as methanol steam reforming and ester hydrogenation \cite{conant2008stability, green1993ester}.

In the Pd-Zn system, site occupancy of Pd and Zn in the gamma-brass ($\gamma$-brass) phase offers distinct advantages for precisely controlling the composition of active sites \cite{dasgupta2019generalized}. In collaboration with experimental efforts in the Rioux group \cite{Dasgupta2022}, we have recently illustrated that subtle changes in composition within the Pd-Zn $\gamma$-brass phase can be used to control the surface active site between isolated Pd1 and Pd3 species isolated by surrounding Zn atoms. The $\gamma$-brass phase has the $\gamma$-Cu$_5$Zn$_8$ structure with space group $I\bar{4}3m$ and 52 atoms per crystallographic cell in four Wyckoff sites, i.e., the outer tetrahedral (OT) site 8c, the inner tetrahedral (IT) site 8c, the octahedral (OH) 12e, and the cuboctahedral (CO) site 24g \cite{strom1969x}. Adjusting compositions in the $\gamma$-brass phase leads to different surface chemistry for given Miller indices, as variation in elemental site occupancies exposes different Pd-Zn arrangements on the surface. Before surface site structure can be considered, a thermodynamic description of the bulk intermetallic phase is needed. A third element can be introduced in the Pd-Zn $\gamma$-brass IMCs, and depending on their site occupancy, further control of catalytic chemistry can be realized. For example, Pd-M-Pd ensembles (M = Cu, Ag, and Au) in the M-Pd-Zn $\gamma$-brass lead to intermediate acetylene semi-hydrogenation activity and selectivity between Pd1 (ie. Pd-Zn-Pd) and Pd3 (ie. Pd-Pd-Pd) active sites \cite{Dasgupta2022}. Thermodynamic description of the $\gamma$-brass phase in M-Pd-Zn is hence fundamental to evaluate phase stability, site occupancy, and in turn, surface constructions, which would be helpful to design active ensembles and improve selectivity for catalytic reactions. 

\section{Thermodynamic modeling of the Pd-Zn system with uncertainty quantification} \label{intermetallics:sec:PdZn}
Pd-Zn intermetallic catalysts show encouraging combinations of activity and selectivity on well-defined active site ensembles \cite{Dasgupta2022}. Thermodynamic description of the Pd–Zn system, delineating phase boundaries and enumerating site occupancies within intermediate alloy phases, is essential to determining the ensembles of Pd–Zn atoms as a function of composition and temperature. 

Combining the present extensive first-principles calculations and available experimental data, the Pd-Zn system was remodeled using the CALPHAD approach. High throughput modeling tools with uncertainty quantification, i.e., ESPEI and PyCalphad, were incorporated in the phase analysis. The site occupancies across the $\gamma$-brass phase composition region were given special attention. A four-sublattice model was used for the $\gamma$-brass phase owing to its four Wyckoff positions, i.e., the outer tetrahedral (OT) site 8c, the inner tetrahedral (IT) site 8c, the octahedral (OH) site 12e, and the cuboctahedral (CO) site 24g. The site fractions of Pd and Zn calculated from the present thermodynamic model show the occupancy preference of Pd in the OT and OH sublattices in agreement with experimental observations. The force constants obtained from DFT-based phonon calculations further support the tendency of Pd to occupy the OH sublattice compared with the IT and CO sublattice. The catalytic ensembles changing from Pd monomers (Pd1) to trimers (Pd3) on the $\gamma$-brass phase surface are attributed to the increasing Pd occupancy in the OH sublattice.

\subsection{Modeling details} \label{intermetallics:ssec:PdZnmodel}
%%% Phase information and Gibbs energy models
The Pd–Zn system contains 6 solution phases, i.e., Liquid, FCC, HCP, BCC$\_$B2($\beta$), FCC$\_$L1$_0$ ($\beta_1$), and Gamma ($\gamma$-brass), and 3 stoichiometric compounds, i.e., Pd$_2$Zn, PdZn$_2$, and Pd$_9$Zn$_{91}$ based on the works summarized by Vizdal et al. \cite{vizdal2006experimental}. Details of these phases can be seen in Table \ref{intermetallics:PdZn_phases}, including phase names, crystallographic information, and the sublattice models of phases used in the present work.

\begin{table}[H]
    \footnotesize
    \centering
    \caption{Crystallographic information for phases in the Pd–Zn system and their sublattice models used in the present CALPHAD modeling.}
    \begin{tabular}{>{\raggedright\arraybackslash}m{2.5cm}>{\raggedright\arraybackslash}m{2.5cm}>{\raggedright\arraybackslash}m{2cm}>{\raggedright\arraybackslash}m{2.5cm}>{\raggedright\arraybackslash}m{6cm}}
        \hline
         \textbf{Phase name} & \textbf{Strukturbericht} & \textbf{Space group} & \textbf{Pearson symbol} & \textbf{Model} \\
        \hline
         Liquid($L$) &  &  &  & (Pd,Zn)$_1$ \\
         FCC$\_$A1 & A1 & $Fm\Bar{3}m$ & cF4 & (Pd,Zn)$_1$ \\
         HCP$\_$Zn & A3 & $P6_3/mmc$ & hP2 & (Pd,Zn)$_1$ \\
         BCC$\_$A2 & A2 & $Im\Bar{3}m$ & cI2 & (Pd,Zn)$_1$ \\
         BCC$\_$B2 ($\beta$) & B2 & $Pm\Bar{3}m$ & cP2 & (Pd,Zn)$_{0.5}$(Pd,Zn)$_{0.5}$ \\
         FCC$\_$L1$_0$($\beta_1$) & L1$_0$ & $P4/mmm$ & tP2 & (Pd,Zn)$_1$(Pd,Zn)$_1$ \\
         $\gamma$-brass & D8$_2$ & $I4\Bar{3}m$ & cI52 & (Pd,Zn)$_2$(Pd,Zn)$_3$(Pd,Zn)$_2$(Pd,Zn)$_6$ \\
         Pd$_2$Zn &  & $Pnma$ & oP12(C23) & (Pd)$_2$(Zn)$_1$ \\
         PdZn$_2$ &  & $Cmm2$ & oS48 & (Pd)$_1$(Zn)$_2$ \\
         Pd$_9$Zn$_{91}$ &  &  &  & (Pd)$_{0.09}$(Zn)$_{0.91}$ \\
        \hline
    \end{tabular}
    \label{intermetallics:PdZn_phases}
\end{table}

Phase equilibrium properties of the Pd–Zn system published before 2006 were reviewed by Vizdal et al. \cite{vizdal2006experimental}. Massalski \cite{massalski1986binary} reported the solubility of Zn in FCC to be about mole fraction of Zn $x_{Zn}$ = $0.18 - 0.20$, and similar values were also observed by Hansen and Anderko \cite{hansen1958constitution}. The maximum solubility of Zn in FCC was around $x_{Zn}$ = $0.26$ at 1000 reported by Chiang et al. \cite{ChiangIpserChang1977} and Kou and Chang \cite{kou1975thermodynamics}. The solubility of Pd in HCP was reported to be lower than $x_{Zn}$ = 0.01 \cite{vizdal2006experimental, massalski1986binary, hansen1958constitution}. Vizdal et al. \cite{vizdal2006experimental} measured the temperatures of the invariant reactions in the Zn-rich region using differential thermal analysis (DTA). Thermochemical measurements for the Pd-Zn system are scarce. Kou and Chang \cite{kou1975thermodynamics} measured the activities of Zn ($\alpha_{Zn}$) in $\beta_1$, showing that increases from $-$9.146 to $-$1.464 with $x_{Zn}$ from 0.3762 to 0.5779. They also reported the enthalpies of formation of as $-66.5$ kJ/mol-atom at 1000. Chiang et al. \cite{ChiangIpserChang1977} measured the vapor pressures of Zn between 750 and 1300 K with $x_{Zn}$ = 0$-$0.83 using the isopiestic method. From these measurements, they determined the activities of Pd and Zn in FCC, $\beta$, and $\beta_1$, and partial molar Gibbs energy and enthalpy in $\beta_1$. According to Chiang et al. \cite{ChiangIpserChang1977}, at 1273 K, the activity values of $ln\alpha_{Zn}$ increase from −15.24 to −1.48 with increasing $x_{Zn}$ from 0.01 to 0.6; and the partial molar Gibbs energy and enthalpy reach the lowest values at $x_{Zn}$ = 0.5, which are $-$50.8 $\pm$ 2.0 kJ/mol-atom and $-$73.9 $\pm$ 10.0 kJ/mol-atom, respectively. Amore et al. \cite{amore2009thermochemistry} used calorimetry to obtain the enthalpy of formation between −33.7 and −35.1 kJ/mol-atom for the alloys with $x_{Zn}$ = 0.77$-$0.8. Site occupancy in $\gamma$ was reported by Edström et al. \cite{strom1969x}, Gourdon et al. \cite{gourdon2006intergrowth}, and Dasgupta et al. \cite{Dasgupta2022} using X-ray diffraction (XRD). The OT sites were fully occupied by Pd, the OH sites were occupied by Pd and Zn, and the IT and CO sites were occupied by Zn.

The Gibbs energy functions of pure Pd and Zn are taken from the Scientific Group Thermodata Europe (SGTE) pure element database \cite{dinsdale1991sgte}. The designation HCP$\_$Zn has been used for Zn to differentiate from the typical HCP$\_$A3 metals since the ratio of lattice parameters $c/a = 1.86$ for Zn is higher than the typical HCP metals with $c/a = 1.57 - 1.62$ \cite{schmid2012zinc}. Schmid-Fetzer \cite{schmid2012zinc} suggested eliminating HCP$\_$Zn from the thermodynamic database. However, Dinsdale \cite{dinsdale2021modelling} indicated that Schmid-Fetzer's argument is less accurate due to the extra stability when compositions close to pure Zn by first-principles calculations. In the present work, terminal Zn-rich solid solutions ($x_{Zn}$ near 1) designated as HCP$\_$Zn are hence selected as the standard reference state for Zn from the SGTE \cite{dinsdale1991sgte}.

Gibbs energies of the solution phases $\theta$ of $L$, FCC, BCC and HCP are formulated as:
\begin{equation} \label{intermetallics:solutionGeq}
    G_m^{\theta} = \sum_{i=Pd,Zn}x_i^oG_i^{\theta} + RT\sum_{i=Pd,Zn}x_i\ln x_i + ^{xs}G
\end{equation}
where $x_i$ is the mole fraction of component $i$, $G_i^{\theta}$ the Gibbs energy of component $i$, $R$ the gas constant, $T$ the temperature, and $^{xs}G$ the excess Gibbs energy. The first term represents the mechanical mixing of the endmembers, here the pure elements. The second term represents the ideal configurational entropy of mixing contribution to Gibbs energy. The third term represents the excess Gibbs energy, which is described by the Redlich-Kister polynomial \cite{redlich1948algebraic}:
\begin{equation} \label{intermetallics:solutionRK}
    ^{xs}G = x_{Pd}x_{Zn}\sum_{v=0}{^vL_{Pd,Zn}}(x_{Pd}-x_{Zn})^k
\end{equation}
where $^vL_{Pd,Zn}$ is the $v$th interaction term between Pd and Zn, which can be modeled using (\ref{method:eq:CEFLT}) as discussed in Section \ref{method:ssec:CEF}.

The BCC$\_$B2 $\beta$ phase (space group $Pm\Bar{3}m$ with two Wyckoff sites 1a and 1c) appears at high temperatures. To account for the order-disorder transition between BCC$\_$A2/B2, a partitioning model is adopted, which treats the ordered and disordered components separately but with the same Gibbs energy function \cite{ansara1988thermodynamic, ansara1997reply}. The general Gibbs energy for modeling this order-disorder transition is formulated as:
\begin{equation} \label{intermetallics:disorderG}
    G_m=G_m^{ord}(y_i^\prime,y_i^{\prime\prime})+G_m^{dis}(x_i)-G_m^{ord}(y_i^\prime=x_i, y_i^{\prime\prime}=x_i)
\end{equation}
where $x_i$  is the mole fraction of Pd or Zn, $y_i^{(s)}$ is the site fraction of component $i$ on sublattice $s$, $G_m^{dis}(x_i)$ is the Gibbs energy of BCC$\_$A2 disordered phase as described by (\ref{intermetallics:solutionGeq}). $G_m^{dis}(x_i)-G_m^{ord}(y_i^{\prime}=x_i, y_i^{\prime\prime}=x_i)$ is ordered contribution to Gibbs energy. A two-sublattice model (Pd,Zn)$_{0.5}$(Pd,Zn)$_{0.5}$ is applied for the ordered BCC$\_$B2 phase. 

FCC$\_$L1$_0$ $\beta_1$ phase is stable at low temperatures (space group $P4/mmm$ with two Wyckoff sites 1a and 1d). A two-sublattice model (Pd,Zn)$_1$(Pd,Zn)$_1$ is applied for this phase with the Gibbs energy formula formulated as:
\begin{equation} \label{intermetallics:L10G}
    \begin{aligned}
    G_m & =\sum_{i=Pd,Zn}{\sum_{j=Pd,Zn}{y_i^\prime y_j^{\prime\prime}}{^o}G_{i:j}}+RT(\sum_{i=Pd,Zn}{y_i^\prime\ln{\left(y_i^\prime\right)}}+\sum_{j=Pd,Zn}{y_j^{\prime\prime}\ln{\left(y_j^{\prime\prime}\right)}})\\&+y_{Pd}^{\prime}y_{Zn}^{\prime}\left(y_{Pd}^{\prime\prime}L_{Pd,Zn:Pd}+y_{Zn}^{\prime\prime}L_{Pd,Zn:Zn}\right)+y_{Pd}^{\prime\prime}y_{Zn}^{\prime\prime}\left(y_{Pd}^\prime L_{Pd:Pd,Zn}+y_{Zn}^\prime L_{Zn:Pd,Zn}\right)
    \end{aligned}
\end{equation}
where $y_i^{(s)}$ is the site fraction of component $i$ on sublattice $s$, ${^o}G_{i:j}$ are the Gibbs energies of the endmembers $(i:j)$, and $L$ are the interaction parameters, which can be expanded using the Redlich-Kister polynomials \cite{redlich1948algebraic} in the same way as in (\ref{intermetallics:solutionRK}). According to the Pd-Zn phase diagram \cite{vizdal2006experimental}, the FCC$\_$A1 and the FCC$\_$L1$_0$ phase are separated by the phases of Pd$_2$Zn and BCC$\_$B2; FCC$\_$A1 and FCC$\_$L1$_0$ are treated as two phases in the present work for the sake of simplicity.

The $\gamma$-brass phase has four Wyckoff sites (space group $I\bar{4}3m$). A four-sublattice model \\ (Pd,Zn)$_2$(Pd,Zn)$_3$(Pd,Zn)$_2$(Pd,Zn)$_6$ is adopted according to its Wyckoff sites (IT, 8c), (OH, 12e), (OT, 8c), and (CO, 24g), respectively. The Gibbs energy of $\gamma$-brass phase is written as: 
\begin{equation} \label{intermetallics:gammaG}
    \begin{aligned}
    G_m&=\sum_{i=Pd,Zn}\sum_{j=Pd,Zn}\sum_{k=Pd,Zn}\sum_{l=Pd,Zn}{y_i^\prime y_j^{\prime\prime}y_k^{\prime\prime\prime}y_l^{\prime\prime\prime\prime}{{^o}G}_{i:j:k:l}}\\&+RT\left(2\sum_{i=Pd,Zn}{y_i^{\prime}\ln{\left(y_i^{\prime}\right)}}+3\sum_{j=Pd,Zn}{y_j^{\prime\prime}\ln{\left(y_j^{\prime\prime}\right)}}\right)\\&+RT\left(2\sum_{k=Pd,Zn}{y_k^{\prime\prime\prime}\ln{\left(y_k^{\prime\prime\prime}\right)}}+6\sum_{l=Pd,Zn}{y_l^{\prime\prime\prime\prime}\ln{\left(y_l^{\prime\prime\prime\prime}\right)}}\right)+{^{xs}}G_m
    \end{aligned}
\end{equation}
where $y_i^{(s)}$ is the site fraction of component $i$ on sublattice $s$, $s=^\prime, \prime\prime, \prime\prime\prime$, and $\prime\prime\prime\prime$ representing the IT, OH, OT, and CO sublattice, respectively, and ${^{xs}}G_m$ is the excess Gibbs energy. In the present work, we considered only the interaction parameters on the second sublattice (the OH site) due to the lower values of formation enthalpy for the endmembers (Zn)$_2$(Pd)$_3$(Pd)$_2$(Pd)$_6$ and (Zn)$_2$(Zn)$_3$(Pd)$_2$(Zn)$_6$ from DFT-based calculations along with experimental observations \cite{strom1969x, gourdon2006intergrowth} showing the OH site occupied by both Pd and Zn. Hence, the excess Gibbs energy is as follows:
\begin{equation} \label{intermetallics:gammaGex}
    {^{xs}}G_m=y_{Pd}^{\prime\prime}y_{Zn}^{\prime\prime}y_{Zn}^\prime y_{Pd}^{\prime\prime\prime}{y_{Zn}^{\prime\prime\prime\prime}L}_{Zn:Pd,Zn:Pd:Zn}
\end{equation}

In the present work, Pd$_2$Zn, PdZn$_2$, and Pd$_9$Zn$_{91}$ are treated as stoichiometric compounds (phases) with their Gibbs energies expressed as:
\begin{equation} \label{intermetallics:stoiG}
    G^{{Pd}_e{Zn}_f}=e\:{^o}G_{Pd}^{fcc}+f\:{^o}G_{Zn}^{hcp}+c_1+c_2T
\end{equation}
where ${^o}G_{Pd}^{fcc}$ and ${^o}G_{Zn}^{hcp}$ are Gibbs energies of FCC-Pd and HCP-Zn, respectively, and $c_1$ and $c_2$ are parameters to be evaluated.

%%% CALPHAD modeling details
Thermodynamic modeling of the Pd-Zn system was carried out using the open-source software ESPEI \cite{bocklund2019espei} and PyCalphad \cite{otis2017pycalphad} as introduced in Section \ref{method:ssec:tools}.  In the present work, the Gibbs energy functions of stoichiometric compounds and endmembers in the $\beta$, $\beta_1$, and $\gamma$-brass phases were evaluated from DFT-based first-principles calculations with the computational details in the following paragraph, and results discussed in Section \ref{intermetallics:ssec:PdZndft}. In the present modeling of the Pd-Zn system, the input data are primarily the experimental phase equilibrium information, including phase boundary data \cite{vizdal2006experimental, massalski1986binary, hansen1958constitution} and thermochemical data \cite{amore2009thermochemistry, ChiangIpserChang1977} were used to refine model parameters. Each model parameter employed two chains with a standard derivation of 0.1 when initializing in a Gaussian distribution. The chain values can be tracked during the modeling process, and the MCMC steps were performed until the model parameters converged. The uncertainty quantification of model parameters and calculated thermodynamic properties and phase stability from the models is performed using PDUQ \cite{paulson2019quantified} (see Section \ref{method:ssec:tools}). In the present work, the values from the last MCMC step were used to estimate the uncertainty, 95\% uncertainty interval (or Bayesian credible intervals containing 95\% of the invariant samples) was applied to quantify the uncertainty. 

%%% First-principles details
DFT-based first-principles and phonon calculations were performed to obtain Helmholtz energies of intermetallic compounds and endmembers at finite temperatures, which are equal to Gibbs energies under ambient pressure. See details of this QHA method in Section \ref{method:sec:firstprinciples}. The Vienna Ab initio Simulation Package (VASP) \cite{kresse1996efficient} was employed for all DFT-based calculations. The projector augmented-wave method (PAW) was used to account for electron-ion interactions to increase computational efficiency compared with the full potential methods \cite{blochl1994projector, kresse1999ultrasoft}. Electron exchange and correlation effects were described using the generalized gradient approximation (GGA) as implemented by Perdew, Burke, and Ernzerhof (PBE) \cite{perdew1996generalized}. The GGA includes the electronic density and its gradient as exchange-correlation functionals. Furthermore, the hybrid exchange-correlation functional HSE06 \cite{heyd2003hybrid} was applied to calculate the enthalpy of formation with higher accuracy. The plane-wave basis cutoff energy was 277 eV for relaxations and 520 eV for the final static calculations. The convergence criterion of the electronic self-consistency was set as 10-6 eV/atom for relaxations and static calculations. 

\begin{table}[H]
    \normalsize
    \centering
    \caption{Details of DFT-based first-principles and phonon calculations for each compound or element, including space group, total atom(s) in the cell for the calculations, k-points meshes for structure relaxations and the final static calculations (indicated by DFT), supercell sizes for phonon calculations, and k-points meshes for phonon calculations. $^a$ Three  Pd$_9$Zn$_{43}$ configurations are used for the analysis of site occupancy as discussed in Section \ref{intermetallics:ssec:PdZnsite}.}
    \begin{tabular}{>{\raggedright\arraybackslash}m{2.5cm}>{\raggedright\arraybackslash}m{2cm}>{\raggedright\arraybackslash}m{2.5cm}>{\raggedright\arraybackslash}m{2.5cm}>{\raggedright\arraybackslash}m{2.8cm}>{\raggedright\arraybackslash}m{2.5cm}}
        \hline
         \textbf{Compounds} & \textbf{Space group} & \textbf{Atom(s) in the cell} & \textbf{k points for DFT} &  \textbf{Supercell for phonon} & \textbf{k points for phonon}\\
        \hline
        Pd	& $Fm\bar{3}m$	& 1	& $22\times22\times22$ &	$3\times3\times3$ &	$5\times5\times5$ \\
        Zn	& $P63/mmc$	& 2	& $24\times24\times24$ &	$\left[\begin{matrix}-1&2&1\\-3&-2&-1\\1&-2&1\\\end{matrix}\right]$	& $4\times4\times4$ \\
        Pd$_2$Zn	& $Pnma$ & 12	& $12\times12\times12$ &	$\left[\begin{matrix}-1&0&-1\\-1&0&1\\0&2&0\\\end{matrix}\right]$	& $4\times4\times4$ \\
        PdZn &	$P4/mmm$ &	2 &	$19\times19\times14$ & $3\times3\times3$ &	$5\times5\times5$ \\
        PdZn$_2$ &	$Cmm2$ &	48 &	$7\times7\times4$ &	$1\times1\times1$ &	$4\times4\times4$ \\
        Pd$_8$Zn$_{44}$ &	$I4\bar{3}m$ &	52 &	$4\times4\times4$	& $1\times1\times1$	& $4\times4\times4$ \\
        Endmembers of $\gamma$-brass phase &	N/A &	26	& $8\times8\times8$	& N/A &	N/A \\
        Pd$_9$Zn$_{43}\ ^a$	& $I4\bar{3}m$ &	52 &	$3\times3\times3$ &	$1\times1\times1$ &	$2\times2\times2$ \\
        \hline
    \end{tabular}
    \label{intermetallics:PdZn_DFT_details}
\end{table}

Table \ref{intermetallics:PdZn_DFT_details} provides parameters for first-principles and phonon calculations, including reciprocal k-points meshes and supercell sizes for compounds Pd$_2$Zn, PdZn, PdZn$_2$, 16 endmembers of $\gamma$-brass phase (26 atoms for each endmember in the primitive cell of $\gamma$-brass phase), and Pd$_8$Zn$_{44}$ (the key endmember of $\gamma$-brass phase in its crystallographic cell). The phonon calculations were performed using the supercell method. The phonon DOS and force constants were analyzed using the YPHON code \cite{wang2014yphon}. Note that Table \ref{intermetallics:PdZn_DFT_details} includes the supercell sizes and k-points meshes for phonon calculations, while the plane-wave cutoff energy of 277 eV was used for phonon calculations. 

\subsection{Properties of Pd–Zn compounds by first-principles calculations} \label{intermetallics:ssec:PdZndft}
Table \ref{intermetallics:PdZn_DFT_lattice} shows the predicted lattice parameters of Pd$_2$Zn, PdZn$_2$, and $\gamma$-brass phase in the present work with experimental data in the literature \cite{strom1969x, stadelmaier1961ternare, neumann1978structural, gourdon2006zn1}. The lattice parameter $c$ of Pd$_2$Zn predicted from the present first-principles calculations is 7.83 \r{A}, slightly higher than the experimental 7.65 \r{A} \cite{stadelmaier1961ternare}. The lattice parameters of PdZn$_2$ and $\gamma$-brass are in good agreement with experimental results with the mean absolute error value around 0.027 \r{A}. 

\begin{table}[H]
    \normalsize
    \centering
    \caption{Predicted lattice parameters of FCC-Pd, HCP-Zn, Pd$_2$Zn, PdZn$_2$, and $\gamma$-brass by first-principles calculations from the relaxed structures at 0 K, together with available experimental (Expt.) data for comparison.}
    \begin{tabular}{>{\raggedright\arraybackslash}m{2.5cm}>{\raggedright\arraybackslash}m{2.5cm}>{\raggedright\arraybackslash}m{2.5cm}>{\raggedright\arraybackslash}m{2.5cm}>{\raggedright\arraybackslash}m{2.5cm}}
    \hline
      \textbf{Phases} &  \textbf{$a$ (\r{A})} & \textbf{$b$ (\r{A})} & \textbf{$c$ (\r{A})} & \textbf{Source} \\
    \hline
    Pd	& 3.9309& &	& This work\\
        & 3.8902& & & Expt.\cite{arblaster1997crystallographic} \\
    Zn & 2.6426	& &	5.0268 & This work\\
       & 2.6594 & & 4.9328 & Expt.\cite{jette1935precision}\\
    Pd$_2$Zn & 5.3975 & 4.1917 & 7.8343 & This work\\
	    & 5.3500 & 4.1400 & 7.6500 & Expt.\cite{stadelmaier1961ternare}\\
    PdZn$_2$ & 5.3975 & 4.1917 & 7.8343 & This work\\
             & 5.3500 & 4.1400 & 7.6500 & Expt.\cite{stadelmaier1961ternare}\\
    $\gamma$-brass phase & 9.1024 & &	& This work \\
                   & 9.1022 & & & Expt.\cite{strom1969x} \\
                   & 9.0906 & & & Expt.\cite{gourdon2006zn1} \\
    \hline
    \end{tabular}
    \label{intermetallics:PdZn_DFT_lattice}
\end{table}

Table \ref{intermetallics:PdZn_DFT_EOS} shows the equilibrium volume V$_0$, bulk modulus B, and the derivative of bulk modulus B$^\prime$ obtained from the EOS E-V fitting at 0 K in comparison with previous DFT calculations and experimental data \cite{shang2016comprehensive}. Figure \ref{intermetallics:fig:PdZnDOS} compares the phonon DOS of FCC-Pd, HCP-Zn, and stoichiometric compounds Pd$_2$Zn and PdZn$_2$. In the low-frequency region (e.g., < 3 THz), HCP-Zn has the highest phonon DOS, followed by PdZn$_2$, Pd$_2$Zn, and FCC-Pd. The higher DOS in the low-frequency region results in a lower average phonon frequency \cite{shang2007phase}. This can be confirmed by the lowest bulk modulus B of HCP-Zn compared with PdZn$_2$, Pd$_2$Zn, and FCC-Pd. The bulk modulus B of HCP-Zn fitted from the present work is 57.5 GPa, which is lower than PdZn$_2$ 118.1 GPa, Pd$_2$Zn 146.0 GPa, and FCC-Pd 167.9 GPa.

\begin{table}[H]
    \normalsize
    \centering
    \caption{Equilibrium volume V$_0$, bulk modulus B, and the derivative of bulk modulus B$^\prime$, based on the present EOS fittings at 0 K in comparison with the previous DFT studies.}
    \begin{tabular}{>{\raggedright\arraybackslash}m{2.5cm}>{\raggedright\arraybackslash}m{2.5cm}>{\raggedright\arraybackslash}m{2.5cm}>{\raggedright\arraybackslash}m{2.5cm}>{\raggedright\arraybackslash}m{2.5cm}}
    \hline
     \textbf{Phases} &  \textbf{V$_0$ (\r{A}$^3$/atom)} & \textbf{B (GPa)} & \textbf{B$^\prime$} & \textbf{Source} \\
    \hline
    Pd & 15.300 & 167.9 & 5.51 & This work\\
       & 15.340 & 163.3 & 5.50 & DFT \cite{shang2016comprehensive}\\
       & 14.716 & 195.5 &   & Expt.\cite{shang2016comprehensive}\\
    Zn & 15.336	& 57.5 & 5.20 & This work\\
       & 15.491 & 58.6 & 5.01 & DFT \cite{shang2016comprehensive}\\
       & 15.185 & 73.2 &   & Expt.\cite{shang2016comprehensive}\\
    Pd$_2$Zn & 14.824 & 146.0 & 5.38 & This work\\
    PdZn$_2$ & 14.576 & 118.1 & 5.33 & This work\\
    \hline
    \end{tabular}
    \label{intermetallics:PdZn_DFT_EOS}
\end{table}

\begin{figure}[H]
    \centering
    \normalsize
    \includegraphics[width=0.5\textwidth]{intermetallics/Intermetallics-PdZnDOS.jpg}
    \caption{Predicted phonon DOS of Pd, Zn, Pd$_2$Zn and PdZn$_2$ from the DFT-based phonon calculations.}
    \label{intermetallics:fig:PdZnDOS}
\end{figure}

Figure \ref{intermetallics:fig:PdZnQHAPd} and Figure \ref{intermetallics:fig:PdZnQHAZn} show the comparison of the entropy and enthalpy of FCC-Pd and HCP-Zn from the phonon calculations to the SGTE pure element database \cite{dinsdale1991sgte}. Both show excellent agreement. For results of Pd obtained from phonon calculations and SGTE, the difference of enthalpy is less than 6.5\% and that of entropy is less than 5\%. The results of Zn show the difference of enthalpy less than 2.1\% and that of entropy less than 3.5\%.

\begin{figure}[H]
    \centering
    \normalsize
    \includegraphics[width=0.4\textwidth]{intermetallics/Intermetallics-PdZnQHAPd.jpg}
    \caption{Comparison of the (a) entropy S and (b) enthalpy H $-$ H$_{300}$ of Pd from the DFT-based phonon calculations to the SGTE data \cite{dinsdale1991sgte}.}
    \label{intermetallics:fig:PdZnQHAPd}
\end{figure}

\begin{figure}[H]
    \centering
    \normalsize
    \includegraphics[width=0.4\textwidth]{intermetallics/Intermetallics-PdZnQHAZn.jpg}
    \caption{Comparison of the (a) entropy S and (b) enthalpy H $-$ H$_{300}$ of Zn from the DFT-based phonon calculations to the SGTE data \cite{dinsdale1991sgte}.}
    \label{intermetallics:fig:PdZnQHAZn}
\end{figure}

Table \ref{intermetallics:tab:PdZn_DFT_Hf} shows the enthalpy of formation $\Delta_f$H at 0 K predicted from the present DFT-based calculations using the exchange-correlation functionals of GGA and HSE06, along with experimental data \cite{ChiangIpserChang1977, kou1975thermodynamics}. For the $\gamma$-brass phase, the configuration of Pd$_{10}$Zn$_{42}$ ($x_{Zn}$ = 0.81) was used for DFT-based calculations and compared with experimental data at $x_{Zn}$ = 0.80. The $\Delta_f$H value predicted using HSE06 is $-33.7$ kJ/mol-atom, agreeing reasonably well with $-35.1$ kJ/mol-atom from experiments using calorimetry \cite{amore2009thermochemistry}. For the $\beta_1$ phase, the configuration of PdZn ($x_{Zn}$ = 0.50) was used. The $\Delta_f$H value of PdZn predicted using HSE06 is $-69.2$ kJ/mol-atom, which is in good agreement with measured $-73.9\pm10$ kJ/mol-atom reported by Chiang et al. \cite{ChiangIpserChang1977} and $-66.6$ kJ/mol-atom reported by Kou and Chang \cite{kou1975thermodynamics}, but is lower than the value predicted by GGA ($-53.7$ kJ/mol-atom). For both $\gamma$-brass and $\beta_1$ phases, the $\Delta_f$H values predicted by HSE06 are more accurate than those predicted by PBE-GGA. Considering the high computational cost, HSE06 was only applied for key endmembers close or on the convex hull in the present work such as (Zn)$_2$(Pd)$_3$(Pd)$_2$(Pd)$_6$ and (Zn)$_2$(Zn)$_3$(Pd)$_2$(Zn)$_6$. 

\begin{table}[H]
    \normalsize
    \centering
    \caption{Predicted enthalpy of formation at 0 K, $\Delta_f$H (kJ/mol-atom), of $\gamma$-brass and $\beta_1$ using DFT-based calculations with GGA and HSE06 as exchange-correlation functionals, respectively, in comparison with available experimental data.}
    \begin{tabular}{>{\raggedright\arraybackslash}m{2.5cm}>{\raggedright\arraybackslash}m{3cm}>{\raggedright\arraybackslash}m{2.5cm}>{\raggedright\arraybackslash}m{3cm}>{\raggedright\arraybackslash}m{3.5cm}}
    \hline
     \textbf{Phases} &  \textbf{Configurations} & \textbf{$x_{Zn}$} & \textbf{$\Delta_f$H} & \textbf{Source} \\
    \hline
    $\gamma$-brass phase & Pd$_{10}$Zn$_{42}$ & 0.81 & $-27.9$ & DFT/GGA\\
        & Pd$_{10}$Zn$_{42}$ & 0.81 & $-33.7$ & DFT/HSE06\\
       & N/A & 0.80 & $-35.1$ & Expt. at 300 K \cite{amore2009thermochemistry}\\
    $\beta_1$ phase & PdZn & 0.5 & $-53.7$ & DFT/GGA\\
       & PdZn & 0.5 & $-69.2$ & DFT/HSE06\\
       & PdZn & 0.5 & $-73.9\pm10$ & Expt. at 1273 K \cite{ChiangIpserChang1977}\\
       & PdZn & 0.5 & $-66.6$ & Expt. at 1273 K \cite{kou1975thermodynamics}\\
    \hline
    \end{tabular}
    \label{intermetallics:tab:PdZn_DFT_Hf}
\end{table}

\subsection{Thermodynamic modeling and phase equilibria} \label{intermetallics:ssec:PdZneq}
Figure \ref{Intermetallics:fig:PdZnPhaseDiagram} shows the calculated phase diagram in comparison with experimental data \cite{vizdal2006experimental, nowotny1951beitrag, alasafi1978mischung}, showing a good agreement, particularly the phase boundaries of the $\gamma$-brass phase. The max difference between calculated and experimental Zn compositions of the $\gamma$-brass phase  $x_{Zn}^{Cal}-x_{Zn}^{Expt.}$ is around 0.016 at 892 K. The solubility range of the $\gamma$-brass phase is between $x_{Zn} = 0.775 - 0.846$ from 300 K to 1000 K. The congruent melting temperature of the $\gamma$-brass phase is 1150 K in good agreement with 1153 K suggested by Massalski \cite{massalski1986binary}.\\

\begin{figure}[H]
    \centering
    \includegraphics[width=0.6\linewidth]{intermetallics/Intermetallics-PdZnPhaseDiagram.jpg}
    \caption{Calculated phase diagram from the present CALPHAD modeling in comparison with experimental data from Nowotny et al. \cite{nowotny1951beitrag} (diamonds), Alasafi et al. \cite{alasafi1978mischung} (hexagons), and experimental data summarized by Vizdal et al. \cite{vizdal2006experimental} (triangles). Hollow diamonds and hexagons represent single phase region and shadowed diamonds and hexagons represent two phase regions reported by Nowotny et al. \cite{nowotny1951beitrag} and Alasafi et al. \cite{alasafi1978mischung}.}
    \label{Intermetallics:fig:PdZnPhaseDiagram}
\end{figure} 

\begin{table}[H]
    \centering
    \caption{Temperatures and compositions of invariant reactions in the Pd-Zn system calculated from the present CALPHAD modeling with available experimental data included.}
    \begin{tabular}{>{\raggedright\arraybackslash}m{5cm}>{\raggedright\arraybackslash}m{1cm}>{\raggedright\arraybackslash}m{1cm}>{\raggedright\arraybackslash}m{1cm}>{\raggedright\arraybackslash}m{3cm}>{\raggedright\arraybackslash}m{2.5cm}}
        \hline
         \textbf{Reaction}& \textbf{$x_{Zn}$} & (\%) &  & \textbf{Temperature (K)} & \textbf{Ref.}\\
        \hline
        Liquid $\rightarrow \beta + \gamma$&74.9&63.7&78.1&1122.1&This work\\
         &75&65&77&1118&Expt.\cite{massalski1986binary}\\
         $\beta \rightarrow \beta_1 + \gamma$&61.5&57.3&77.5&834&This work\\
	&57&55&76&838&Expt.\cite{massalski1986binary}\\
        $\beta_1 + \gamma \rightarrow$ PdZn$_2$&57.2&77.5&66.7&799.6&This work\\
	&56&76&66.7&$803\pm10$&Expt.\cite{massalski1986binary}\\
        Liquid $+ \gamma \rightarrow$ Pd$_9$Zn$_{91}$&97.7&84.6&91.0&707.2&This work\\
	&98&85&92&703&Expt.\cite{hansen1958constitution}\\
	&&&&707&Expt.\cite{vizdal2006experimental}\\
        Liquid $\rightarrow$ Pd$_9$Zn$_{91}$ $+$ HCP&98.2&91.0&99.0&681.1&This work\\
	&&&&690&Expt.\cite{vizdal2006experimental}\\
         \hline
    \end{tabular}
    \label{intermetallics:tab:inv}
\end{table}

Table \ref{intermetallics:tab:inv} lists the temperatures and compositions of invariant reactions. The experimentally reported invariant reactions, i.e., Liquid $\rightarrow$ Pd$_9$Zn$_{91}$ $+$ HCP by Vizdal et al. \cite{vizdal2006experimental} and other four invariant reactions by Massalski \cite{massalski1986binary}, are well reproduced. The largest discrepancy is seen for the eutectic reaction, Liquid $\rightarrow$ Pd$_9$Zn$_{91}$ + HCP, i.e., 681 K calculated from the present work versus 690 K in the literature \cite{vizdal2006experimental}.

\begin{figure}[H]
    \centering
    \includegraphics[width=0.45\linewidth]{intermetallics/Intermetallics-PdZnQHAPd2Zn.jpg}
    \caption{Predicted (a) heat capacity C$_p$, (b) entropy S, and (c) enthalpy H $-$ H$_{300}$ of Pd$_2$Zn using the present results form DFT-based phonon calculations, compared with those from the present CALPHAD modeling.}
    \label{intermetallics:fig:PdZnQHAPd2Zn}
\end{figure}

Figure \ref{intermetallics:fig:PdZnQHAPd2Zn} and Figure \ref{intermetallics:fig:PdZnQHAPdZn2} show the heat capacity, entropy, and enthalpy of stoichiometric compounds Pd$_2$Zn and PdZn$_2$ from the present model in comparison with results from the phonon calculations. Good agreement is found, especially for enthalpy with a difference less than 4.7\% for Pd$_2$Zn and 2.8\% for PdZn$_2$. The entropy of PdZn$_2$ from phonon calculations compared with the present model shows the largest discrepancy with a difference of around 6.5 J/mol-atom-K. This is because the model parameters of stoichiometric compounds, which are obtained from first-principles calculated enthalpy and entropy, were adjusted with the experimental data of the peritectoid temperature.

\begin{figure}[H]
    \centering
    \includegraphics[width=0.45\linewidth]{intermetallics/Intermetallics-PdZnQHAPdZn2.jpg}
    \caption{Predicted (a) heat capacity C$_p$, (b) entropy S, and (c) enthalpy H $-$ H$_{300}$ of PdZn$_2$ using the present results form DFT-based phonon calculations, compared with those from the present CALPHAD modeling.}
    \label{intermetallics:fig:PdZnQHAPdZn2}
\end{figure}

The parameters of the FCC\_A1 phase in the database were optimized based on phase boundary data and activity data of the FCC\_A1 phase. Considering fitting different types of data, a balanced modeling result between phase boundary data and activity data is chosen. The experimental data for liquidus and solidus on the Pd-rich side are scarce. In the present work, data for liquidus and solidus in the Pd-rich side summarized by Vizdal et al. \cite{vizdal2006experimental} are used for modeling, thus the shape of liquidus and solidus in the Pd-rich side followed the trend of experimental data. Figure \ref{intermetallics:fig:PdZnACR} shows the activity values of Zn at 1273 K calculated from the present model in comparison with experimental data by Chiang et al. \cite{ChiangIpserChang1977}. The activity values of Zn in $\beta_1$ from the present model agree well with the experiments \cite{ChiangIpserChang1977} with the mean absolute error of $\ln{a_{Zn}}$ being 0.4. A higher discrepancy occurs in the composition range $x_{Zn}$ = 0.51-0.6. For example, at $x_{Zn}$ = 0.55, $\ln{a_{Zn}}$ calculated from the present model is $-3.52$ compared with -1.56 measured by Chiang et al. \cite{ChiangIpserChang1977}. They reported a single $\beta_1$ phase for $x_{Zn}$ = 0.52 and a single $\beta$ phase for $x_{Zn}$ = 0.6. However, the present work predicts that $\beta_1$ is in equilibrium with $\beta$ at $x_{Zn}$ = 0.52, and $\beta$ is in equilibrium with Liquid at $x_{Zn}$=0.6 (see Figure \ref{Intermetallics:fig:PdZnPhaseDiagram}). Figure \ref{intermetallics:fig:PdZnACRUQ} shows the activity values of Zn in FCC, $\beta$, and $\beta_1$ phases, respectively, calculated from the present model with the shaded regions for the uncertainty of each phase. For FCC, larger uncertainty occurs when $x_{Zn}$ < 0.2, where the largest uncertainty is around $\pm15$\% from the mean value at $x_{Zn}$ = 0.02, and experimental data are located within this uncertainty region. For $\beta_1$, the experimental data are in the uncertainty region when $x_{Zn}$ < 0.5, and the largest error is around $x_{Zn}$ = 0.52 with $\ln{a_{Zn}}$ = $-1.66$ from experiments \cite{ChiangIpserChang1977} compared with $-3.35$ calculated from the present model. For $\beta$, $\ln{a_{Zn}} = -10.33$ from experiments \cite{ChiangIpserChang1977} at $x_{Zn}$ = 0.3 are closer to the lower limit of the uncertainty region, where $\ln{a_{Zn}}$ at the lower uncertainty is $-10.81$. The shaded ranges in Figure \ref{intermetallics:fig:PdZnACRUQ} decrease with increasing $x_{Zn}$, indicating a larger uncertainty of activity occurs in the Pd-rich region. 

\begin{figure}[H]
    \centering
    \includegraphics[width=0.5\linewidth]{intermetallics/Intermetallics-PdZnACR.jpg}
    \caption{Calculated activity of Zn at 1273 K with experimental data measured by Chiang et al. \cite{ChiangIpserChang1977} superimposed.}
    \label{intermetallics:fig:PdZnACR}
\end{figure}

\begin{figure}[H]
    \centering
    \includegraphics[width=0.5\linewidth]{intermetallics/Intermetallics-PdZnACRUQ.jpg}
    \caption{Uncertainty quantification of the activity of (a) FCC, (c) $\beta$, and (d) $\beta_1$, marked in the shaded regions with the corresponding color of each phase and compared with experimental data by Chiang et al. \cite{ChiangIpserChang1977}.}
    \label{intermetallics:fig:PdZnACRUQ}
\end{figure}

Figure \ref{intermetallics:fig:PdZnHMR} plots the enthalpy of formation $\Delta_f$H of the Pd-Zn phases at 1273 K and 300 K from the present model and available experimental data \cite{amore2009thermochemistry, ChiangIpserChang1977, kou1975thermodynamics}, along with the calculated results from the previous CALPHAD modeling \cite{vizdal2006experimental} at 300 K and the present first-principles results of the $\gamma$-brass phase at 0 K and high temperatures. $\Delta_f$H at 1273 K agrees reasonably well with experimental data \cite{ChiangIpserChang1977, kou1975thermodynamics}. $\Delta_f$H of $\beta_1$ at $x_{Zn}$ = 0.5 and 1273 K is $-70.1$ kJ/mol-atom from the present work, compared with -$73.9\pm10$ kJ/mol-atom measured by Chiang et al. \cite{ChiangIpserChang1977} and $-66.6$ kJ/mol-atom measured by Kou et al. \cite{kou1975thermodynamics}. Figure \ref{intermetallics:fig:PdZnHMR} shows that $\Delta_f$H of $\gamma$-brass from the present model has better agreement with experiments than the previous model \cite{vizdal2006experimental}. At $x_{Zn}$ = 0.8, $\Delta_f$H value of $\gamma$-brass from the present model is $-40.6$ kJ/mol-atom at 300 K and from the previous model \cite{vizdal2006experimental} is $-48.5$ kJ/mol-atom, compared with $-35.1$ kJ/mol-atom measured by Amore et al. \cite{amore2009thermochemistry}.

\begin{figure}[H]
    \centering
    \includegraphics[width=0.5\linewidth]{intermetallics/Intermetallics-PdZnHMR.jpg}
    \caption{Calculated enthalpy of formation $\Delta_f$H at (a) 1273 K and (b) 300 K, along with DFT-based calculations and available experimental data by Chiang et al. \cite{ChiangIpserChang1977}, Kou and Chang \cite{kou1975thermodynamics}, and Amore et al. \cite{amore2009thermochemistry}. DFT using HSE is calculated at 0 K. DFT using GGA is calculated at 0 K and high temperatures 300 K and 1270 K respectively, shown as purple bars in the figure.}
    \label{intermetallics:fig:PdZnHMR}
\end{figure}

\subsection{Site occupancy in the $\gamma$-brass phase and surface construction} \label{intermetallics:ssec:PdZnsite}
Figure \ref{intermetallics:fig:PdZnSOC} shows the calculated site fractions in $\gamma$-brass at 773 K and 1023 K from the present model in comparison with XRD results by Edström et al. \cite{strom1969x}, Gourdon et al. \cite{gourdon2006intergrowth} and Dasgupta et al. \cite{Dasgupta2022}. Temperature has little influence on the site fraction of the $\gamma$-brass phase. With increasing Pd content, Pd is predicted first to occupy the OT sublattice, and then the OH sublattice after the OT sublattice is fully occupied. The site fractions of Pd in the OH sublattice are in good agreement with experimental data. For example, at $x_{Pd}$ = 0.173, the calculated site fraction of Pd in the OH sublattice $y_{Pd}^{\rm OH}$ is 0.08, slightly higher than 0.07 measured by Dasgupta et al.\cite{Dasgupta2022} At $x_{Pd}$ = 0.181, 0.192, and 0.23, the calculated $y_{Pd}^{\rm OH}$ values are 0.118, 0.165, and 0.333, respectively, which agree well with experimental data with a mean absolute error (MAE) of 0.002. Vizdal et al.'s \cite{vizdal2006experimental} model could not predict site fractions in $\gamma$-brass due to the 2-sublattice model used.

\begin{figure}[H]
    \centering
    \includegraphics[width=0.5\linewidth]{intermetallics/Intermetallics-PdZnSOC.jpg}
    \caption{Calculated site fractions in $\gamma$-brass at 773K (solid lines) and 1023 K (dash lines) with the experimental data by Edström et al. \cite{strom1969x} at 923 K, Gourdon et al. \cite{gourdon2006intergrowth} at 1023 K, and Dasgupta et al. \cite{Dasgupta2022} at 773 K superimposed.}
    \label{intermetallics:fig:PdZnSOC}
\end{figure}

Force constants can be used to quantitatively understand interactions between atomic pairs \cite{shang2009first}. A large and positive force constant suggests a strong bonding interaction of an atomic pair, whereas a negative force constant indicates the tendency to separate \cite{yu2019synthesis}. To understand the site occupancy of Pd in $\gamma$-brass, we examined energies and force constants in three Pd$_9$Zn$_{43}$ configurations to analyze the occupancy of an additional Pd atom compared to the full Pd OT occupation in Pd$_8$Zn$_{44}$. Three configurations are $({{\rm Pd}_8)}^{\rm OH}{{(\rm Pd}_1{\rm Zn}_{11})}^{\rm OH}({{\rm Zn}_8)}^{\rm IT}({{\rm Zn}_{24})}^{\rm CO}$, \\$({{\rm Pd}_8)}^{\rm OH}{({\rm Zn}_{12})}^{\rm OH}({{{\rm Pd}_1\rm Zn}_7)}^{IT}({{\rm Zn}_{24})}^{\rm CO}$, and $({{\rm Pd}_8)}^{\rm OH}{({\rm Zn}_{12})}^{\rm OH}({{\rm Zn}_8)}^{IT}({{{\rm Pd}_1\rm Zn}_{23})}^{\rm CO}$, where 8 Pd atoms occupy 8 OT sites and the 9th Pd atom occupies one of the OH (conf\_OH), IT (conf\_IT), or CO (conf\_CO) sites, respectively. Force constants can be predicted by DFT-based phonon calculations \cite{shang2018understanding}. Table \ref{intermetallics:PdZn_DFT_details} lists details of phonon calculations for Pd$_9$Zn$_{43}$ configurations. Figure \ref{intermetallics:fig:PdZnFC} shows the force constants of atomic pairs Pd-Pd, Pd-Zn, and Zn-Zn in three Pd$_9$Zn$_{43}$ configurations. The Pd-Zn atomic pairs have the largest force constants 5.373 eV/\r{A}$^2$, compared with 4.875 eV/\r{A}$^2$ of Pd-Pd and 3.569 eV/\r{A}$^2$ of Zn-Zn pairs. It indicates that Pd-Zn pairs have the strongest bonding in the configuration. Table \ref{intermetallics:tab:PdZnfc} shows the energy and bonding in three Pd$_9$Zn$_{43}$ configurations, i.e., conf\_OH, conf\_IT, and conf\_CO. conf\_OH has the lowest energy $E_{conf\_OH}$ = -105.30 eV/atom and the shortest Pd-Zn bonding distance d$_{conf\_OH}^{Pd-Zn}$ = 2.538 \r{A} in comparison with conf\_CO and conf\_IT. In contrast, $d_{conf\_OH}^{Pd-Pd}$ = 2.940 \r{A} is larger than that of conf\_OH and conf\_IT. Table \ref{intermetallics:tab:PdZnfc} also shows the bonding of the 9th Pd when occupying OH, CO, and IT. In conf\_OH, the 9th Pd is bonding with Zn atoms in its first nearest neighbors ($d^{1NN}$ < 3.2 \r{A}, seen in Figure \ref{intermetallics:fig:PdZnFC}), with the nearest Pd atom is 4.713 \r{A} away. In conf\_CO and conf\_IT, there are Pd-Pd pairs in the first nearest neighbors of the 9th Pd, which makes the bonding between Pd and surroundings weaker than that in conf\_OH. We conclude that the stability preference for OH occupancy of additional Pd atoms results from stronger Pd-Zn bonding interactions.

\begin{figure}[H]
    \centering
    \includegraphics[width=0.5\linewidth]{intermetallics/Intermetallics-PdZnFC.jpg}
    \caption{Force constants of Pd-Pd, Pd-Zn, and Zn-Zn atom pairs in Pd9Zn43 configurations obtained from phonon calculations.}
    \label{intermetallics:fig:PdZnFC}
\end{figure}

\begin{table}[H]
    \centering
    \caption{Energies, and distances of bonds (d) of the nearest pairs for three Pd$_9$Zn$_{43}$ configurations with the first 8 Pd atoms occupying the OT site, and the 9th Pd atom (Pd$^{9th}$) occupying the OH, CO, or IT site, denoted by conf\_OH, conf\_CO, and conf\_IT, respectively. Configurations are relaxed using DFT calculations.}
    \begin{tabular}{>{\raggedright\arraybackslash}m{3cm}>{\raggedright\arraybackslash}m{3cm}>{\raggedright\arraybackslash}m{2cm}>{\raggedright\arraybackslash}m{2cm}>{\raggedright\arraybackslash}m{2.5cm}>{\raggedright\arraybackslash}m{2.5cm}}
        \hline
         \textbf{Configurations}& \textbf{Energy (eV/atom)} & d of Pd-Zn (\r{A}) & d of Pd-Pd (\r{A}) & d of Pd$^{9th}$-Zn (\r{A}) & d of Pd$^{9th}$-Pd (\r{A})\\
        \hline
        conf\_OH&-105.30&2.538&2.940&2.563&4.713\\
        conf\_CO&-104.89&2.558&2.783&2.562&2.783\\
        conf\_IT&-104.74&2.560&2.887&2.613&3.192\\
         \hline
    \end{tabular}
    \label{intermetallics:tab:PdZnfc}
\end{table}

Site fractions of $\gamma$-brass phase calculated from the CALPHAD modeling are further applied to analyze the surface structures. The possible stable composition of the $\gamma$-brass phase is evaluated ranging from Pd$_8$Zn$_{44}$ to Pd$_{12}$Zn$_{40}$ from the present model. It indicates that site occupancy in OH will be changed with changing Pd composition in the $\gamma$-brass phase, with OT remaining occupied by Pd, IT and CO occupied by Zn. DFT-based calculations have suggested that the $(1\bar{1}0)$ surface of the $\gamma$-brass phase has a lower surface energy than $(110)$ and ${111}$ \cite{Dasgupta2022}. Figure \ref{intermetallics:fig:PdZnSurface} shows $(1\bar{1}0)$ surface constructions of $\gamma$-brass phase. OT sites separately locate on the surface, forming Pd monomers (Pd1). When increase Pd occupy OH sites, OT-OH-OT ensembles on the surface can then become Pd trimers (Pd3). The ability to control the exposure of specific surface ensembles between Pd1 and Pd3 sites is a direct consequence of the site occupancies of the bulk structure and has catalytic consequences. For example, the activity for ethylene hydrogenation and selectivity for acetylene semi-hydrogenation were drastically altered by tuning active ensembles between Pd monomers (Pd1) and Pd trimers (Pd3) \cite{Dasgupta2022}. 

\begin{figure}[H]
    \centering
    \includegraphics[width=0.4\linewidth]{intermetallics/Intermetallics-PdZnSurface.jpg}
    \caption{$(1\bar{1}0)$ surface of $\gamma$-brass in $2\times2\times2$ supercell. Blue atoms are OT sites, which are occupied by Pd atoms. Grey atoms are IT and CO sites, which are occupied by Zn atoms. Purple atoms are OH sites, which can be occupied by both Pd and Zn atoms.}
    \label{intermetallics:fig:PdZnSurface}
\end{figure}

\section{Determination of site occupancy in the M-Pd-Zn (M = Cu, Ag, and Au) \texorpdfstring{$\gamma$}--brass phase} \label{intermetallics:sec:PdZnM}
The Pd-Zn $\gamma$-brass phase provides exciting opportunities for synthesizing site-isolated catalysts with precisely controlled Pd active site ensembles. Introducing a third metal into the $\gamma$-brass lattice further perturbs the catalytic active site ensembles. Here, we introduce coinage metals M (M = Cu, Ag, and Au) into the Pd-Zn $\gamma$-brass phase and investigate the site occupation factors (SOF) of each element in the $\gamma$-brass lattice. The choice of M is motivated by the potential to synthesize new selective hydrogenation catalysts, as each of these metals have been previously utilized for catalyzing hydrogenation chemistries either as single metal catalysts \cite{chou1987benzene, chou1987benzeneII} or as Pd-M \cite{zhang2000synergetic, choudhary2003acetylene, chen2005promotional, friedrich2013order, mccue2014cu, kyriakou2012isolated} (or even M-Zn \cite{spanjers2014zinc}) bimetallic catalysts. The CALPHAD modeling approach and X-ray or neutron diffraction with Rietveld refinement were used to identify the SOF on each Wyckoff site for various M ratios alloyed into the Pd-Zn $\gamma$-brass phase. The present analysis unveils the strong preference for Pd occupying the OT site in the $\gamma$-brass lattice while the coinage metals tend to substitute for Zn on the OH site.  The determination of site occupancy in the bulk M-Pd-Zn $\gamma$-brass phase provides opportunities to investigate possible catalytic active site ensembles in the $\gamma$-brass phase materials.

\subsection{Modeling details} \label{intermetallics:ssec:PdZnMmodel}
The model of the M-Pd-Zn $\gamma$-brass phase is based on the compound energy formalism according to its Wyckoff sites (see Section \ref{method:ssec:CEF}). To this end, a four-sublattice model has been used:
\begin{equation} \label{intermetallics:eq:PdZnMmodel}
    \mathrm{(Pd,Zn,M)_2(Pd,Zn,M)_3(Pd,Zn,M)_2(Pd,Zn,M)_6}
\end{equation}
to describe the $\gamma$-brass phase, corresponding to the OT, OH, IT, and CO sites of space group $I\overline{4}3m$, respectively \cite{strom1969x}. The Gibbs energy expression for this $\gamma$-brass phase modeled using four sublattices is:
\begin{equation}
    \begin{aligned}
        G_m=&\sum_{i={\rm Pd,Zn,M}}\sum_{j={\rm Pd,Zn,M}}\sum_{k={\rm Pd,Zn,M}}\sum_{l={\rm Pd,Zn,M}}{y_i^\prime y_j^{\prime\prime}y_k^{\prime\prime\prime}y_l^{\prime\prime\prime\prime}{{^o}G}_{i:j:k:l}}\\&+RT(2\sum_{i={\rm Pd,Zn,M}}{y_i^\prime\ln{\left(y_i^\prime\right)}+}3\sum_{j={\rm Pd,Zn,M}}{y_j^{\prime\prime}\ln{\left(y_j^{\prime\prime}\right)}}\\&+2\sum_{k={\rm Pd,Zn,M}}{y_k^{\prime\prime\prime}\ln{\left(y_k^{\prime\prime\prime}\right)}+}6\sum_{l={\rm Pd,Zn,M}}{y_l^{\prime\prime\prime\prime}\ln{\left(y_l^{\prime\prime\prime\prime}\right)}})+{^{xs}}G_m
    \end{aligned}
\end{equation}
where $y_i^{\left(s\right)}$ is the site fraction of component $i$ on sublattice $s$ with $s$ = $\prime$, $\prime\prime$, $\prime\prime\prime$, and $\prime\prime\prime\prime$, representing the OT, OH, IT, and CO sublattice, respectively. ${^o}G_{i:j:k:l}$ is the energy of the endmember $\left(i\right)_2\left(j\right)_3\left(k\right)_2\left(l\right)_6$ with only one component ($i$, $j$, $k$, or $l$) in each sublattice.The second to the fifth terms represent the configurational entropy contribution to Gibbs energy, and $^{xs}G_m$ is the excess Gibbs energy. For simplicity, the effects of $^{xs}G_m$ are ignored, given the complexity of describing the Gibbs energy with the four-sublattice model and the numerous terms involved.

The thermodynamic model of the $\gamma$-brass phase in Pd-M-Zn was established using the open-source software ESPEI \cite{bocklund2019espei}. The parameters in the model were generated using thermochemical data from DFT-based first-principles calculations as discussed in the following paragraphs, and the Gibbs energies of pure elements (Pd, Zn, Cu, Ag, and Au) were taken from the Scientific Group Thermodata Europe (SGTE) database \cite{sgteurl}.

DFT-based first-principles calculations in the present work were performed to predict the energetics of the endmembers for the M-Pd-Zn $\gamma$-brass phase. The static total energy, $E\left(V\right)$, of a given endmember at 0 K is obtained as a function of volume ($V$), fitted using a four-parameter Birch-Murnaghan equation of state as shown in (\ref{method:eq:EOS}). The VASP package \cite{kresse1996efficient} was employed for all DFT calculations. The projector augmented-wave method (PAW) was used to account for the electron-ion interactions in order to increase computational efficiency in comparison with the full potential methods \cite{blochl1994projector}. Electron exchange and correlation effects were described using the generalized gradient approximation (GGA) as implemented by Perdew, Burke, and Ernzerhof (PBE) \cite{perdew1996generalized}. In the present work, 81 endmembers using a 26-atom cell in each of the M-Pd-Zn systems (M = Cu, Ag, and Au) were calculated. For each endmember, DFT calculations were performed using 8 volumes for the energy versus volume (E-V) EOS fitting with $V/V_0$ in the range of 0.91 $-$ 1.12. The k-points meshes were $7\times7\times7$ for structure relaxations and static calculations. The plane-wave basis cutoff energy was set as accurate (i.e., the setting of “PREC = Accurate” for VASP) for relaxations and 520 eV for the final static calculations. The convergence criterion of the electronic self-consistency was set as 5×10$^{-6}$ eV/atom for relaxations and final calculations.

\subsection{Properties of M-Pd-Zn (M = Cu, Ag, and Au) $\gamma$-brass compounds by first-principles calculations} \label{intermetallics:ssec:PdZnMDFTresult}
Figure \ref{intermetallics:fig:PdZnM-ConvexHull} shows the convex hull of formation enthalpy in the Cu-Pd-Zn, Ag-Pd-Zn, and Au-Pd-Zn $\gamma$-brass phase. Figure \ref{intermetallics:fig:PdZnM-ConvexHull} only displays endmembers within the $\gamma$-brass phase, disregarding other phases that may be present in the ternary system. Six ternary endmembers are identified on the convex hull in the Au-Pd-Zn $\gamma$-brass phase, i.e., $\mathrm{\left(Au\right)_4^{OT}\left(Zn\right)_6^{OH}\left(Pd\right)_4^{IT}\left(Au\right)_{12}^{CO}}$, $\mathrm{\left(Pd\right)_4^{OT}\left(Au\right)_6^{OH}\left(Zn\right)_4^{IT}\left(Zn\right)_{12}^{CO}}$, $\mathrm{\left(Pd\right)_4^{OT}\left(Pd\right)_6^{OH}\left(Zn\right)_4^{IT}\left(Au\right)_{12}^{CO}}$, $\mathrm{\left(Pd\right)_4^{OT}\left(Zn\right)_6^{OH}\left(Au\right)_4^{IT}\left(Pd\right)_{12}^{CO}}$, \\$\mathrm{\left(Zn\right)_4^{OT}\left(Pd\right)_6^{OH}\left(Zn\right)_4^{IT}\left(Au\right)_{12}^{CO}}$, and $\mathrm{\left(Zn\right)_4^{OT}\left(Zn\right)_6^{OH}\left(Pd\right)_4^{IT}\left(Au\right)_{12}^{CO}}$. In the Cu-Pd-Zn $\gamma$-brass phase, two ternary endmembers lie on the convex hull, including $\mathrm{\left(Cu\right)_4^{OT}\left(Zn\right)_6^{OH}\left(Zn\right)_4\left(Pd\right)_{12}^{CO}}$ in the Pd rich region and $\mathrm{\left(Pd\right)_4^{OT}\left(Cu\right)_6^{OH}\left(Zn\right)_4^{IT}\left(Zn\right)_{12}^{CO}}$ in the Zn rich region. For the Ag-Pd-Zn system, only $\mathrm{\left(Pd\right)_4^{OT}\left(Ag\right)_6^{OH}\left(Zn\right)_4^{IT}\left(Zn\right)_{12}^{CO}}$ is on the convex hull within the ternary composition region. Notably, $\mathrm{\left(Pd\right)_4^{OT}\left(M\right)_6^{OH}\left(Zn\right)_4^{IT}\left(Zn\right)_{12}^{CO}}$ (M=Cu, Ag, and Au) are on the convex hull in all three systems, indicating Pd prefers the OT site while the coinage metals M prefer the OH site in the $\gamma$-brass phase. 

\begin{figure}[H]
    \centering
    \includegraphics[width=1.0\linewidth]{intermetallics/Intermetallics-PdZnM-ConvexHull.png}
    \caption{Convex hull at 0 K in (a) Cu-Pd-Zn, (b) Ag-Pd-Zn, and (c) Au-Pd-Zn $\gamma$-brass phases. Green circles represent compounds on the convex hull. Diamonds represent compounds above the hull, with more purple showing lower formation enthalpy and more blue showing higher formation enthalpy. The reference states used for calculations are FCC Cu, Ag, Au, Pd, and HCP Zn.}
    \label{intermetallics:fig:PdZnM-ConvexHull}
\end{figure}

The convex hull facet shape near the Zn rich corner (x(Zn) > 0.6) should be emphasized, especially considering the appearance of the $\gamma$-brass phase in the Zn rich region in the binary phases diagrams (x(Zn) = 0.77 ~ 0.85 in the Pd-Zn system7, x(Zn) around 0.6 in the Ag-Zn system, and x(Zn) = 0.65~0.75 in the Au-Zn system66). Considering x(Zn) > 0.6 and x(Pd) < 0.15 in Figure 1, the convex hull facet shape in the Ag-Pd-Zn $\gamma$-brass phase differs from the Cu-Pd-Zn and Au-Pd-Zn $\gamma$-brass phases. In the Cu-Pd-Zn and Au-Pd-Zn $\gamma$-brass phases, the convex hull facet consists of $\mathrm{\left(Pd\right)_4^{OT}\left(Zn\right)_6^{OH}\left(Zn\right)_4^{IT}\left(Zn\right)_{12}^{CO}}$, $\mathrm{\left(Pd\right)_4^{OT}\left(M\right)_6^{OH}\left(Zn\right)_4^{IT}\left(Zn\right)_{12}^{CO}}$, and $\mathrm{\left(M\right)_4^{OT}\left(M\right)_6^{OH}\left(Zn\right)_4^{IT}\left(Zn\right)_{12}^{CO}}$. However, in the Ag-Pd-Zn $\gamma$-brass phase, one facet is composed of $\mathrm{\left(Pd\right)_4^{OT}\left(Zn\right)_6^{OH}\left(Zn\right)_4^{IT}\left(Zn\right)_{12}^{CO}}$, $\mathrm{\left(Pd\right)_4^{OT}\left(Ag\right)_6^{OH}\left(Zn\right)_4^{IT}\left(Zn\right)_{12}^{CO}}$, and pure Ag. The other consists of $\mathrm{\left(Pd\right)_4^{OT}\left(Zn\right)_6^{OH}\left(Zn\right)_4^{IT}\left(Zn\right)_{12}^{CO}}$, $\mathrm{\left(Ag\right)_4^{OT}\left(Ag\right)_6^{OH}\left(Zn\right)_4^{IT}\left(Zn\right)_{12}^{CO}}$, and pure Ag. In Ag-Pd-Zn, the endmembers on the convex hull are notably affected by the Pd ratio in the $\gamma$-brass phase. With decreasing Pd composition, the ternary endmember $\mathrm{\left(Pd\right)_4^{OT}\left(Ag\right)_6^{OH}\left(Zn\right)_4^{IT}\left(Zn\right)_{12}^{CO}}$ disappears from the convex hull. The solubility range of a homogeneous ternary $\gamma$-brass phase in the Ag-Pd-Zn system is narrower, and its stability is more sensitive to Pd composition than the Cu-Pd-Zn and Au-Pd-Zn systems. It is consistent with the experimental observation from the collaborators that the lower Pd composition, due to the increased Ag content, causes the decomposition of the homogeneous Ag-Pd-Zn ternary $\gamma$-brass into Ag-Zn $\gamma$-brass and Pd-Zn $\gamma$-brass phases.   

\subsection{Site occupancy predicted by CALPHAD modeling approach and compared with experiments} \label{intermetallics:ssec:PdZnMCALPHADresult}
The site fraction of each element in each sublattice can be calculated using the CALPHAD modeling approach under given conditions such as the fixed temperature $T$, pressure $P$, and mole fraction of Pd, i.e., $x$(Pd). We consider the conditions of $T$ = 300 K, $P$ =101325 Pa, and $x$(Pd) = 0.1538 (8 Pd atoms in a 52-atom supercell) to predict the site fractions of alloying elements M on each sublattice (OT, OH, IT, and CO). The selected composition x(Pd) = 0.1538 is consistent with the low $x$(Pd) limit evaluated from thermodynamic modeling of the Pd-Zn system for formation of a stable Pd-Zn $\gamma$-brass system \cite{gong2022thermodynamic}. Figure \ref{intermetallics:fig:PdZnM-SiteFractionCalphad} shows the predicted site occupancies of the alloying elements M (= Cu, Ag, and Au), Pd, and Zn as a function of the mole fraction of M, i.e., $x$(M). Figure \ref{intermetallics:fig:PdZnM-SiteFractionCalphad} plots the composition range of $x$(M) = 0 $-$ 0.1 (Pd8M0Zn44 to around Pd8M5Zn39) for dilute alloying elements. Figure \ref{intermetallics:fig:PdZnM-SiteFractionCalphad} shows the same trend of site occupancy in the three M-Pd-Zn (M=Cu, Ag, and Au) $\gamma$-brass phases. The site fraction of M on the OH sublattice increases with increasing $x$(M) in this composition range. The OT sites are fully occupied by Pd initially and remain unchanged when alloying $x$(M) into the $\gamma$-brass phase. The site fraction of Zn on the OH site decreases when increasing $x$(M). 

Table \ref{intermallics:tab:SFPdMZn} shows the detailed site fraction values at the composition of Pd$_8$M$_1$Zn$_{43}$ calculated by the CALPHAD approach. This composition is chosen to unveil the site preference of one M atom alloyed with the Pd-Zn $\gamma$-brass phase (Pd$_8$Zn$_{44}$) \cite{gong2022thermodynamic}. In all M-Pd-Zn (M=Cu, Ag, and Au) $\gamma$-brass phases, the site fraction $y_{\rm M}^{\rm OH}$ is around 0.08. In addition, there is a small amount of M entering the IT and CO sites, for example, $y_{\rm Ag}^{\rm IT}$=0.0015 and $y_{\rm Ag}^{\rm CO}$=0.0010. However, $y_{\rm Pd}^{\rm OT}$ remains at 1 in all three M-Pd-Zn $\gamma$-brass phases when adding M, indicating Pd atoms strongly outcompete M for the OT site. Overall, CALPHAD modeling predicts coinage M atoms prefer substituting Zn on the OH site when alloyed into the Pd-Zn $\gamma$-brass phase. 

\begin{figure}[H]
    \centering
    \includegraphics[width=1\linewidth]{intermetallics/Intermetallics-PdZnM-SiteFractionCalphad.png}
    \caption{Site fraction of each element in the (a) Cu-Pd-Zn, (b) Ag-Pd-Zn, and (c) Au-Pd-Zn $\gamma$-brass phases in OT (solid lines), OH (dashed lines), IT (dot-dashed lines), and CO (dotted lines) sublattice changing with increasing $x$(M) calculated from the CALPHAD modeling at T=300K, P=101325 Pa, and $x$(Pd)=0.1538. Red lines represent the site fraction of Cu, green lines represent the site fraction of Ag, orange lines represent the site fraction of Au, blue lines represent the site fraction of Pd, and grey lines represent the site fraction of Zn.}
    \label{intermetallics:fig:PdZnM-SiteFractionCalphad}
\end{figure}

\begin{table}[H]
    \centering
    \caption{Site fractions $y$ of Pd, Zn, and M (M= Cu, Ag, and Au) in each sublattice, denoted as $y_i^{\rm OT}$, $y_i^{\rm OH}$, $y_i^{\rm IT}$, and $y_i^{\rm CO}$, in ternary M-Pd-Zn $\gamma$-brass phases calculated from the CALPHAD modeling at temperature of 300 K, $x$(Pd)=0.1538, $x$(M)=0.0192 (Pd$_8$M$_1$Zn$_{43}$).}
    \begin{tabular}{>{\raggedright\arraybackslash}m{3cm}>{\raggedright\arraybackslash}m{2.5cm}>{\raggedright\arraybackslash}m{2cm}>{\raggedright\arraybackslash}m{2cm}>{\raggedright\arraybackslash}m{2cm}>{\raggedright\arraybackslash}m{2cm}}
        \hline
        \textbf{Composition}&\textbf{Element $i$}&$y_i^{\rm OT}$&$y_i^{\rm OH}$&$y_i^{\rm IT}$&$y_i^{\rm CO}$\\
        \hline
        Pd$_8$Cu$_1$Zn$_{43}$&Cu&0.000&0.082&0.001&0.000\\
	&Pd&1.000&0.0000&0.0000&0.000\\
	&Zn&0.000&0.918&0.999&1.000\\
        Pd$_8$Ag$_1$Zn$_{43}$&Ag&0.000&0.080&0.001&0.001\\
	&Pd&1.000&0.000&0.000&0.000\\
	&Zn&0.000&0.920&0.999&0.999\\
        Pd$_8$Au$_1$Zn$_{43}$&Au&0.000&0.083&0.000&0.000\\
	&Pd&1.000&0.000&0.000&0.000\\
	&Zn&0.000&0.917&1.000&1.000\\
        \hline
    \end{tabular}
    \label{intermallics:tab:SFPdMZn}
\end{table}

To evaluate the stability of the M-Pd-Zn $\gamma$-brass phase and determine SOFs, the values of $x$ and $y$ in Pd$_x$M$_y$Zn$_{52-(x+y)}$ were varied for the syntheses of homogenous ternary phases. The SOFs are determined by X-ray and neutron diffraction with Rietveld refinements to validate the CALPHAD predictions. These experiments were carried out by our collaborators. This dissertation focuses on the comparison of experimental results and CALPHAD predictions.

Figure \ref{intermetallics:fig:PdZnM-SFCalExp} shows the site fraction on the OH site from CALPHAD modeling in comparison with Rietveld refinements. For Pd$_8$Au$_1$Zn$_{43}$ and Pd$_8$Au$_3$Zn$_{41}$, $y_{\rm Au}^{\rm OH}$ from CALPHAD modeling is lower than the experimentally determined SOF of Au. This result is due to the ICP (Inductively Coupled Plasma-Optical Emission Spectroscopy)-determined actual composition showing an Au content slightly higher than the nominal values in Pd$_8$Au$_1$Zn$_{43}$ and Pd$_8$Au$_3$Zn$_{41}$ samples (Pd$_{7.9}$Au$_{1.3}$Zn$_{42.8}$ and Pd$_{8.4}$Au$_{3.1}$Zn$_{40.5}$), while CALPHAD modeling predictions are carried out based on the exact compositions Pd$_8$Au$_1$Zn$_{43}$ and Pd$_8$Au$_3$Zn$_{41}$. Similarly, $y_{\rm Au}^{\rm OH}$ = 0.083 in Pd$_9$Au$_1$Zn$_{42}$, slightly higher than the experimental SOF of Au as 0.053 due to the actual Pd$_{9.6}$Au$_{0.95}$Zn$_{41.4}$ composition. Figure \ref{intermetallics:fig:PdZnM-SFCalExp} also shows the comparison of Rietveld refinements of Pd$_9$Cu$_1$Zn$_42$ with the CALPHAD modeling predictions. The good agreement suggests 8 Pd remain on the OT site; additional Pd and Cu occupy the OH site. 

\begin{figure}[H]
    \centering
    \includegraphics[width=0.8\linewidth]{intermetallics/Intermetallics-PdZnM-SFCalExp.png}
    \caption{Site occupancy on the OH site in Pd8Au1Zn43, Pd8Au3Zn41, Pd9Au1Zn42, and Pd9Cu1Zn42 from Rietveld refinement results compared with CALPHAD modeling predictions (marked with slashes). Site fraction of Pd is marked in blue, Zn in grey, Au in yellow, and Cu in brown.}
    \label{intermetallics:fig:PdZnM-SFCalExp}
\end{figure}

\section{Summary} \label{intermetallics:sec:Summary}

Thermodynamic modeling of the Pd-Zn system and the M-Pd-Zn (M = Cu, Ag, and Au) based on the CALPHAD approach has been performed. A 4-sublattice model is used to describe the $\gamma$-brass phase in accordance with its four Wyckoff positions providing a better prediction of surface construction and enabling the understanding of active surface ensembles for catalysts. DFT-based first-principles calculations are used to obtain Gibbs energies of endmembers for the $\gamma$-brass phase. In the binary Pd-Zn $\gamma$-brass phase, Pd atoms first occupy the OT sublattice followed by the OH sublattice, as indicated by DFT-based total energy and phonon calculations supported by the bonding distance and force constants analyses and in agreement with experimental data. Site fractions calculated from the CALPHAD modeling as a function of temperature and composition contribute to analyzing surface construction of the Pd-Zn $\gamma$-brass phase. 

Site occupancy in the ternary M-Pd-Zn (M = Cu, Ag, and Au) $\gamma$-brass phases was determined through the CALPHAD modeling approach and Rietveld refinement. Site occupancy calculations are carried out through the Gibbs energy minimization of the $\gamma$-brass phases. CALPHAD modeling predicts an increase in the site fraction of the coinage metals on the OH site upon alloying M into the Pd-Zn $\gamma$-brass phase. The experimental Rietveld refinements verify the CALPHAD modeling predictions, demonstrating a trend that the coinage metals M cannot displace Pd from the OT site and preferentially occupy the OH site by Zn substitution. This observed trend, irrespective of the atomic radius of M, suggests the electronic interaction between the atoms is more likely to govern atomic site occupation preferences in the $\gamma$-brass phase. The introduction of coinage metals M into the Pd-Zn $\gamma$-brass phase, therefore, leads to Pd$^{\rm OT}$-M$^{\rm OH}$-Pd$^{\rm OT}$ trimer moieties isolated from other Pd and M atoms in the bulk structure by Zn atoms. The M-Pd-Zn (M = Cu, Ag, Au) $\gamma$-brass ternary intermetallics, can therefore offer combinations of isolated Pd atoms, Pd$^{\rm OT}$-Pd$^{\rm OH}$-Pd$^{\rm OT}$ and Pd$^{\rm OT}$-M$^{\rm OH}$-Pd$^{\rm OT}$ clusters. These isolated trimers may be exposed on their surfaces. This understanding of site occupancy and the formation of different types of trimer clusters has significant implications for designing intermetallic catalysts. The isolated Pd, Pd trimers, and Pd-M-Pd clusters within the Zn atoms can enhance the selectivity of catalytic reactions, providing a pathway for tailoring the catalytic properties of $\gamma$-brass phases by strategically introducing coinage metals, thereby optimizing their performance for specific catalytic reactions.

\chapter{Thermodynamic modeling of fluoride and chloride molten salts with model selection, uncertainty quantification, and uncertainty propagation} \label{chap:moltensalts}

\section{Introduction} \label{moltensalts:sec:intro}
Chromium (Cr) is one of the key elements in current reference structural materials for fluoride salt reactors, for example, the Hastelloy-N as the container material (72\%Ni-16\%Mo-7\%Cr-5\%Fe in wt.\%) \cite{koger1972evaluation, manly1958metallurgical}. In the fluoride salt environment, Cr in Ni-based alloy is more susceptible to corrosion compared to other metal elements \cite{olson2009materials, ouyang2014long, liu2020corrosion}. For example, Olson et al. \cite{olson2009materials} performed corrosion tests on a number of Ni-based alloys with different Cr alloying contents. They showed the dissolution of Cr into molten salt and correlated the relationship between Cr content and corrosion resistance \cite{olson2009materials}. Ouyang et al. \cite{ouyang2014long} investigated the corrosion behavior of Hastelloy-N in FLiNaK after 100-1000 h at 700 $^o$C, and the aggregate dissolution of Cr was observed. Recently, Liu et al. \cite{liu2020corrosion} studied Hastelloy-N and showed that the corrosion rate of Cr in FLiNaK-CrF$_3$ is higher than that in FLiBe-CrF$_3$ with Cr$^{3+}$ as the product. Thus, it is important to investigate solubility and multivariate distribution patterns of Cr in FLiNaK. The LiF-NaF-KF system has been extensively studied by many researchers in terms of modeling. Chartrand and Pelton \cite{chartrand2001thermodynamic}, Wang et al. \cite{wang2014phase}, and Ard et al. (MSTDB-TC database) \cite{ard2022development} have investigated this system through CALPHAD modeling. However, we noticed that thermodynamic modeling of the FLiNaK-CrF$_3$ system has been performed based on incomplete and empirically estimated thermochemical data by Yin et al. \cite{yin2018thermodynamic, yin2015thermodynamic} and Dumaire et al. \cite{dumaire2021thermodynamic}. A comprehensive study on the thermodynamic properties of all solid and liquid phases in the FLiNaK-CrF$_3$ system is yet to be conducted.

Chloride molten salts have gained significant attention for their potential applications in advanced nuclear reactors and pyroprocessing techniques \cite{sridharan2012thermal, mourogov2006potentialities, iizuka1997actinides}. Effective separation of fission products, such as lanthanides using pyroprocessing techniques, requires a comprehensive understanding of the thermodynamic properties of molten salts. The CALPHAD  modeling approach \cite{liu2020computational, lukas2007computational} is effective in investigating phase equilibrium and thermodynamic properties in multicomponent systems. For the modeling of complex molten salts, various thermodynamic models within the CALPHAD framework have been developed to capture intricate behaviors such as short-range ordering, including the associate model, two-sublattice ionic model, and MQMQA (see Section \ref{method:ssec:liqmodels}). However, systematically comparing these models and selecting the most appropriate model for molten salts remain a challenge due to different physical interpretations and the complexities involved in quantifying model performance.

In the present work, the fluoride system (LiF, NaF, KF, CrF${_2}$)-CrF${_3}$ and the chloride system LiCl-KCl-LaCl${_3}$ are modeled using the CALPHAD method, supplemented with thermochemical data by first-principles calculations. The open-source software tools of ESPEI \cite{bocklund2019espei} and PyCalphad \cite{otis2017pycalphad} were employed for the modeling process, which facilitates Bayesian parameter estimation, UQ, and Bayesian model selection in modeling molten salts. The present work compares thermodynamic models commonly used in CALPHAD modeling for liquid, providing insights into an effective model selection strategy. 

\section{Revisiting thermodynamics in (LiF, NaF, KF, CrF${_2}$)-CrF${_3}$ by first-principles calculations and CALPHAD modeling} \label{moltensalts:sec:FLiNaKCr}
The thermodynamic description of the (LiF, NaF, KF, CrF$_2$)-CrF$_3$ systems has been revisited, aiming for a better understanding of the effects of Cr on the FLiNaK molten salt. First-principles calculations based on DFT were performed to determine the electronic and structural properties of each compound, including the formation enthalpy, volume, and bulk modulus. DFT-based phonon calculations were carried out to determine the thermodynamic properties of compounds, for example, enthalpy, entropy, and heat capacity as functions of temperature. Phonon-based thermodynamic properties show a good agreement with experimental data of binary compounds LiF, NaF, KF, CrF$_3$, and CrF$_2$, establishing a solid foundation to determine thermodynamic properties of ternary compounds as well as to verify results estimated by the Neumann-Kopp rule. Additionally, DFT-based AIMD simulations were employed to predict the mixing enthalpies of liquid salts. Using DFT-based results and experimental data in the literature, the (LiF, NaF, KF, CrF$_2$)-CrF$_3$ system has been remodeled in terms of the CALPHAD approach using the MQMQA for liquid. Calculated phase stability in the present work shows an excellent agreement with experiments, indicating the effectiveness of combining DFT-based total energy, phonon, and AIMD calculations, and CALPHAD modeling to provide the thermodynamic description in complex molten salt systems.

\subsection{Modeling details} \label{moltensalts:ssec:FLiNaKCrmodel}
%%% Compounds information
The (LiF, NaF, KF, CrF$_2$)-CrF$_3$ system includes five binary (endmember) compounds, i.e., LiF, NaF, KF, CrF$_3$ and CrF$_2$, and eight ternary (intermetallic) compounds, i.e., Li$_3$CrF$_6$, Na$_3$CrF$_6$, Na$_5$Cr$_3$F$_{14}$, NaCrF$_4$, K$_3$CrF$_6$, K$_2$CrF$_5$, KCrF$_4$, and K$_2$Cr$_5$F$_{17}$. De Kozak first suggested these ternary compounds \cite{DeKozak1969} and confirmed by structural studies \cite{de1975systeme,miranday1975croissance, sturm1962phase, garcia2014electrostatic, brunton1969crystal, le2003distorted, manaka2011effects, sassoye2006crystal} via the X-ray diffraction (XRD) method, which was summarized by Dumaire et al. \cite{dumaire2021thermodynamic}. However, thermochemical data on these compounds is scarce. Yin et al. \cite{yin2018thermodynamic, yin2015thermodynamic, yin2014thermodynamic} performed DFT calculations at 0 K to determine the formation enthalpies of Li$_3$CrF$_6$, Na$_3$CrF$_6$, Na$_5$Cr$_3$F$_{14}$, NaCrF$_4$, and KCrF$_4$. Dumaire et al. \cite{dumaire2021thermodynamic} estimated the heat capacities of these compounds based on the Neumann-Kopp rule in terms of the compositional average of heat capacity values of the corresponding compounds or elements \cite{leitner2010application}. 

%%% Modeling and First-principles details for compounds
The present work treats the compounds and endmembers in the AF-CrF${_3}$ (A=Li, Na, and K) systems as stoichiometric compounds. Thermodynamic functions of the binary endmembers are taken from the JRC database \cite{konings2020comprehensive}, JANAF tables \cite{chase1982janaf}, IVTAN tables \cite{gurvich1993ivtanthermo}, and SSUB database \cite{sgteurl}. The Gibbs energy can be expressed as 
\begin{equation} \label{ms:eq:Gstoi}
    G_m=\Delta_f H_m^0 (298.15)-T S_m^0 (298.15)+\int_{298.15}^T C_{P,m} dT - T\int_{298.15}^T \dfrac{C_{P,m}}{T} dT
\end{equation}
where $\Delta_f H_m^0 (298.15)$ is the standard formation enthalpy, $S_m^0 (298.15)$ the standard entropy at 298.15 K, and $C_{P,m}$ the heat capacity. The thermodynamic data of ternary compounds, including enthalpy, entropy, and heat capacity are obtained through DFT-based first-principles and phonon calculations (see Section \ref{method:sec:firstprinciples}).

All DFT-based first-principles and phonon calculations in the present work were performed by the VASP \cite{kresse1996efficient} using the open-source software DFTTK \cite{wang2021dfttk}. The projector augmented-wave method (PAW) was used for electron-ion interactions to increase computational efficiency compared with the full potential methods \cite{blochl1994projector, kresse1999ultrasoft}. Electron exchange and correlation effects were described using both the local density approximation (LDA) \cite{perdew1981self} and the GGA \cite{perdew1996generalized}. In addition, the DFT+U approach was employed for 11 compounds containing Cr, i.e., CrF${_2}$, CrF${_3}$, Cr$_2$F$_5$, Li$_3$CrF$_6$, Na$_3$CrF$_6$, Na$_5$Cr$_3$F$_{14}$, NaCrF$_4$, K$_3$CrF$_6$, K$_2$CrF$_5$, KCrF$_4$, K$_2$Cr$_5$F$_{17}$. The effective U values for Cr were selected as 3.7 eV, considering 3$-$4eV was commonly used in the literature \cite{shi2009magnetism, mattsson2019density, huang2022dft}. The spin configurations were also considered for these 11 compounds containing Cr. All possible configurations by varying spin up and spin down of Cr atoms were explored by the ATAT code \cite{van2009multicomponent}. The spin configuration with the lowest energy for each Cr-containing compound was used for DFT and phonon calculations. Using DFTTK, structure information is the only required input, then robust relaxation schemes can be automatically performed to obtain equilibrium results at 0 K and thermodynamic properties at finite temperatures through the QHA. During DFTTK calculations, the plane-wave cutoff energy was set as 520 eV. Table \ref{ms:tab:DFTdetails} lists the k-points meshes for DFT-based total energy calculations, the supercell sizes and the k-points meshes for phonon calculations. The phonon DOS was analyzed using the YPHON code \cite{wang2014yphon}, which has been integrated into DFTTK \cite{wang2021dfttk}. 

\begin{table}[H]
    \centering
    \caption{Details of DFT-based first-principles calculations for each compound or phase, including space group, total atoms in the supercells, k-point meshes for structure relaxations, and the final static calculations (indicated by DFT), supercell sizes for phonon calculations, k-point meshes for phonon calculations.}
    \begin{tabular}{>{\raggedright\arraybackslash}m{2.5cm}>{\raggedright\arraybackslash}m{2cm}>{\raggedright\arraybackslash}m{2.5cm}>{\raggedright\arraybackslash}m{2.5cm}>{\raggedright\arraybackslash}m{2.8cm}>{\raggedright\arraybackslash}m{2.5cm}}
    \hline
     \textbf{Phase}&\textbf{Space group}&\textbf{Atoms in crystallographic cell}&\textbf{k-points for DFT}&\textbf{Atoms in supercell for phonon}&\textbf{k-points for phonon}\\
    \hline
        LiF&$Fm\Bar{3}m$&8&$10\times10\times10$&32&$10\times10\times10$\\
        NaF&$Fm\Bar{3}m$&8&$10\times10\times10$&32&$10\times10\times10$\\
        KF&$Fm\Bar{3}m$&8&$10\times10\times10$&32&$10\times10\times10$\\
        CrF${_3}$&$R\Bar{3}c$&24&$9\times9\times3$&24&$9\times9\times3$\\
        CrF${_2}$&$P2_1/c$&6&$14\times10\times9$&24&$13\times10\times9$\\
        Li$_3$CrF$_6$&$C2/c$&60&$2\times2\times2$&60&$2\times2\times2$\\
        Na$_3$CrF$_6$&$P2_1/c$&20&$9\times8\times5$&40&$9\times8\times5$\\
        Na$_5$Cr$_3$F$_{14}$&$P2_1/c$&44&$6\times6\times3$&44&$6\times6\times3$\\
        NaCrF$_4$&$P2_1/c$&24&$6\times8\times5$&24&$6\times8\times6$\\
        K$_3$CrF$_6$&$Fm\Bar{3}m$&40&$5\times5\times5$&40&$5\times5\times5$\\
        K$_2$CrF$_5$&$Pbcn$&128&$3\times1\times1$&128&$3\times1\times1$\\
        KCrF$_4$&$Pnma$&144&$3\times1\times1$&144&$3\times1\times1$\\
        K$_2$Cr$_5$F$_{17}$&$Cmcm$&96&$2\times2\times2$&96&$2\times2\times2$\\
        Cr$_2$F$_5$&$C2/c$&28&$6\times6\times6$&N/A&N/A\\
    \hline
    \end{tabular}
    \label{ms:tab:DFTdetails}
\end{table}

%%% CALPHAD modeling details
The MQMQA \cite{pelton2001modified} is used to describe the liquid phase by considering the short-range ordering (SRO) that occurs in liquid salts. Here, the model (A, Cr)(F) is introduced and hence the A$_2$F$_2$, Cr$_2$F$_2$, and ACrF$_2$ quadruplets (A=Li, Na, and K) are formed to consider the interactions among them. Coordination numbers Z describe the SNN coordination number of the species i (= Li, Na, K, Cr, or F) in quadruplets. Z of anions can be calculated to maintain charge neutrality as follows:
\begin{equation} \label{ms:eq:MQMZ}
    \dfrac{q_A}{(Z_{AB/FF}^A)}+\dfrac{q_B}{(Z_{AB/FF}^B)}=2\times \dfrac{q_F}{(Z_{AB/FF}^F)}
\end{equation}
where $q_i$ is the charges of ion $i$ (= Li, Na, K, Cr, or F). Table \ref{ms:tab:CrZ} shows the coordination numbers used in the present work.

\begin{table}[H]
    \centering
    \caption{Coordination number used in the present CALPHAD modeling work with MQMQA for the liquid phase.}
    \begin{tabular}{>{\raggedright\arraybackslash}m{2.5cm}>{\raggedright\arraybackslash}m{2.5cm}>{\raggedright\arraybackslash}m{2.5cm}>{\raggedright\arraybackslash}m{2.5cm}>{\raggedright\arraybackslash}m{2.5cm}}
    \hline
    \textbf{A}&\textbf{B}&\textbf{$Z_{AB/FF}^A$}&\textbf{$Z_{AB/FF}^B$}&\textbf{$Z_{AB/FF}^F$}\\
    \hline
    Li$^+$&Li$^+$&6&6&6 \\
    Na$^+$&Na$^+$&6&6&6\\
    K$^+$&K$^+$&6&6&6\\
    Cr$^{3+}$&Cr$^{3+}$&6&6&2\\
    Li$^+$&Cr$^{3+}$&2&6&2\\
    Na$^+$&Cr$^{3+}$&4&6&2.7\\
    K$^+$&Cr$^{3+}$&6&6&3\\
    Cr$^{2+}$&Cr$^{3+}$&6&6&2.4\\
    Cr$^{2+}$&Cr$^{2+}$&6&6&3\\
    \hline
    \end{tabular}
    \label{ms:tab:CrZ}
\end{table}

The excess Gibbs energy $G^{excess}$ relates to the formation Gibbs energy of the quadruplet, $\Delta g_{quadruplet}^{ex}$, by considering the following reaction:
\begin{equation} \label{ms:eq:MQMGrea}
    \left(\rm {A_2/F_2}\right)_{quad}+\left({\rm Cr_2/F_2}\right)_{quad}=2\left({\rm ACr/F_2}\right)_{quad}\;\;\;\;\;\Delta g_{\rm ACr/F_2}^{ex}
\end{equation}
where $\Delta g_{\rm ACr/F_2}^{ex}$ represents the Gibbs energy change when forming the quadruplet and can be described by:
\begin{equation} \label{ms:eq:MQMGex}
    \Delta g_{\rm {ACr/F_2}}^{ex}=\Delta g_{\rm {ACr/F_2}}^o+\sum_{(i+j)\geq1} g_{\rm {ACr/F_2}}^{ij}\chi _{\rm {ACr/F_2}}^{i}\chi _{\rm {CrA/F_2}}^{j}
\end{equation}
where $g_{\rm {ACr/F_2}}^{ij}$ is a function of temperature and it is independent of composition. $\chi _{\rm {ACr/F_2}}^{i}$ and $\chi _{\rm {CrA/F_2}}^{j}$ are composition-dependent terms as:
\begin{equation} \label{ms:eq:chi}
    \chi_{\rm {ACr/F_2}} = \dfrac{X_{\rm {A_2/F_2}}}{X_{\rm {A_2/F_2}}+X_{\rm {ACr/F_2}}+X_{\rm {Cr_2/F_2}}}
\end{equation}
where $X_{\rm{ACr/F_2}}$ is the fractions of $\left(\rm{ACr/F_2}\right)_{quad}$ shown in (\ref{ms:eq:MQMGrea}). 

In the CrF${_2}$-CrF${_3}$ system, there are three solid solution phases, i.e., CrF${_2}$ near the Cr-rich region, CrF${_3}$ near the F-rich region, and Cr$_2$F$_5$ showing on the middle region of the CrF${_2}$-CrF${_3}$ phase diagram. The present work adopts the same models used by Dumaire et al. \cite{dumaire2021thermodynamic}, where the regular solution model with the Kohler-Toop interpolation \cite{kohler1960estimation, toop1965predicting, chartrand2000choice, pelton2001general} is used for Cr$_2$F$_5$. For solid solution phases near CrF${_2}$ and CrF${_3}$, the sublattice model is used for each phase, respectively, considering the Wyckoff positions of CrF${_2}$ and CrF${_3}$ as follows. CrF${_2}$ possesses the symmetry with space group $P2_1/c$ with two Wyckoff sites of 2a and 4e, and the sublattice model (Cr, Va)$_1$(F, Va)$_2$ is hence used with Va representing the vacancy. CrF${_3}$ phase is modeled by (Cr, Va)$_1$(F, Va)$_3$ by considering its space group $R\bar{3}c$ and the two Wyckoff sites of 2b and 6e. The Gibbs energy is formulated as:
\begin{equation} \label{ms:eq:Crssoln}
    \begin{aligned}
        G_m&=\sum_{i=Cr,Va}{\sum_{j=F,Va}{y_i^\prime y_j^{\prime\prime}}\:^oG_{i:j}}\\
        &+RT\left(\sum_{i=Cr,Va}{y_i^\prime\ln{\left(y_i^\prime\right)}}+\sum_{j=F,Va}{y_j^{\prime\prime}\ln{\left(y_j^{\prime\prime}\right)}}\right)\\&+y_{Cr}^\prime y_{Va}^\prime\left(y_F^{\prime\prime}L_{Cr,Va:F}\right)+y_F^{\prime\prime}y_{Va}^{\prime\prime}\left(y_{Cr}^\prime L_{Cr:F,Va}\right)
    \end{aligned}
\end{equation}
where $y_i^{(s)}$ is the site fraction of component i on sublattice s, ${^o}G_{i:j}$ the Gibbs energy of the endmember (i:j), and $L$ the interaction parameters which can be expanded using the Redlich-Kister polynomials \cite{redlich1948algebraic}. 

Phase equilibria in the LiF-CrF${_3}$, NaF-CrF${_3}$, and KF-CrF${_3}$ binary systems were investigated by De Kozak \cite{de1975systeme, DeKozak1969} using differential thermal analysis (DTA). In LiF-CrF${_3}$, two eutectic reactions were measured, i.e., Liquid $\leftrightarrow$ LiF + Li$_3$CrF$_6$ at 1003 K and around mole fraction $x$(CrF${_3}$) = 0.15 and Liquid $\leftrightarrow$ CrF${_3}$ + Li$_3$CrF$_6$ at 1059 K and $x$(CrF${_3}$) = 0.35. In NaF-CrF${_3}$, one peritectic reaction of Liquid + CrF${_3}$ $\leftrightarrow$ NaCrF$_4$ at 1234 K and three eutectic reactions were determined, i.e., Liquid $\leftrightarrow$ NaCrF$_4$ + Na$_5$Cr$_3$F$_{14}$ at 1133 K, Liquid $\leftrightarrow$ Na$_3$CrF$_6$ + Na$_5$Cr$_3$F$_{14}$ at 1143 K, and Liquid $\leftrightarrow$ Na$_3$CrF$_6$ + NaF at 1166 K and around $x$(CrF${_3}$) = 0.123. In KF-CrF${_3}$, De Kozak \cite{de1975systeme, DeKozak1969} reported three peritectic reactions and two eutectic reactions, i.e., Liquid + CrF${_3}$ $\leftrightarrow$ K$_2$Cr$_5$F$_{17}$ at 1390 K, Liquid + K$_3$CrF$_6$ $\leftrightarrow$ K$_2$CrF$_5$ at 1133 K, and Liquid + K$_2$Cr$_5$F$_{17}$ $\leftrightarrow$ KCrF$_4$ at 1200 K, and Liquid $\leftrightarrow$ K$_3$CrF$_6$ + KF at 1115 K and around $x$(CrF${_3}$) = 0.048, and Liquid $\leftrightarrow$ K$_2$CrF$_5$ + KCrF$_4$ at 1112 K and around $x$(CrF${_3}$) = 0.45. Sturm \cite{sturm1962phase} reported phase equilibria in CrF${_2}$-CrF${_3}$ via quenching experiments and suggested one solid solution phase in CrF${_2}$-CrF${_3}$ with composition of CrF${_3}$ between 0.42 and 0.46 (near Cr$_2$F$_5$). However, the stability of this Cr$_2$F$_5$ solid solution phase was not explored in temperatures below 1023 K. The melting point of Cr$_2$F$_5$ was determined to be around 1270 K \cite{sturm1962phase}. Sturm \cite{sturm1962phase} reported one eutectic reaction, Liquid $\leftrightarrow$ CrF${_2}$ + Cr$_2$F$_5$ at 1103 K around $x$(CrF${_3}$) = 0.14, and one peritectic reaction Liquid + CrF${_3}$ $\leftrightarrow$ Cr$_2$F$_5$ at 1272 K around $x$(CrF${_3}$) = 0.29. Two solid solution phases near the endmembers CrF${_3}$ and CrF${_2}$ were identified from $x$(CrF${_3}$) = 0$-$0.01 and $x$(CrF${_3}$) = 0.90$-$1, respectively. 

Machine learning (ML) was used to estimate more phase equilibria data in terms of a graphic neural network model developed by Hong et al. \cite{hong2022melting} to predict melting points of compounds with composition as input. Melting temperatures of the present ternary compounds including Li$_3$CrF$_6$, Na$_3$CrF$_6$, Na$_5$Cr$_3$F$_{14}$, NaCrF$_4$, K$_3$CrF$_6$, K$_2$CrF$_5$, KCrF$_4$, and K$_2$Cr$_5$F$_{17}$ are estimated by this ML model \cite{hong2022melting}. 

%%%AIMD
For the liquid phase, experimental data such as mixing enthalpy are not available in the AF-CrF${_3}$ (A=Li, Na, and K) systems. Instead, Yin et al. \cite{yin2018thermodynamic, yin2015thermodynamic, yin2014thermodynamic} applied an empirical model to estimate the mixing enthalpy of liquid from the parameters of ions such as ionic radius. In the present work, AIMD simulations were performed to obtain the mixing enthalpy of liquid by VASP \cite{kresse1996efficient}. The supercells containing 108 or 128 atoms with periodic boundaries were used for at least six different compositions in the AF-CrF${_3}$ (A= Li, Na, and K) systems, including A$_{64}$F$_{64}$, A$_{42}$Cr$_6$F$_{60}$, A$_{36}$Cr$_9$F$_{63}$, A$_{32}$Cr$_{16}$F$_{80}$, A$_{18}$Cr$_{18}$F$_{72}$, A$_{16}$Cr$_{24}$F$_{88}$, A$_{10}$Cr$_{22}$F$_{76}$, and Cr$_{32}$F$_{96}$ (A=Li, Na, and K). The NVT canonical ensemble (i.e., the fixed number of atoms N, volume V, and temperature T) with a Nosé thermostat for temperature control \cite{nose1984unified} was employed in the present work. The temperature for each supercell was set as 1700 K, which is above all the temperatures of liquidus in the AF-CrF${_3}$ (A= Li, Na, and K) systems. A single $\Gamma$ point $1\times1\times1$ was used as the k-point mesh, together with a 400 eV cutoff energy. During AIMD simulations, Newton’s equation of motion was solved via the Verlet algorithm with a time step of 2 fs, and each calculation is run for 10,000 steps to reach thermal equilibrium.

%%%CALPHAD modeling
Thermodynamic modeling of the (LiF, NaF, KF, CrF${_2}$)-CrF${_3}$ system was carried out using the open-source software ESPEI \cite{bocklund2019espei} and PyCalphad \cite{otis2017pycalphad} with the newly implemented MQMQA \cite{palma2023thermodynamic}. All model parameters were simultaneously optimized through the Bayesian approach using MCMC \cite{bocklund2019espei}. The input data were primarily experimental phase equilibrium data including two or more co-existing phases. For stochiometric compounds, their thermochemical data from DFT-based calculations were also used as input. For the liquid phase, its mixing enthalpy from AIMD calculations was used as input for refining model parameters. In the present work, each model parameter employed two Markov chains with a standard derivation of 0.1 when initializing its Gaussian distribution. During the modeling process, the chain values can be tracked and the MCMC processes were performed until the model parameters converged.

\subsection{Thermodynamic properties in (LiF, NaF, KF, CrF$_2$)-CrF$_3$ by first-principles calculations} \label{moltensalts:ssec:FLiNaKCrsolids}

Table \ref{ms:tab:CrEOS} shows the predicted equilibrium properties of V$_0$, B$_0$, and B$^\prime$ by the EOS E-V fitting at 0 K in comparison with experimental bulk moduli \cite{yagi1978experimental, haussuhl1960thermo, boehler1980thermal, rao1974anderson, bensch1972third, jorgensen2004compression}. The B$_0$ results from GGA show a good agreement with experimental measurements. For the LiF compound, GGA predicts B$_0$ = 67.6 GPa and B$^\prime$ = 4.17, aligning well with the measured values of 65.4 GPa and 4.98 by Boehler et al. \cite{boehler1980thermal}. In comparison with the three measured B$_0$ values of 66.5 GPa by Yagi \cite{yagi1978experimental}, 76.9 GPa by Haussühl \cite{haussuhl1960thermo}, and 65.4 GPa by Boehler et al. \cite{boehler1980thermal}, the B$_0$ result by GGA shows a mean absolute error (MAE) of 4.2 GPa, while LDA predicts B$_0$ = 86.5 GPa with a higher MAE of 16.9 GPa. Considering the NaF compound, GGA predicts B$_0$ = 45.0 GPa, matching with the measured 45.9 GPa by Yagi \cite{yagi1978experimental} but lower than the measured values of 53.8 GPa by Haussühl \cite{haussuhl1960thermo}, 52.3 GPa by Rao \cite{rao1974anderson}, and 48.2 GPa by Bensch et al. \cite{bensch1972third} with the MAE around 5 GPa. On the other hand, LDA predicts B$_0$ = 61.4 GPa, which is higher than the experimental B$_0$ \cite{yagi1978experimental, haussuhl1960thermo, rao1974anderson, bensch1972third} with MAE = 11.4 GPa. Additionally, for the B$^\prime$ values of NaF, LDA predicts B$^\prime$ = 4.74, which is higher than the predicted 4.60 by GGA and closer to 5.89 reported by Bensch et al. \cite{bensch1972third}. Regarding the KF compound, the measured B$_0$ values (37.0 GPa by Yagi \cite{yagi1978experimental} and 35.5 GPa by Haussühl \cite{haussuhl1960thermo}) fall between the LDA result of 43.5 GPa and the GGA result of 28.9 GPa. The MAE value of LDA with respect to the measured B$_0$ values is 7.25 GPa, slightly lower than MAE = 7.35 GPa by GGA. As for the CrF$_3$ compound, GGA+U reports B$_0$ = 29.3 GPa, showing a 0.3\% difference compared to 29.2 GPa measured by Jørgensen et al. \cite{jorgensen2004compression}. LDA+U predicts B$_0$ = 46.2 GPa, which is 58\% higher than the measured 29.2 GPa \cite{jorgensen2004compression}. The present results indicate that LDA predicts smaller V$_0$ and higher B$_0$ values than those from GGA. It is consistent with the previous observations that LDA tends to underestimate lattice constants and overestimate cohesive energy with respect to GGA \cite{haas2009calculation, he2014accuracy}.

\begin{table}[H]
    \centering
    \caption{Predicted equilibrium properties of volume V$_0$, bulk modulus B$_0$, and the first derivative of bulk modulus with respect to pressure B$^\prime$ for each compound based on the present EOS fitting at 0 K in the FLiNaK-CrF$_3$-CrF$_2$ system. Experimental data (marked with *) \cite{yagi1978experimental, haussuhl1960thermo, boehler1980thermal, rao1974anderson, bensch1972third, jorgensen2004compression} are also listed for comparison}
    \begin{tabular}{>{\raggedright\arraybackslash}m{2.5cm}>{\raggedright\arraybackslash}m{4cm}>{\raggedright\arraybackslash}m{3cm}>{\raggedright\arraybackslash}m{2cm}>{\raggedright\arraybackslash}m{2cm}}
    \hline
    \textbf{Phases}&\textbf{Method}&\textbf{V$_0$} (\r{A}$^3$/atom)&\textbf{B$_0$}&\textbf{B$^\prime$}\\
    \hline
    LiF&LDA&7.488&86.5&4.25\\
    &GGA&8.419&67.6&4.17\\
    &Yagi$^*$\cite{yagi1978experimental}&&66.5&\\
    &Haussühl$^*$\cite{haussuhl1960thermo}&&76.9&\\
    &Boehler et al.$^*$ \cite{boehler1980thermal}&&65.4&4.98\\
    NaF&LDA&11.457&61.4&4.74\\
	&GGA&13.020&45.0&4.60\\
    &Yagi$^*$\cite{yagi1978experimental}&&45.9&\\
    &Haussühl$^*$\cite{haussuhl1960thermo}&&53.8&\\
    &Ramji Rao$^*$\cite{rao1974anderson}&&52.3&\\	
    &Bensch et al.$^*$ \cite{boehler1980thermal}&&48.2&5.89\\
    KF&LDA&17.363&43.5&5.06\\
	&GGA&20.050&28.9&4.84\\
	&Yagi$^*$\cite{yagi1978experimental}&&37.0&\\
	&Haussühl$^*$\cite{haussuhl1960thermo}&&35.5&\\	
    CrF$_3$&LDA+U&11.152&46.2&8.46\\
	&GGA+U&12.902&29.3&7.29\\
	&Jørgensen et al.$^*$\cite{jorgensen2004compression}&&29.2&10.3\\
    CrF$_2$&LDA+U&12.443&71.1&2.74\\
    Li$_3$CrF$_6$&LDA+U&9.543&56.1&5.85\\
    Na$_3$CrF$_6$&LDA+U&11.577&57.5&5.46\\
    Na$_5$Cr$_3$F$_{14}$&LDA+U&11.757&52.2&5.15\\
    NaCrF$_4$&LDA+U&11.933&53.0&4.35\\
    K$_3$CrF$_6$&LDA+U&15.705&51.5&5.67\\
    K$_2$CrF$_5$&LDA+U&13.858&45.9&5.65\\
    KCrF$_4$&LDA+U&13.920&38.1&4.88\\
    K$_2$Cr$_5$F$_{17}$&LDA+U&13.314&49.3&6.91\\
    Cr$_2$F$_5$&LDA+U&12.225&46.6&7.91\\
    \hline
    \end{tabular}
    \label{ms:tab:CrEOS}
\end{table}

Figure \ref{ms:fig:FLiNaKCrphonon} compares the phonon DOS of LiF, NaF, and KF obtained using LDA and GGA in comparison with direct measurements by neutron scattering \cite{dolling1968lattice, buhrer1970lattice} or fittings in terms of measurements \cite{karo1969lattice}. Overall, the peak positions of the experimental phonon DOS are well reproduced by both LDA and GGA. However, in the low-frequency region (e.g., < 5 THz for LiF and NaF, and < 3 THz for KF), the phonon DOS predicted by LDA show a better match in both the shape and the peak position with respect to experimental data \cite{dolling1968lattice, buhrer1970lattice, karo1969lattice} than those from GGA. This observation suggests that LDA predicts more reliable thermodynamic properties of LiF, NaF, and KF when employing the phonon-based QHA since these properties are mainly regulated by phonon DOS at low-frequency regions \cite{shang2019achieving}. 

\begin{figure}[H]
    \centering
    \includegraphics[width=0.45\linewidth]{moltensalts/Moltensalts-FLiNaKCr-PhononDOS.jpg}
    \caption{Predicted phonon density of states (DOS) of (a) LiF, (b) NaF, and (c) KF from DFT-based phonon calculations in comparison with phonon DOS from experiments \cite{dolling1968lattice, buhrer1970lattice, karo1969lattice}. Blue lines show results from the LDA approach, red dot-dashed lines show results from the GGA-PBE approach, and the green dotted lines are results from experiments \cite{dolling1968lattice, buhrer1970lattice, karo1969lattice}. }
    \label{ms:fig:FLiNaKCrphonon}
\end{figure}

Figure \ref{ms:fig:FLiNaKCr-Benchmark} illustrates a comparison of the predicted heat capacity (C$_p$), entropy (S), and enthalpy (H$-$H$_{300}$) of LiF, NaF, and KF in terms of the phonon-based QHA, where the DFT calculations were conducted using both the LDA and GGA, and the predicted results are compared to the data from the SSUB database \cite{sgteurl}. In general, thermodynamic properties predicted by LDA align well with the results from SSUB. The most substantial difference between LDA and SSUB is the C$_p$ values of KF, where a 6\% disparity is noted. Furthermore, these comparisons reveal that the LDA results exhibit a better agreement with SSUB than the GGA results, particularly for LiF. For example, at 1100 K, the C$_p$ values predicted by LDA demonstrate only a 2\% difference compared to those from the SSUB \cite{sgteurl}, whereas an 18\% difference is observed when using GGA. Figure \ref{ms:fig:FLiNaKCr-Benchmark} indicates that the QHA in terms of LDA yields more reliable predictions of thermodynamic properties in LiF-NaF-KF-based system, agreeing with the observations in phonon DOS in Figure \ref{ms:fig:FLiNaKCrphonon}. Subsequent DFT calculations were hence performed using the LDA approach. 

\begin{figure}[ht]
    \centering
    \includegraphics[width=1\linewidth]{moltensalts/Moltensalts-FLiNaKCr-Benchmark.jpg}
    \caption{Predicted heat capacity C$_p$, entropy S, and enthalpy with reference at 300 K (H$-$H$_{300}$) of (a) LiF, (b) NaF, and (c) KF by DFT-based QHA via phonon calculations. Results by LDA is marked as solid blue lines, calculations by PBE are marked as dashed dot red lines, and the SSUB results \cite{sgteurl} are in dashed yellow lines. The SSUB results are implemented in the present modeling work.}
    \label{ms:fig:FLiNaKCr-Benchmark}
\end{figure}

Figure \ref{ms:fig:FLiNaKCr-Benchmark-Cr} shows the present DFT predictions and the values in SSUB \cite{sgteurl} regarding C$_p$, S, and H$-$H$_{300}$ for CrF$_3$ and CrF$_2$. In general, the present predictions tend to be lower than those obtained from SSUB \cite{sgteurl}. For CrF$_3$, the DFT predicted C$_p$ = 19.77 J/mol-atom-K closely aligns with the SSUB value of 19.73 J/mol-atom-K at 300 K. As the temperature increases to 1300 K, the difference increases to 9\%. Similarly, for CrF$_2$, the C$_p$ = 21.22 J/mol-atom-K at 300 K by DFT is in good agreement with the value of 21.66 from SSUB. At a higher temperature of 1140 K, the difference expands to 12\%. Regarding entropy, DFT predicts lower S values for both CrF$_3$ and CrF$_2$ than those in SSUB across the temperature range shown in Figure \ref{ms:fig:FLiNaKCr-Benchmark-Cr}. At high temperatures (e.g., 1600 K for CrF$_3$ and 1140 K for CrF$_2$), it shows a 10\% difference in CrF$_3$ and 14\% in CrF$_2$. It is found that a good agreement is observed regarding H$-$H$_{300}$ values of CrF$_3$ and CrF$_2$ between the DFT calculations and the SSUB at lower temperatures (< 1000 K for CrF$_3$ and < 600 K for CrF$_2$). H$-$H$_{300}$ from DFT becomes slightly lower than that in SSUB \cite{sgteurl} with increasing temperature, with the differences, for example, around 5\% in CrF$_3$ at 1650 K and 9\% in CrF$_2$ at 1140 K.

\begin{figure}[ht]
    \centering
    \includegraphics[width=0.75\linewidth]{moltensalts/Moltensalts-FLiNaKCr-Benchmark-Cr.jpg}
    \caption{Predicted heat capacity C$_p$, entropy S, and enthalpy with reference at 300 K (H$-$H$_{300}$) of (a) CrF$_3$ and (b) CrF$_2$ by DFT-based QHA via phonon calculations marked in solid lines in comparison with SSUB \cite{sgteurl} in dashed lines.}
    \label{ms:fig:FLiNaKCr-Benchmark-Cr}
\end{figure}

Table \ref{ms:tab:FLiNaK-Cr-Hf} shows the present DFT values of formation enthalpy ($\Delta_fH_m$) of these ternary compounds using the LDA+U approach, together with the reactions to form these compounds. Table \ref{ms:tab:FLiNaK-Cr-Hf} shows that the $\Delta_fH_m$ values are negative for all ternary compounds with reference to their corresponding binary compounds. Yin et al. \cite{yin2018thermodynamic, yin2014thermodynamic} conducted DFT calculations for Li$_3$CrF$_6$ and Na$_3$CrF$_6$. Predicted $\Delta_fH_m$ values from the Materials Project \cite{jain2013commentary} and the Open Quantum Materials Database (OQMD) \cite{kirklin2015open} are also listed in Table \ref{ms:tab:FLiNaK-Cr-Hf} and displayed in Figure \ref{ms:fig:FLiNaKCr-Hf}. The present DFT calculations by LDA+U predict higher $\Delta_fH_m$ values of Li$_3$CrF$_6$ and Na$_3$CrF$_6$ than those by Yin et al. \cite{yin2018thermodynamic, yin2014thermodynamic} using GGA. The present DFT calculations align better with results from the Materials Project \cite{jain2013commentary} and OQMD \cite{kirklin2015open} than calculations from Yin et al. \cite{yin2018thermodynamic, yin2014thermodynamic}. Figure \ref{ms:fig:FLiNaKCr-Hf} displays the convex hulls for the LiF-CrF$_3$, NaF-CrF$_3$, and KF-CrF$_3$ systems based on the $\Delta_fH_m$ values listed in Table \ref{ms:tab:FLiNaK-Cr-Hf}. These convex hulls serve as indicators regarding the stability of ternary compounds in these systems. Li$_3$CrF$_6$ is on the convex hull, suggesting that it is stable in the LiF-CrF$_3$ system. In the NaF-CrF$_3$ system, Na$_3$CrF$_6$ is located on the hull, indicating its stability at 0 K. Na$_5$Cr$_3$F$_{14}$ shows an elevation of 1.09 kJ/mol-atom above the hull, and NaCrF$_4$ shows 0.42 kJ/mol-atom above the hull. In the KF-CrF$_3$ system, K$_2$CrF$_5$ is on the convex hull at 0 K. K$_2$Cr$_5$F$_{17}$ is the farthest away from the convex hull (1.31 kJ/mol-atom above it), while K$_3$CrF$_6$ and KCrF$_4$ are 0.62 kJ/mol-atom and 0.61 kJ/mol-atom above the hull, respectively. These calculations at 0 K suggest that Na$_5$Cr$_3$F$_{14}$, NaCrF$_4$, K$_2$Cr$_5$F$_{17}$, K$_3$CrF$_6$, and KCrF$_4$ are not stable at 0 K, while De Kozak \cite{DeKozak1969} reported the existence of the above ternary compounds at high temperature. It suggests that phonon-based QHA is necessary to investigate the thermodynamic properties of ternary compounds at high temperatures.

\begin{table}[H]
    \centering
    \caption{DFT-based results of formation enthalpy ($\Delta_fH_m$) at 0 K of ternary compounds in the AF-CrF$_3$ (A=Li, Na, and K) systems with the reference states as shown in the reactions. DFT results from Yin et al. \cite{yin2018thermodynamic, yin2014thermodynamic}, the Materials Project \cite{jain2013commentary}, and the Open Quantum Materials Database (OQMD) \cite{kirklin2015open} are listed for comparison.}
    \begin{tabular}{>{\raggedright\arraybackslash}m{2.5cm}>{\raggedright\arraybackslash}m{5cm}>{\raggedright\arraybackslash}m{3cm}>{\raggedright\arraybackslash}m{5cm}}
    \hline
    \textbf{Compound}&\textbf{Reaction}&\textbf{$\Delta_fH_m$ (J/mol-atom)}&\textbf{Source}\\
    \hline
    Li$_3$CrF$_6$&Li$_3$CrF$_6$$=$3LiF$+$CrF$_3$&$-1815$&This work\\
    &&$-4144$&Yin et al. \cite{yin2014thermodynamic}\\
    &&$-2238$&Materials Project \cite{jain2013commentary}\\
    &&$-2161$&OQMD \cite{kirklin2015open}\\
    Na$_3$CrF$_6$&Na$_3$CrF$_6$$=$3NaF$+$CrF$_3$&$-7849$&This work\\
	&&$-8592$&Yin et al. \cite{yin2018thermodynamic}\\
    &&$-6657$&Materials Project \cite{jain2013commentary}\\
    &&$-8683$&OQMD \cite{kirklin2015open}\\
    Na$_3$CrF$_6$&Na$_3$CrF$_6$$=$3NaF$+$CrF$_3$&$-7849$&This work\\
	&&$-8592$&Yin et al. \cite{yin2018thermodynamic}\\
    &&$-6657$&Materials Project  \cite{jain2013commentary}\\
    &&$-8683$&OQMD \cite{kirklin2015open}\\
    Na$_5$Cr$_3$F$_{14}$&Na$_5$Cr$_3$F$_{14}$$=$5NaF$+$3CrF$_3$&$-5447$&This work\\
	&&$-5148$&Materials Project \cite{jain2013commentary}\\
	&&$-6859$&OQMD \cite{kirklin2015open}\\
    NaCrF$_4$&NaCrF$_4$$=$NaF$+$CrF$_3$&$-4816$&This work\\
    &&$-5071$&Materials Project \cite{jain2013commentary}\\
	&&$-6657$&OQMD \cite{kirklin2015open}\\
    K$_3$CrF$_6$&K$_3$CrF$_6$$=$3KF$+$CrF$_3$&$-8621$&This work\\
    &&$-11038$&Materials Project \cite{jain2013commentary}\\
	&&$-13855$&OQMD \cite{kirklin2015open}\\
    K$_2$CrF$_5$&K$_2$CrF$_5$$=$2KF$+$CrF$_3$&$-12322$&This work\\
	&&$-12446$&Materials Project \cite{jain2013commentary}\\
    KCrF$_4$&KCrF$_4$$=$KF$+$CrF$_3$&$-8631$&This work\\
    &&$-9970$&Materials Project \cite{jain2013commentary}\\
    K$_2$Cr$_5$F$_{17}$&K$_2$Cr$_5$F$_{17}$$=$2KF$+$5CrF$_3$&$-3974$&This work\\
    \hline
    \end{tabular}
    \label{ms:tab:FLiNaK-Cr-Hf}
\end{table}

\begin{figure}[H]
    \centering
    \includegraphics[width=0.5\linewidth]{moltensalts/Moltensalts-FLiNaKCr-DFTHf.jpg}
    \caption{Convex hull of ternary compounds in LiF-CrF$_3$ (green), NaF-CrF$_3$ (red), and KF-CrF$_3$ (blue) at 0 K from DFT-based calculations from the present work. Circles ($\circ$) are the formation enthalpy of compounds from the present work; cross markers ($\times$) are the formation enthalpy of compounds from Yin et al. \cite{yin2018thermodynamic, yin2014thermodynamic}; plus markers ($+$) result from The Materials Project \cite{jain2013commentary}; and diamond markers ($\diamond$) results from OQMD \cite{kirklin2015open}. }
    \label{ms:fig:FLiNaKCr-Hf}
\end{figure}

Figure \ref{ms:fig:FLiNaKCr-DFTcompounds} shows the predicted heat capacities C$_p$ of ternary compounds in the AF-CrF$_3$ (A= Li, Na, and K) systems from the phonon-based QHA, in comparison with the results estimated by the Neumann-Kopp rule \cite{leitner2010application}, which were used in CALPHAD modeling by Dumaire et al. \cite{dumaire2021thermodynamic}. It shows that the Neumann-Kopp rule matches the results of phonon-based QHA in the LiF-CrF$_3$ system. However, in the NaF-CrF$_3$ and KF-CrF$_3$ systems, the Neumann-Kopp rule estimates higher values of C$_p$ with respect to the values predicted by phonon-based QHA. The differences between the LiF-CrF$_3$ system and the NaF/KF-CrF$_3$ may be attributed to variations in melting temperatures between ternary and corresponding binary compounds. For example, in the LiF-CrF$_3$ system, the melting temperature of Li$_3$CrF$_6$ is reported at 1129 K \cite{DeKozak1969}, which is close to that of LiF at 1121 K and below that of CrF$_3$ at 1698 K \cite{sgteurl}. Considering the temperature below the melting point of Li$_3$CrF$_6$ (T < 1129 K), there are reliable resources of C$_p$ data from two endmembers LiF and CrF$_3$, thus the Neumann-Kopp rule C$_p$ estimation of Li$_3$CrF$_6$ is acceptable. However, in the NaF-CrF$_3$ system, Na$_3$CrF$_6$ melts at 1413 K \cite{DeKozak1969}, while NaF melts at 1269 K \cite{sgteurl}, indicating that there is an approximately 150 K temperature range without reliable C$_p$ data for NaF. In contrast, the phonon-based calculations are direct predictions of ternary compounds and they provide more accurate descriptions of thermodynamic properties for compounds than those by the Neumann-Kopp rule used by Dumaire et al. \cite{dumaire2021thermodynamic}, especially at high temperatures. Therefore, the phonon-based QHA results were used in the present CALPHAD modeling to improve the accuracy in describing ternary compounds. Note that the C$_p$ values for compounds in the (LiF, NaF, KF, and CrF$_2$)-CrF$_3$ systems can be predicted using the Supplementary XML file \cite{gong2024revisiting}.

\begin{figure}[H]
    \centering
    \includegraphics[width=0.45\linewidth]{moltensalts/Moltensalts-FLiNaKCr-DFTcompounds.jpg}
    \caption{Predicted heat capacities of ternary compounds in the (a) LiF–CrF$_3$, (b) NaF–CrF$_3$, and (c) KF-CrF$_3$ systems by DFT-based QHA via phonon calculations (solid lines) compared with the Dumaire et al. \cite{dumaire2021thermodynamic}'s work (dashed lines). The QHA results are implemented in the present modeling work for ternary compounds.}
    \label{ms:fig:FLiNaKCr-DFTcompounds}
\end{figure}



\subsection{Thermodynamic modeling of the (LiF, NaF, KF, CrF$_2$)-CrF$_3$ system} \label{moltensalts:ssec:FLiNaKCrmodeling}

Figure \ref{ms:fig:FLiNaKCr-Phasediagram} shows the phase diagrams of the (LiF, NaF, KF, and CrF$_2$)-CrF$_3$ systems calculated from present models and compared with experimental data by De Kozak \cite{de1975systeme, DeKozak1969} and Sturm \cite{sturm1962phase}. The present CALPHAD modeling shows a good agreement regarding phase boundaries with experimental data. The presently modeled parameters for liquid and the complete thermodynamic database can be found in the Supplementary XML file \cite{gong2024revisiting}. Details of the invariant reactions and the congruent melting temperature calculated from the present modeling work compared to experiments \cite{de1975systeme, DeKozak1969, sturm1962phase} and ML predictions using Hong et al.’s model \cite{hong2022melting} are listed in Table \ref{ms:tab:FLiNaK-Cr-inv} with discussion below. 

\begin{figure}[H]
    \centering
    \includegraphics[width=1\linewidth]{moltensalts/Moltensalts-FLiNaKCr-Phasediagram.jpg}
    \caption{Predicted phase diagrams of the (a) LiF–CrF$_3$, (b) NaF–CrF$_3$, (c) KF-CrF$_3$, and (d) CrF$_2$–CrF$_3$ systems by the present CALPHAD modeling in comparison with experimental data \cite{de1975systeme, DeKozak1969, sturm1962phase}.}
    \label{ms:fig:FLiNaKCr-Phasediagram}
\end{figure}
\newpage
\begingroup
\renewcommand\arraystretch{0.8}
\begin{longtable}[H]{>{\raggedright\arraybackslash}m{2cm}>{\raggedright\arraybackslash}m{6cm}>{\raggedright\arraybackslash}m{1.5cm}>{\raggedright\arraybackslash}m{3cm}>{\raggedright\arraybackslash}m{3.5cm}}
    \caption{Predicted invariant equilibria in the AF-CrF$_3$ (A=Li, Na, K) systems by the present CALPHAD modeling, compared with experimental data from De Kozak \cite{DeKozak1969} and Sturm \cite{sturm1962phase} (marked with *), and other modeling works \cite{yin2014thermodynamic, yin2015thermodynamic, yin2018thermodynamic, dumaire2021thermodynamic}.}\\
    \hline
    \textbf{Reaction}&&\textbf{$x$(CrF$_3$)}&\textbf{Temperature (K)}&\textbf{Source}\\
    \hline
    Eutectic&Liquid$\leftrightarrow$LiF+Li$_3$CrF$_6$&0.135&1002&This work\\
    &&0.150&1003&De Kozak \cite{DeKozak1969}$^*$\\
    &&0.136&1008&Dumaire et al. \cite{dumaire2021thermodynamic}\\
    &&0.148&1003&Yin et al. \cite{yin2014thermodynamic}\\
    Congruent melting&Liquid$\leftrightarrow$Li$_3$CrF$_6$&0.25&1131&This work\\
    &&0.25&1129&De Kozak \cite{DeKozak1969}$^*$\\
    &&0.25&1114&Hong et al. \cite{hong2022melting}\\
    &&0.25&1111&Dumaire et al. \cite{dumaire2021thermodynamic}\\
    &&0.25&1125&Yin et al. \cite{yin2014thermodynamic}\\
    Eutectic&Liquid$\leftrightarrow$CrF$_3$+Li$_3$CrF$_6$&0.382&1056&This work\\
    &&0.350&1059&De Kozak \cite{DeKozak1969}$^*$\\
    &&0.363&1062&Dumaire et al. \cite{dumaire2021thermodynamic}\\
    &&0.354&1058&Yin et al. \cite{yin2014thermodynamic}\\
    Eutectic&Liquid$\leftrightarrow$NaF+Na$_3$CrF$_6$&0.098&1155&This work\\
    &&0.123&1166&De Kozak \cite{DeKozak1969}$^*$\\
    &&0.106&1175&Dumaire et al. \cite{dumaire2021thermodynamic}\\
    &&0.114&1162&Yin et al. \cite{yin2018thermodynamic}\\
    Congruent melting&Liquid$\leftrightarrow$Na$_3$CrF$_6$&0.25&1410&This work\\
    &&0.25&1413&De Kozak \cite{DeKozak1969}$^*$\\
    &&0.25&1404&Hong et al. \cite{hong2022melting}\\
    &&0.25&1385&Dumaire et al. \cite{dumaire2021thermodynamic}\\
    &&0.25&1416&Yin et al. \cite{yin2018thermodynamic}\\
    Eutectic&Liquid$\leftrightarrow$Na$_5$Cr$_3$F$_{14}$+Na$_3$CrF$_6$&0.368&1155&This work\\
    &&0.371&1145&Dumaire et al. \cite{dumaire2021thermodynamic}\\
    &&0.367&1142&Yin et al. \cite{yin2018thermodynamic}\\
    Eutectic&Liquid$\leftrightarrow$Na$_5$Cr$_3$F$_{14}$+NaCrF$_4$&0.377&1154&This work\\
    &&0.381&1144&Dumaire et al. \cite{dumaire2021thermodynamic}\\
    &&0.383&1141&Yin et al. \cite{yin2018thermodynamic}\\
    Peritectic&Liquid+CrF$_3$$\leftrightarrow$NaCrF$_4$&0.5&1235&This work\\
    &&0.5&1234&De Kozak \cite{DeKozak1969}$^*$\\
    &&0.5&1232&Dumaire et al. \cite{dumaire2021thermodynamic}\\
    &&0.5&1239&Yin et al. \cite{yin2018thermodynamic}\\
    Eutectic&Liquid$\leftrightarrow$KF+K$_3$CrF$_6$&0.050&1096&This work\\
    &&0.048&1115&De Kozak \cite{DeKozak1969}$^*$\\
    &&0.041&1108&Dumaire et al. \cite{dumaire2021thermodynamic}\\
    &&0.045&1113&Yin et al. \cite{yin2018thermodynamic}\\ 
    Congruent melting&Liquid$\leftrightarrow$K$_3$CrF$_6$&0.25&1551&This work\\
    &&0.25&1553&De Kozak \cite{DeKozak1969}$^*$\\
    &&0.25&1520&Hong et al. \cite{hong2022melting}\\
    &&0.25&1553&Dumaire et al. \cite{dumaire2021thermodynamic}\\
    &&0.25&1548&Yin et al. \cite{yin2015thermodynamic}\\ 
    Peritectic&Liquid+K$_3$CrF$_6$$\leftrightarrow$K$_2$CrF$_5$&0.333&1141&This work\\
    &&0.333&1133&De Kozak \cite{DeKozak1969}$^*$\\
    &&0.333&1130&Dumaire et al. \cite{dumaire2021thermodynamic}\\
    &&0.333&1135&Yin et al. \cite{yin2015thermodynamic}\\  
    Eutectic&Liquid$\leftrightarrow$KCrF$_4$+K$_2$CrF$_5$&0.420&1120&This work\\
    &&0.45&1112&De Kozak \cite{DeKozak1969}$^*$\\
    &&0.432&1112&Dumaire et al. \cite{dumaire2021thermodynamic}\\
    &&0.426&1107&Yin et al. \cite{yin2015thermodynamic}\\  
    Peritectic&Liquid+K$_2$Cr$_5$F$_{17}$$\leftrightarrow$KCrF$_4$&0.5&1192&This work\\
    &&0.5&1200&De Kozak \cite{DeKozak1969}$^*$\\
    &&0.5&1191&Dumaire et al.  \cite{dumaire2021thermodynamic}\\
    &&0.5&1195&Yin et al. \cite{yin2015thermodynamic}\\ 
    Peritectic&Liquid+CrF$_3$$\leftrightarrow$K$_2$Cr$_5$F$_{17}$&0.714&1390&This work\\
    &&0.714&1390&De Kozak \cite{DeKozak1969}$^*$\\
    &&0.714&1390&Dumaire et al. \cite{dumaire2021thermodynamic}\\
    &&0.714&1388&Yin et al. \cite{yin2015thermodynamic}\\ 
    Eutectic&Liquid$\leftrightarrow$CrF$_2$+Cr$_2$F$_5$&0.134&1086&This work\\
    &&0.14&1103&Sturm \cite{sturm1962phase}$^*$\\
    &&0.115&1104&Dumaire et al. \cite{dumaire2021thermodynamic}\\
    Peritectic&Liquid+CrF$_3$$\leftrightarrow$Cr$_2$F$_5$&0.293&1273&This work\\
    &&0.29&1272&Sturm \cite{sturm1962phase}$^*$\\
    &&0.28&1271&Dumaire et al. \cite{dumaire2021thermodynamic}\\
    \hline
    \label{ms:tab:FLiNaK-Cr-inv}
\end{longtable}
\endgroup

In the LiF-CrF$_3$ system, the present prediction of the eutectic reaction Liquid $\leftrightarrow$ CrF$_3$ + Li$_3$CrF$_6$ at T = 1056 K matches closely with experimental values of T = 1059 K \cite{DeKozak1969}. There is a minor difference of 1 K observed for Liquid $\leftrightarrow$ LiF + Li$_3$CrF$_6$, with the present modeling predicts an eutectic temperature of 1002 K compared to 1003 K reported by De Kozak \cite{DeKozak1969}. This presents an improvement over the 1008 K predicted by Dumaire et al. \cite{dumaire2021thermodynamic}’s modeling. The present prediction of the congruent melting temperature of 1131 K is 2 K higher than the experimental value of 1129 K reported by De Kozak \cite{DeKozak1969}, significantly improved from the prediction of 1111 K by Dumaire et al. \cite{dumaire2021thermodynamic}. 

In the NaF-CrF$_3$ system, the present modeling gives good predictions of eutectic and peritectic temperatures compared with experiments \cite{DeKozak1969}. The difference of 1 K is observed for Liquid + CrF$_3$ $\leftrightarrow$ NaCrF$_4$ (1235 K from the present work and 1234 K by De Kozak \cite{DeKozak1969}). The eutectic composition for the reaction Liquid $\leftrightarrow$ NaF + Na$_3$CrF$_6$ is $x$(CrF$_3$) = 0.098, which is around 0.025 away from the experimental value $x$(CrF$_3$) = 0.123 \cite{DeKozak1969}. The eutectic temperature for this reaction by the present modeling is 1155 K, in comparison to the experimental value of 1166 K by De Kozak \cite{DeKozak1969}. The present congruent melting temperature of Na$_3$CrF$_6$ is predicted at 1410 K, improving from the predicted 1385 K by Dumaire et al. \cite{dumaire2021thermodynamic} but slightly lower than the experimental value of 1413 K \cite{DeKozak1969}. For the two eutectic reactions without experimental data, i.e., Liquid $\leftrightarrow$ Na$_5$Cr$_3$F$_{14}$ + Na$_3$CrF$_6$ and Liquid $\leftrightarrow$ Na$_5$Cr$_3$F$_{14}$ + NaCrF$_4$, the present modeling provides similar predictions ($x$(CrF$_3$) = 0.368, T = 1155 K; $x$(CrF$_3$) = 0.377, T = 1154 K respectively) compared to those modeled by Dumaire et al. \cite{dumaire2021thermodynamic} ($x$(CrF$_3$) = 0.371, T = 1145 K; $x$(CrF$_3$) = 0.381, T = 1144 K, respectively). 

In the KF-CrF$_3$ system, the present modeling work predicts the congruent melting of K$_3$CrF$_6$ at 1551 K, which is in good agreement with the experimental value of 1553 K \cite{DeKozak1969}. For the eutectic reaction, Liquid $\leftrightarrow$ KCrF$_4$ + K$_2$CrF$_5$, the present modeling predicts $x$(CrF$_3$) = 0.42 and T = 1120 K, whereas the values are $x$(CrF$_3$) = 0.45 and T = 1112 K modeled by De Kozak \cite{DeKozak1969}. The present modeling predicts a eutectic temperature of Liquid $\leftrightarrow$ KF + K$_3$CrF$_6$ at 1096 K, slightly lower than the experimental value of 1115 K by De Kozak \cite{DeKozak1969}. For the three peritectic reactions in the KF-CrF$_3$ system, the present modeling work predicts 1141 K for Liquid + K$_3$CrF$_6$ $\leftrightarrow$ K$_2$CrF$_5$, 1192 K for Liquid + K$_2$Cr$_5$F$_{17}$ $\leftrightarrow$ KCrF$_4$, and 1390 K for Liquid + CrF$_3$ $\leftrightarrow$ K$_2$Cr$_5$F$_{17}$. These temperatures are comparable to 1133 K, 1200 K, and 1390 K, respectively, as measured by De Kozak \cite{DeKozak1969}, showing an MAE of 5.33 K.

In the CrF$_2$-CrF$_3$ system, the present modeling work improves the predicted invariant compositions. For the eutectic reaction Liquid$\leftrightarrow$CrF$_2$+Cr$_2$F$_5$, the present work predicts the eutectic point at $x$(CrF$_3$) = 0.134. This value is higher than $x$(CrF$_3$) = 0.115 by Dumaire et al. \cite{dumaire2021thermodynamic} but aligns more closely with the experimental value $x$(CrF$_3$) = 0.14 by Sturm \cite{sturm1962phase}. The eutectic temperature for this reaction is 1086 K predicted by the present modeling, demonstrating a slightly lower value of 17 K than the measured 1103 K by Sturm \cite{sturm1962phase}. For the peritectic reaction Liquid + CrF$_3$ $\leftrightarrow$ Cr$_2$F$_5$, the present work predicts $x$(CrF$_3$) = 0.293 compared to measured $x$(CrF$_3$) = 0.29 by Sturm \cite{sturm1962phase}. The present invariant temperature is predicted at 1273 K, which remains close to the experimental 1272 K by Sturm \cite{sturm1962phase}. In the present work, the single-phase region of the Cr$_2$F$_5$ solid solution phase ranges from $x$(CrF$_3$) = 0.38 to $x$(CrF$_3$) = 0.46. This range aligns well with the suggested values of $x$(CrF$_3$) from 0.40 to 0.45 by Sturm \cite{sturm1962phase}. Overall, by incorporating thermodynamic data of compounds by the present DFT calculations, the present modeling work yields improved predictions of phase diagrams.

The present CALPHAD modeling work implements the mixing enthalpy of liquid at 1700 K, which was obtained by the present AIMD simulations as described in Section \ref{moltensalts:ssec:FLiNaKCrmodel}. Note that the mixing enthalpy values from AIMD simulations can be found in the Supplementary JSON files \cite{gong2024revisiting}. Figure \ref{ms:fig:FLiNaKCr-Hmix} shows the mixing enthalpy of liquid at 1700 K by AIMD at different compositions compared to the present modeling results and results by Dumaire et al. \cite{dumaire2021thermodynamic}. It clearly shows that the mixing enthalpy of liquid (dot-dashed lines) by Dumaire et al.’s modeling \cite{dumaire2021thermodynamic} is much less negative compared to the present AIMD calculations. For example, in the LiF-CrF$_3$ system at $x$(CrF$_3$) = 0.2, AIMD predicts the mixing enthalpy of $-20.60$ kJ/mol-atom at 1700 K, while Dumaire et al’s modeling \cite{dumaire2021thermodynamic} shows $-12.65$ kJ/mol-atom, representing a 39\% higher value. Using the present AIMD data of liquid for modeling, the present modeling work (solid lines) shows a great improvement. The present modeling work improves the prediction to $-18.99$ kJ/mol-atom at $x$(CrF$_3$) = 0.2 for the LiF-CrF$_3$ system. The differences between the AIMD results and Dumaire et al.’s work \cite{dumaire2021thermodynamic} are more pronounced in the NaF-CrF$_3$ and KF-CrF$_3$ systems. In the NaF-CrF$_3$ system, at $x$(CrF$_3$) = 0.333 (a composition region near the lowest mixing enthalpy), Dumaire et al.’s modeling \cite{dumaire2021thermodynamic} shows the mixing enthalpy of $-25.45$ kJ/mol-atom, while our modeling predicts $-49.96$ kJ/mol-atom. At the same condition, AIMD gives the mixing enthalpy of $-47.37$ kJ/mol-atom, which is about 22 kJ/mol-atom lower than Dumaire et al.’s results \cite{dumaire2021thermodynamic}. In the KF-CrF$_3$ system at $x$(CrF$_3$) = 0.333, Dumaire et al. \cite{dumaire2021thermodynamic} predicts $-30.58$ kJ/mol-atom, significantly higher than $-50.20$ kJ/mol-atom by AIMD. The present modeling predicts $-53.70$ kJ/mol-atom, reducing the difference from the above-mentioned 39\% to the present 7\%. It highlights that the present AIMD simulations enhance the reliability of the present modeling in describing liquid compared to the previous work \cite{dumaire2021thermodynamic}. The mixing enthalpy (dotted lines) modeling by Yin et al. \cite{yin2018thermodynamic} in the NaF-CrF$_3$ and KF-CrF$_3$ systems agree well with the results of the present modeling. Near the low mixing enthalpy region $x$(CrF$_3$) = 0.333 in the NaF-CrF$_3$ system, Yin et al. \cite{yin2018thermodynamic} suggest the mixing enthalpy of $-40.54$ kJ/atom, which is 19\% higher than $-49.96$ kJ/atom from the present work. Note that Yin et al. \cite{yin2018thermodynamic} used an empirical model proposed by Robelin and Chartrand \cite{robelin2011thermodynamic} to estimate the mixing enthalpy in liquid. Overall, the present modeling work has improved the predictions of liquid than the modeling works by Dumaire et al. \cite{dumaire2021thermodynamic} and Yin et al. \cite{yin2018thermodynamic}. 

\begin{figure}[H]
    \centering
    \includegraphics[width=0.45\linewidth]{moltensalts/Moltensalts-FLiNaKCr-Hmix.jpg}
    \caption{Predicted mixing enthalpy of liquid at 1700 K in (a) LiF-CrF$_3$, (b) NaF-CrF$_3$, and (c) KF-CrF$_3$ by the present CALPHAD modeling work (black solid lines), compared with the present AIMD results (circles) and modeling results by Dumaire et al. (blue dashed lines) \cite{dumaire2021thermodynamic} and Yin et al. (brown dotted lines) \cite{yin2018thermodynamic}.}
    \label{ms:fig:FLiNaKCr-Hmix}
\end{figure}

\begin{figure}[H]
    \centering
    \includegraphics[width=0.45\linewidth]{moltensalts/Moltensalts-FLiNaKCr-QuadFrac.jpg}
    \caption{Precited quadruplet fractions in (a) LiF-CrF$_3$ (green lines), (b) NaF-CrF$_3$ (red lines), and (c) KF-CrF$_3$ (blue lines) liquid at 1700 K according to the present CALPHAD modeling. }
    \label{ms:fig:FLiNaKCr-QuadFrac}
\end{figure}

In Yin et al.’s modeling work \cite{yin2018thermodynamic}, the associate model was used to describe the liquid phase. This model was applied to describe the short-range ordering (SRO) by assuming ‘associates’, such as K$_3$CrF$_6$ and Na$_3$CrF$_6$ associates. This kind of assumption may cause issues when extrapolation into higher-order systems \cite{pelton2018phase}. In the present work, the MQMQA was employed to describe liquid and provide information on the first and the second nearest neighbors in complex liquid. As an example, Figure \ref{ms:fig:FLiNaKCr-QuadFrac} shows the predicted fraction of each quadruplet in the liquid phase. The composition where the peak fraction of the ACr/FF (A=Li, Na, and K) quadruplets appears, indicates the SRO. In the LiF-CrF$_3$ system, the peak fraction of LiCr/FF with strong SRO is around $x$(CrF$_3$) = 0.25, which is consistent with the lowest mixing enthalpy around the $x$(CrF$_3$) = 0.25 as shown in Figure \ref{ms:fig:FLiNaKCr-Hmix}. In addition, Figure \ref{ms:fig:FLiNaKCr-QuadFrac} presents the quadruplet fractions and the neighboring environments of ions in liquid, which are difficult to obtain from the associate model due to its focus on associate clusters.

\section{Thermodynamic modeling of LiF-LnF${_3}$ systems with uncertainty propagation to vapor-liquid equilibrium property} \label{moltensalts:sec:LiFLnF3}

\subsection{Literature review} \label{moltensalts:ssec:LiFLnF3lit}


\subsection{Modeling details} \label{moltensalts:ssec:LiFLnF3model}


\subsection{Results and discussion} \label{moltensalts:ssec:LiFLnF3result}


\section{Bayesian model selection in thermodynamic modeling of LiCl-KCl-LaCl${_3}$} \label{moltensalts:sec:LaCl3}

\subsection{Literature review} \label{moltensalts:ssec:LaCl3lit}


\subsection{Modeling details} \label{moltensalts:ssec:LaCl3model}


    \subsection{Results and discussion} \label{moltensalts:ssec:LaCl3result}


\section{Summary} \label{moltensalts:sec:Summary}
The present work revisits the thermodynamic properties of compounds and liquids in the (LiF, NaF, KF, CrF$_2$)-CrF$_3$ systems by utilizing CALPHAD modeling with inputs from the DFT-based first-principles, phonon, and AIMD calculations. Thermodynamic properties, including enthalpy, entropy, and heat capacity of the binary (endmember) compounds LiF, NaF, KF, CrF$_3$, and CrF$_2$ as a function of temperature, have been predicted by DFT-based phonon calculations, agreeing with available experimental data in the literature and validating the reliability of the present methodology. They enabled the remodeling of the (LiF, NaF, KF, CrF$_2$)-CrF$_3$ systems with more accurate inputs. The MQMQA is employed to describe the liquid phase, providing valuable insights into the complex nature of molten salts such as the short-range ordering and neighboring of cations. Phase equilibria from the present CALPHAD modeling match better with experimental data in comparison with previous modeling work in the literature. The present thermodynamic data, including equilibrium volumes, bulk moduli, enthalpies, entropies, and heat capacities of compounds in the (LiF, NaF, KF, CrF$_2$)-CrF$_3$ system can be used to facilitate the development of advanced molten salt reactors.

\chapter{Exploring and implementing thermodynamic models for complex liquid in open-source software PyCalphad} \label{chap:models}

\section{Introduction} \label{models:sec:intro}

\section{Implementation of UNIversal QUAsiChemical Model} \label{models:sec:UNIQUAC}


\subsection{UNIQUAC model} \label{models:ssec:UNIQUACfund}


\subsection{Calculations examples} \label{models:ssec:UNIQUACexamples}


\section{Custom model template generator} \label{models:sec:CMTG}


\subsection{Framework} \label{models:ssec:CMTGframework}


\subsection{Applications} \label{models:ssec:CMTGapp}


\section{Summary} \label{models:sec:Summary}


\chapter{Conclusion} \label{chap:conclusion}

\section{Conclusions} \label{conclusion:sec:conclusions}
The objectives of this research were to investigate atomic environments in the Pd-Zn-based intermetallic catalysts and fluoride and chloride molten salts using computational thermodynamics. A critical aspect of this work was selecting appropriate thermodynamic models to accurately describe the Gibbs energy of various phases. Within the CALPHAD community, the selection of models for solid phases is typically based on their structural symmetry and Wyckoff positions. For liquid phases, various models, such as the associate model, the two-sublattice ionic model, and MQMQA, are frequently employed to capture complex behaviors. The implementation of MQMQA in PyCalphad and ESPEI has facilitated the modeling of the same system using different models with uncertainty quantification and propagation into thermodynamic properties predictions. Chapter \ref{chap:method} discussed the application of Bayesian statistics in ESPEI, enabling robust statistical evaluation and comparison of liquid thermodynamic models for the first time. The use of Bayesian model selection in CALPHAD modeling enhances the reliability and accuracy in describing phases and predicting thermodynamic properties. 

Chapter \ref{chap:intermetallics} presented the thermodynamic modeling of the Pd-Zn system. The focus was on the $\gamma$-brass phase. 

Chapter \ref{chap:moltensalts}

In addition, a custom model template generator was developed for the efficient implementation of more thermodynamic models into PyCalphad. Chapter \ref{chap:models}



\section{Impact} \label{conclusion:sec:impact}



\section{Future work} \label{conclusion:sec:future}
% Implementation of more models
% Molten salts platform
% Automation of CALPHAD modeling, AI-driven



% Appendices

%\begin{appendices}

%\mypart{Appendices}

%\include{supdiscussions}

%\end{appendices}

\printbibliography[
heading=bibintoc,
title={Bibliography}
]


\newpage
\chapter*{Vita}
\thispagestyle{empty}
\vspace{-6pt}

{\small
Rushi Gong was born on October 15, 1997, in the city of Yiyang, Hunan Province in China. She graduated from Beihang University, Beijing, China in 2019 with a B.S. in Materials Science and Engineering and a minor in Mathematics. In 2020, Rushi began her Ph.D. studies at the Pennsylvania State University, advised by Dr. Zi-Kui Liu. Listed below are the publications she contributed to during her Ph.D.:
\fontsize{10}{10}\selectfont
%!TEX root = cv.tex
% TODO: Try to use bibliographic files here. Some WIP in min/minimal.tex
% TODO: Have some kind of highlighted flagging system where a short list could be generated with only key highlighted references.

\begin{etaremune}

% Formatting:
% \item <Authors>
% <Title>
% <Publication>
% <Link>

%QHA model selection
\item S.L. Shang, \textbf{R. Gong}, M.C. Gao, D. Pagan, and Z.K. Liu,
Revisiting First-Principles Thermodynamics by Quasiharmonic Approach: Application to Study Thermal Expansion of Additively-Manufactured Inconel 625,
\textbf{Scripta Materialia,} 250 (2024) 116200.
\href{https://doi.org/10.1016/j.scriptamat.2024.116200}
{doi: 10.1016/j.scriptamat.2024.116200}.
%FLiNaK-Cr Modeling
\item \textbf{R. Gong}, S.L. Shang, Y. Wang, J.P.S. Palma, H. Kim, and Z.K. Liu, 
Revisiting thermodynamics in (LiF, NaF, KF, CrF2)-CrF3 by first-principles calculations and CALPHAD modeling,
\textbf{Calphad}, 85, (2024),
\href{https://doi.org/10.1016/j.calphad.2024.102703}
{doi: 10.1016/j.calphad.2024.102703}.
%NEAMS activities
\item S. Shahbazi, M. Tano, S. Thomas, S. Walker, A.A. Jaoude, Y. Jeong, \textbf{R. Gong}, D.H. Kam, B. Chen, D. Grabaskas,
NEAMS activities supporting mechanistic source term model development for molten salt reactors,
\textbf{PSA conference}, 2023,
\href{https://doi.org/10.13182/PSA23-41261}
{doi: 10.13182/PSA23-41261}.
%MQMQA demo
\item J.P.S. Palma, \textbf{R. Gong}, B.J. Bocklund, R. Otis, M. Poschmann, M. Piro, S. Shahbazi, T.G. Levitskaia, S. Hu,  N.D. Smith, Y. Wang, H. Kim, Z.K. Liu, and S.L. Shang, 
Thermodynamic modeling with uncertainty quantification using the modified quasichemical model in quadruplet approximation: Implementation into PyCalphad and ESPEI,  
\textbf{Calphad}, 83, (2023),
\href{https://doi.org/10.1016/j.calphad.2023.102618}
{doi: 10.1016/j.calphad.2023.102618}.
%Nb-Ni Modeling
\item H. Sun, S.L. Shang, \textbf{R. Gong}, B.J. Bocklund, A.M. Beese, Z.K. Liu, 
Thermodynamic modeling of the Nb-Ni system with uncertainty quantification using PyCalphad and ESPEI, 
\textbf{Calphad}, 82, (2023),
\href{https://doi.org/10.1016/j.calphad.2023.102563}{doi: 10.1016/j.calphad.2023.102563}.
%Pd-Zn Modeling
\item \textbf{R. Gong}, S.L. Shang, H. Sun, M.J. Janik, and Z.K. Liu,
Thermodynamic modeling of the Pd-Zn system with uncertainty quantification and its implication to tailor catalysts, 
\textbf{Calphad}, 79, (2022),
\href{https://doi.org/10.1016/j.calphad.2022.102491}{doi: 10.1016/j.calphad.2022.102491}.
%Active site design
\item  A. Dasgupta, H. He, \textbf{R. Gong}, S.L. Shang, E.K. Zimmerer, R.J. Meyer, Z.K. Liu, M.J. Janik, and R.M. Rioux,
Atomic control of active site ensembles in ordered alloys to enhance hydrogenation selectivity, 
\textbf{Nature Chemistry}, 14, 523–529 (2022),
\href{https://doi.org/10.1038/s41557-021-00855-3}{doi: 10.1038/s41557-021-00855-3}.

\end{etaremune}
}

\end{document}
