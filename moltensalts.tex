\chapter{Thermodynamic modeling of fluoride and chloride molten salts with model selection, uncertainty quantification, and uncertainty propagation} \label{chap:moltensalts}

\section{Introduction} \label{moltensalts:sec:intro}

\section{Revisiting thermodynamics in (LiF, NaF, KF, CrF${_2}$)-CrF${_3}$ by first-principles calculations and CALPHAD modeling } \label{moltensalts:sec:FLiNaKCr}
The thermodynamic description of the (LiF, NaF, KF, CrF$_2$)-CrF$_3$ systems has been revisited, aiming for a better understanding of the effects of Cr on the FLiNaK molten salt. First-principles calculations based on DFT were performed to determine the electronic and structural properties of each compound, including the formation enthalpy, volume, and bulk modulus. DFT-based phonon calculations were carried out to determine the thermodynamic properties of compounds, for example, enthalpy, entropy, and heat capacity as functions of temperature. Phonon-based thermodynamic properties show a good agreement with experimental data of binary compounds LiF, NaF, KF, CrF$_3$, and CrF$_2$, establishing a solid foundation to determine thermodynamic properties of ternary compounds as well as to verify results estimated by the Neumann-Kopp rule. Additionally, DFT-based AIMD simulations were employed to predict the mixing enthalpies of liquid salts. Using DFT-based results and experimental data in the literature, the (LiF, NaF, KF, CrF$_2$)-CrF$_3$ system has been remodeled in terms of the CALPHAD approach using the MQMQA for liquid. Calculated phase stability in the present work shows an excellent agreement with experiments, indicating the effectiveness of combining DFT-based total energy, phonon, and AIMD calculations, and CALPHAD modeling to provide the thermodynamic description in complex molten salt systems.

\subsection{Modeling details} \label{moltensalts:ssec:FLiNaKCrmodel}
%%% Compounds information
The (LiF, NaF, KF, CrF$_2$)-CrF$_3$ system includes five binary (endmember) compounds, i.e., LiF, NaF, KF, CrF$_3$ and CrF$_2$, and eight ternary (intermetallic) compounds, i.e., Li$_3$CrF$_6$, Na$_3$CrF$_6$, Na$_5$Cr$_3$F$_{14}$, NaCrF$_4$, K$_3$CrF$_6$, K$_2$CrF$_5$, KCrF$_4$, and K$_2$Cr$_5$F$_{17}$. These ternary compounds were first suggested by De Kozak \cite{DeKozak1969} and confirmed by structural studies \cite{de1975systeme,miranday1975croissance, sturm1962phase, garcia2014electrostatic, brunton1969crystal, le2003distorted, manaka2011effects, sassoye2006crystal} via the X-ray diffraction (XRD) method, which was summarized by Dumaire et al. \cite{dumaire2021thermodynamic}. However, thermochemical data of these compounds are scarce. Yin et al. \cite{yin2018thermodynamic, yin2015thermodynamic, yin2014thermodynamic} performed DFT calculations at 0 K to determine the formation enthalpies of Li$_3$CrF$_6$, Na$_3$CrF$_6$, Na$_5$Cr$_3$F$_{14}$, NaCrF$_4$, and KCrF$_4$. Dumaire et al. \cite{dumaire2021thermodynamic} estimated the heat capacities of these compounds based on the Neumann-Kopp rule in terms of the compositional average of heat capacity values of the corresponding compounds or elements \cite{leitner2010application}. 

%%% Modeling and First-principles details for compounds
The present work treats the compounds and endmembers in the AF-CrF${_3}$ (A=Li, Na, and K) systems as stoichiometric compounds. Thermodynamic functions of the binary endmembers are taken from the JRC database \cite{konings2020comprehensive}, JANAF tables \cite{chase1982janaf}, IVTAN tables \cite{gurvich1993ivtanthermo}, and SSUB database \cite{sgteurl}. The Gibbs energy can be expressed as 

\begin{equation} \label{ms:eq:Gstoi}
    G_m=\Delta_f H_m^0 (298.15)-T S_m^0 (298.15)+\int_{298.15}^T C_{P,m} dT - T\int_{298.15}^T \dfrac{C_{P,m}}{T} dT
\end{equation}

where $\Delta_f H_m^0 (298.15)$ is the standard formation enthalpy, $S_m^0 (298.15)$ the standard entropy at 298.15 K, and $C_{P,m}$ the heat capacity. The thermodynamic data of ternary compounds, including enthalpy, entropy, and heat capacity are obtained through DFT-based first-principles and phonon calculations (see Section \ref{method:ssec:dft}).

All DFT-based first-principles and phonon calculations in the present work were performed by the VASP \cite{kresse1996efficient} using the open-source software DFTTK \cite{wang2021dfttk}. The projector augmented-wave method (PAW) was used for electron-ion interactions to increase computational efficiency compared with the full potential methods \cite{blochl1994projector, kresse1999ultrasoft}. Electron exchange and correlation effects were described using both the local density approximation (LDA) \cite{perdew1981self} and the GGA \cite{perdew1996generalized}. In addition, the DFT+U approach was employed for 11 compounds containing Cr, i.e., CrF${_2}$, CrF${_3}$, Cr$_2$F$_5$, Li$_3$CrF$_6$, Na$_3$CrF$_6$, Na$_5$Cr$_3$F$_{14}$, NaCrF$_4$, K$_3$CrF$_6$, K$_2$CrF$_5$, KCrF$_4$, K$_2$Cr$_5$F$_{17}$. The effective U values for Cr were selected as 3.7 eV, considering 3$-$4eV was commonly used in the literature \cite{shi2009magnetism, mattsson2019density, huang2022dft}. The spin configurations were also considered for these 11 compounds containing Cr. All possible configurations by varying spin up and spin down of Cr atoms were explored by the ATAT code \cite{van2009multicomponent}. The spin configuration with the lowest energy for each Cr-containing compound was used for DFT and phonon calculations. Using DFTTK, structure information is the only required input, then robust relaxation schemes can be automatically performed to obtain equilibrium results at 0 K and thermodynamic properties at finite temperatures through the QHA. During DFTTK calculations, the plane-wave cutoff energy was set as 520 eV. Table \ref{ms:tab:DFTdetails} lists the k-points meshes for DFT-based total energy calculations, the supercell sizes and the k-points meshes for phonon calculations. The phonon DOS was analyzed using the YPHON code \cite{wang2014yphon}, which has been integrated into DFTTK \cite{wang2021dfttk}. 

\begin{table}[H]
    \centering
    \caption{Details of DFT-based first-principles calculations for each compound or phase, including space group, total atoms in the supercells, k-point meshes for structure relaxations, and the final static calculations (indicated by DFT), supercell sizes for phonon calculations, k-point meshes for phonon calculations.}
    \begin{tabular}{>{\raggedright\arraybackslash}m{2.5cm}>{\raggedright\arraybackslash}m{2cm}>{\raggedright\arraybackslash}m{2.5cm}>{\raggedright\arraybackslash}m{2.5cm}>{\raggedright\arraybackslash}m{2.8cm}>{\raggedright\arraybackslash}m{2.5cm}}
    \hline
     \textbf{Phase}&\textbf{Space group}&\textbf{Atoms in crystallographic cell}&\textbf{k-points for DFT}&\textbf{Atoms in supercell for phonon}&\textbf{k-points for phonon}\\
    \hline
        LiF&$Fm\Bar{3}m$&8&$10\times10\times10$&32&$10\times10\times10$\\
        NaF&$Fm\Bar{3}m$&8&$10\times10\times10$&32&$10\times10\times10$\\
        KF&$Fm\Bar{3}m$&8&$10\times10\times10$&32&$10\times10\times10$\\
        CrF${_3}$&$R\Bar{3}c$&24&$9\times9\times3$&24&$9\times9\times3$\\
        CrF${_2}$&$P2_1/c$&6&$14\times10\times9$&24&$13\times10\times9$\\
        Li$_3$CrF$_6$&$C2/c$&60&$2\times2\times2$&60&$2\times2\times2$\\
        Na$_3$CrF$_6$&$P2_1/c$&20&$9\times8\times5$&40&$9\times8\times5$\\
        Na$_5$Cr$_3$F$_{14}$&$P2_1/c$&44&$6\times6\times3$&44&$6\times6\times3$\\
        NaCrF$_4$&$P2_1/c$&24&$6\times8\times5$&24&$6\times8\times6$\\
        K$_3$CrF$_6$&$Fm\Bar{3}m$&40&$5\times5\times5$&40&$5\times5\times5$\\
        K$_2$CrF$_5$&$Pbcn$&128&$3\times1\times1$&128&$3\times1\times1$\\
        KCrF$_4$&$Pnma$&144&$3\times1\times1$&144&$3\times1\times1$\\
        K$_2$Cr$_5$F$_{17}$&$Cmcm$&96&$2\times2\times2$&96&$2\times2\times2$\\
        Cr$_2$F$_5$&$C2/c$&28&$6\times6\times6$&N/A&N/A\\
    \hline
    \end{tabular}
    \label{ms:tab:DFTdetails}
\end{table}

%%% CALPHAD modeling details
The MQMQA \cite{pelton2001modified} is used to describe the liquid phase by considering the short-range ordering (SRO) that occurs in liquid salts. Here, the model (A, Cr)(F) is introduced and hence the A$_2$F$_2$, Cr$_2$F$_2$, and ACrF$_2$ quadruplets (A=Li, Na, and K) are formed to consider the interactions among them. Coordination numbers Z describe the SNN coordination number of the species i (= Li, Na, K, Cr, or F) in quadruplets. Z of anions can be calculated to maintain charge neutrality as follows:

\begin{equation} \label{ms:eq:MQMZ}
    \dfrac{q_A}{(Z_{AB/FF}^A)}+\dfrac{q_B}{(Z_{AB/FF}^B)}=2\times \dfrac{q_F}{(Z_{AB/FF}^F)}
\end{equation}

where $q_i$ is the charges of ion $i$ (= Li, Na, K, Cr, or F). Table \ref{ms:tab:CrZ} shows the coordination numbers used in the present work.

\begin{table}[H]
    \centering
    \caption{Coordination number used in the present CALPHAD modeling work with MQMQA for the liquid phase.}
    \begin{tabular}{>{\raggedright\arraybackslash}m{2.5cm}>{\raggedright\arraybackslash}m{2.5cm}>{\raggedright\arraybackslash}m{2.5cm}>{\raggedright\arraybackslash}m{2.5cm}>{\raggedright\arraybackslash}m{2.5cm}}
    \hline
    \textbf{A}&\textbf{B}&\textbf{$Z_{AB/FF}^A$}&\textbf{$Z_{AB/FF}^B$}&\textbf{$Z_{AB/FF}^F$}\\
    \hline
    Li$^+$&Li$^+$&6&6&6 \\
    Na$^+$&Na$^+$&6&6&6\\
    K$^+$&K$^+$&6&6&6\\
    Cr$^{3+}$&Cr$^{3+}$&6&6&2\\
    Li$^+$&Cr$^{3+}$&2&6&2\\
    Na$^+$&Cr$^{3+}$&4&6&2.7\\
    K$^+$&Cr$^{3+}$&6&6&3\\
    Cr$^{2+}$&Cr$^{3+}$&6&6&2.4\\
    Cr$^{2+}$&Cr$^{2+}$&6&6&3\\
    \hline
    \end{tabular}
    \label{ms:tab:CrZ}
\end{table}

The excess Gibbs energy $G^{excess}$ relates to the formation Gibbs energy of the quadruplet, $\Delta g_{quadruplet}^{ex}$, by considering the following reaction:

\begin{equation} \label{ms:eq:MQMGrea}
    \left(\rm {A_2/F_2}\right)_{quad}+\left({\rm Cr_2/F_2}\right)_{quad}=2\left({\rm ACr/F_2}\right)_{quad}\;\;\;\;\;\Delta g_{\rm ACr/F_2}^{ex}
\end{equation}

where $\Delta g_{\rm ACr/F_2}^{ex}$ represents the Gibbs energy change when forming the quadruplet and can be described by:

\begin{equation} \label{ms:eq:MQMGex}
    \Delta g_{\rm {ACr/F_2}}^{ex}=\Delta g_{\rm {ACr/F_2}}^o+\sum_{(i+j)\geq1} g_{\rm {ACr/F_2}}^{ij}\chi _{\rm {ACr/F_2}}^{i}\chi _{\rm {CrA/F_2}}^{j}
\end{equation}

where $g_{\rm {ACr/F_2}}^{ij}$ is a function of temperature and it is independent of composition. $\chi _{\rm {ACr/F_2}}^{i}$ and $\chi _{\rm {CrA/F_2}}^{j}$ are composition-dependent terms as:

\begin{equation} \label{ms:eq:chi}
    \chi_{\rm {ACr/F_2}} = \dfrac{X_{\rm {A_2/F_2}}}{X_{\rm {A_2/F_2}}+X_{\rm {ACr/F_2}}+X_{\rm {Cr_2/F_2}}}
\end{equation}

where $X_{\rm{ACr/F_2}}$ is the fractions of $\left(\rm{ACr/F_2}\right)_{quad}$ shown in (\ref{ms:eq:MQMGrea}). 

In the CrF${_2}$-CrF${_3}$ system, there are three solid solution phases, i.e., CrF${_2}$ near the Cr-rich region, CrF${_3}$ near the F-rich region, and Cr$_2$F$_5$ showing on the middle region of the CrF${_2}$-CrF${_3}$ phase diagram. The present work adopts the same models used by Dumaire et al. \cite{dumaire2021thermodynamic}, where the regular solution model with the Kohler-Toop interpolation \cite{kohler1960estimation, toop1965predicting, chartrand2000choice, pelton2001general} is used for Cr$_2$F$_5$. For solid solution phases near CrF${_2}$ and CrF${_3}$, the sublattice model is used for each phase, respectively, considering the Wyckoff positions of CrF${_2}$ and CrF${_3}$ as follows. CrF${_2}$ possesses the symmetry with space group $P2_1/c$ with two Wyckoff sites of 2a and 4e, and the sublattice model (Cr, Va)$_1$(F, Va)$_2$ is hence used with Va representing the vacancy. CrF${_3}$ phase is modeled by (Cr, Va)$_1$(F, Va)$_3$ by considering its space group $R\bar{3}c$ and the two Wyckoff sites of 2b and 6e. The Gibbs energy is formulated as:

\begin{equation} \label{ms:eq:Crssoln}
    \begin{aligned}
        G_m&=\sum_{i=Cr,Va}{\sum_{j=F,Va}{y_i^\prime y_j^{\prime\prime}}\:^oG_{i:j}}\\
        &+RT\left(\sum_{i=Cr,Va}{y_i^\prime\ln{\left(y_i^\prime\right)}}+\sum_{j=F,Va}{y_j^{\prime\prime}\ln{\left(y_j^{\prime\prime}\right)}}\right)\\&+y_{Cr}^\prime y_{Va}^\prime\left(y_F^{\prime\prime}L_{Cr,Va:F}\right)+y_F^{\prime\prime}y_{Va}^{\prime\prime}\left(y_{Cr}^\prime L_{Cr:F,Va}\right)
    \end{aligned}
\end{equation}

where $y_i^{(s)}$ is the site fraction of component i on sublattice s, ${^o}G_{i:j}$ the Gibbs energy of the endmember (i:j), and $L$ the interaction parameters which can be expanded using the Redlich-Kister polynomials \cite{redlich1948algebraic}. 

Phase equilibria in the LiF-CrF${_3}$, NaF-CrF${_3}$, and KF-CrF${_3}$ binary systems were investigated by De Kozak \cite{de1975systeme, DeKozak1969} using differential thermal analysis (DTA). In LiF-CrF${_3}$, two eutectic reactions were measured, i.e., Liquid $\leftrightarrow$ LiF + Li$_3$CrF$_6$ at 1003 K and around mole fraction $x$(CrF${_3}$) = 0.15 and Liquid $\leftrightarrow$ CrF${_3}$ + Li$_3$CrF$_6$ at 1059 K and $x$(CrF${_3}$) = 0.35. In NaF-CrF${_3}$, one peritectic reaction of Liquid + CrF${_3}$ $\leftrightarrow$ NaCrF$_4$ at 1234 K and three eutectic reactions were determined, i.e., Liquid $\leftrightarrow$ NaCrF$_4$ + Na$_5$Cr$_3$F$_{14}$ at 1133 K, Liquid $\leftrightarrow$ Na$_3$CrF$_6$ + Na$_5$Cr$_3$F$_{14}$ at 1143 K, and Liquid $\leftrightarrow$ Na$_3$CrF$_6$ + NaF at 1166 K and around $x$(CrF${_3}$) = 0.123. In KF-CrF${_3}$, De Kozak \cite{de1975systeme, DeKozak1969} reported three peritectic reactions and two eutectic reactions, i.e., Liquid + CrF${_3}$ $\leftrightarrow$ K$_2$Cr$_5$F$_{17}$ at 1390 K, Liquid + K$_3$CrF$_6$ $\leftrightarrow$ K$_2$CrF$_5$ at 1133 K, and Liquid + K$_2$Cr$_5$F$_{17}$ $\leftrightarrow$ KCrF$_4$ at 1200 K, and Liquid $\leftrightarrow$ K$_3$CrF$_6$ + KF at 1115 K and around $x$(CrF${_3}$) = 0.048, and Liquid $\leftrightarrow$ K$_2$CrF$_5$ + KCrF$_4$ at 1112 K and around $x$(CrF${_3}$) = 0.45. Sturm \cite{sturm1962phase} reported phase equilibria in CrF${_2}$-CrF${_3}$ via quenching experiments and suggested one solid solution phase in CrF${_2}$-CrF${_3}$ with composition of CrF${_3}$ between 0.42 and 0.46 (near Cr$_2$F$_5$). However, the stability of this Cr$_2$F$_5$ solid solution phase was not explored in temperatures below 1023 K. The melting point of Cr$_2$F$_5$ was determined to be around 1270 K \cite{sturm1962phase}. Sturm \cite{sturm1962phase} reported one eutectic reaction, Liquid $\leftrightarrow$ CrF${_2}$ + Cr$_2$F$_5$ at 1103 K around $x$(CrF${_3}$) = 0.14, and one peritectic reaction Liquid + CrF${_3}$ $\leftrightarrow$ Cr$_2$F$_5$ at 1272 K around $x$(CrF${_3}$) = 0.29. Two solid solution phases near the endmembers CrF${_3}$ and CrF${_2}$ were identified from $x$(CrF${_3}$) = 0$-$0.01 and $x$(CrF${_3}$) = 0.90$-$1, respectively. 

Machine learning (ML) was used to estimate more phase equilibria data in terms of a graphic neural network model developed by Hong et al. \cite{hong2022melting} to predict melting points of compounds with composition as input. Melting temperatures of the present ternary compounds including Li$_3$CrF$_6$, Na$_3$CrF$_6$, Na$_5$Cr$_3$F$_{14}$, NaCrF$_4$, K$_3$CrF$_6$, K$_2$CrF$_5$, KCrF$_4$, and K$_2$Cr$_5$F$_{17}$ are estimated by this ML model \cite{hong2022melting}. 

%%%AIMD
For the liquid phase, experimental data such as mixing enthalpy are not available in the AF-CrF${_3}$ (A=Li, Na, and K) systems. Instead, Yin et al. \cite{yin2018thermodynamic, yin2015thermodynamic, yin2014thermodynamic} applied an empirical model to estimate the mixing enthalpy of liquid from the parameters of ions such as ionic radius. In the present work, AIMD simulations (see Section \ref{method:ssec:AIMD}) were performed to obtain the mixing enthalpy of liquid by VASP \cite{kresse1996efficient}. The supercells containing 108 or 128 atoms with periodic boundaries were used for at least six different compositions in the AF-CrF${_3}$ (A= Li, Na, and K) systems, including A$_{64}$F$_{64}$, A$_{42}$Cr$_6$F$_{60}$, A$_{36}$Cr$_9$F$_{63}$, A$_{32}$Cr$_{16}$F$_{80}$, A$_{18}$Cr$_{18}$F$_{72}$, A$_{16}$Cr$_{24}$F$_{88}$, A$_{10}$Cr$_{22}$F$_{76}$, and Cr$_{32}$F$_{96}$ (A=Li, Na, and K). The NVT canonical ensemble (i.e., the fixed number of atoms N, volume V, and temperature T) with a Nosé thermostat for temperature control \cite{nose1984unified} was employed in the present work. The temperature for each supercell was set as 1700 K, which is above all the temperatures of liquidus in the AF-CrF${_3}$ (A= Li, Na, and K) systems. A single $\Gamma$ point $1\times1\times1$ was used as the k-point mesh, together with a 400 eV cutoff energy. During AIMD simulations, Newton’s equation of motion was solved via the Verlet algorithm with a time step of 2 fs, and each calculation is run for 10,000 steps to reach thermal equilibrium.

%%%CALPHAD modeling
Thermodynamic modeling of the (LiF, NaF, KF, CrF${_2}$)-CrF${_3}$ system was carried out using the open-source software ESPEI \cite{bocklund2019espei} and PyCalphad \cite{otis2017pycalphad} with the newly implemented MQMQA \cite{palma2023thermodynamic}. All model parameters were simultaneously optimized through the Bayesian approach using MCMC \cite{bocklund2019espei}. The input data were primarily experimental phase equilibrium data including two or more co-existing phases. For stochiometric compounds, their thermochemical data from DFT-based calculations were also used as input. For the liquid phase, its mixing enthalpy from AIMD calculations was used as input for refining model parameters. In the present work, each model parameter employed two Markov chains with a standard derivation of 0.1 when initializing its Gaussian distribution. During the modeling process, the chain values can be tracked and the MCMC processes were performed until the model parameters converged.

\subsection{Thermodynamic properties in (LiF, NaF, KF, CrF$_2$)-CrF$_3$ by first-principles calculations} \label{moltensalts:ssec:FLiNaKCrsolids}





\section{Thermodynamic modeling of LiF-LnF${_3}$ systems with uncertainty propagation to vapor-liquid equilibrium property} \label{moltensalts:sec:LiFLnF3}

\subsection{Literature review} \label{moltensalts:ssec:LiFLnF3lit}


\subsection{Modeling details} \label{moltensalts:ssec:LiFLnF3model}


\subsection{Results and discussion} \label{moltensalts:ssec:LiFLnF3result}


\section{Bayesian model selection in thermodynamic modeling of LiCl-KCl-LaCl${_3}$} \label{moltensalts:sec:LaCl3}

\subsection{Literature review} \label{moltensalts:ssec:LaCl3lit}


\subsection{Modeling details} \label{moltensalts:ssec:LaCl3model}


    \subsection{Results and discussion} \label{moltensalts:ssec:LaCl3result}


\section{Summary} \label{moltensalts:sec:Summary}
