\chapter{Conclusion} \label{chap:conclusion}

\section{Conclusions} \label{conclusion:sec:conclusions}
The objectives of this research were to investigate atomic environments in solids (the Pd-Zn-based intermetallic catalysts) and liquids (fluoride and chloride molten salts) using computational thermodynamics. A critical aspect of this work was selecting appropriate thermodynamic models to accurately describe the Gibbs energy of various phases. Within the CALPHAD community, the selection of models for solid phases is typically based on their structural symmetry and Wyckoff positions. For liquid phases, various models, such as the associate model, the two-sublattice ionic model, and MQMQA, are frequently employed to capture complex behaviors. The implementation of MQMQA in PyCalphad and ESPEI has facilitated the modeling of the same system using different models with uncertainty quantification and propagation into thermodynamic properties predictions. Chapter \ref{chap:method} discussed the application of Bayesian statistics in ESPEI, enabling robust statistical evaluation and comparison of liquid thermodynamic models for the first time. The use of Bayesian model selection in CALPHAD modeling enhances the reliability and accuracy in describing phases and predicting thermodynamic properties. 

Chapter \ref{chap:intermetallics} presented the thermodynamic modeling of the Pd-Zn-M (M = Cu, Ag, and Au) system with the aid of first-principles calculations, with a focus on the $\gamma$-brass phase. A four-sublattice model was employed to describe four Wyckoff positions, i.e., OT, IT, OH, and CO, to facilitate the investigation of site occupancy by the active metals Pd, Cu, Ag, and Au. This model reveals that, in the binary Pd-Zn $\gamma$-brass phase, Pd preferentially occupies OT sites first, followed by OH sites, leading to the formation of Pd monomers and Pd3 trimers on the catalyst surfaces. When alloyed with third metals Cu, Ag, and Au, CALPHAD modeling predicts their occupancy at OH sites, thereby forming Pd-M-Pd trimers. These variations in active site distribution significantly influence the selectivity of hydrogenation reactions.

Chapter \ref{chap:moltensalts} addressed the selection of thermodynamic models for describing atomic environments in fluoride and chloride molten salts. The (LiF, NaF, KF, CrF${_2}$)-CrF${_3}$ system was modeled using the MQMQA method for the liquid phase. The accuracy of the thermodynamic modeling was enhanced through first-principles calculations for solid phases and AIMD simulations for the liquid phase. A comparative analysis of the associate model, two-sublattice ionic model, and MQMQA was conducted for the KCl-LaCl${_3}$ system. Bayesian parameter estimation was employed in the optimization process, indicating that MQMQA was favored by the input data, as supported by the Bayes factor. The ternary LiCl-KCl-LaCl${_3}$ system was also modeled using MQMQA, with uncertainty quantification and propagation performed to evaluate the model's performance in predicting thermodynamic properties.

Beyond the CALPHAD community, numerous models are available for describing the non-ideal behavior of the liquid phase.  Chapter \ref{chap:models} demonstrated the process of implementation of the UNIQUAC model into the open-source software PyCalphad. Benchmarking was performed to verify the calculations of Gibbs energy and its derivatives, as well as phase diagram calculations and plotting, demonstrating the successful implementation of the model. In addition, a custom model template generator was developed to facilitate the efficient implementation of additional thermodynamic models into PyCalphad. Chapter \ref{chap:models} also illustrates the use of the template generator with the Peng-Robinson Equation of State, which significantly enhanced the efficiency of integrating custom models into PyCalphad. The newly implemented model can directly take advantage of high-throughput modeling capabilities with uncertainty quantification within PyCalphad and ESPEI.

\section{Impact} \label{conclusion:sec:impact}
% Scientific
The selection of 
% Tools


\section{Future work} \label{conclusion:sec:future}
% Implementation of more models
% Molten salts platform
% Automation of CALPHAD modeling, AI-driven
