\chapter{Conclusion} \label{chap:conclusion}

\section{Conclusions} \label{conclusion:sec:conclusions}
The objectives of this research were to investigate atomic environments in solids (the Pd-Zn-based intermetallic catalysts) and liquids (fluoride and chloride molten salts) using computational thermodynamics. A critical aspect of this work was selecting appropriate thermodynamic models to accurately describe the Gibbs energy of various phases. Within the CALPHAD community, the selection of models for solid phases is typically based on their structural symmetry and Wyckoff positions. For liquid phases, various models, such as the associate model, the two-sublattice ionic model, and MQMQA, are frequently employed to capture complex behaviors. Chapter \ref{chap:method} provided a detailed discussion of the Gibbs energy described by these models. The implementation of MQMQA in PyCalphad and ESPEI has facilitated the modeling of the same system using different models with uncertainty quantification and propagation into thermodynamic properties predictions. Chapter \ref{chap:method} discussed the application of Bayesian statistics in ESPEI, enabling robust statistical evaluation and comparison of liquid thermodynamic models for the first time. The use of Bayesian model selection in CALPHAD modeling enhances the reliability and accuracy in describing phases and predicting thermodynamic properties. 

Chapter \ref{chap:intermetallics} presented the thermodynamic modeling of the Pd-Zn-M (M = Cu, Ag, and Au) system with the aid of first-principles calculations, with a focus on the $\gamma$-brass phase. A four-sublattice model was employed to describe four Wyckoff positions, i.e., OT, IT, OH, and CO, to facilitate the investigation of site occupancy by the active metals Pd, Cu, Ag, and Au. This model reveals that, in the binary Pd-Zn $\gamma$-brass phase, Pd preferentially occupies OT sites first, followed by OH sites, leading to the formation of Pd monomers and Pd3 trimers on the catalyst surfaces. When alloyed with third metals Cu, Ag, and Au, CALPHAD modeling predicts their occupancy at OH sites, thereby forming Pd-M-Pd trimers. These variations in active site distribution significantly influence the selectivity of hydrogenation reactions.

Chapter \ref{chap:moltensalts} addressed the selection of thermodynamic models for describing atomic environments in fluoride and chloride molten salts. The (LiF, NaF, KF, CrF${_2}$)-CrF${_3}$ system was modeled using the MQMQA method for the liquid phase. The accuracy of the thermodynamic modeling was enhanced through first-principles calculations for solid phases and AIMD simulations for the liquid phase. A comparative analysis of the associate model, two-sublattice ionic model, and MQMQA was conducted for the KCl-LaCl${_3}$ system. Bayesian parameter estimation was employed in the optimization process, indicating that MQMQA was favored by the input data, as supported by the Bayes factor. The ternary LiCl-KCl-LaCl${_3}$ system was also modeled using MQMQA, with uncertainty quantification and propagation performed to evaluate the model's performance in predicting thermodynamic properties.

Beyond the CALPHAD community, numerous models are available for describing the non-ideal behavior of the solution phase.  Chapter \ref{chap:models} demonstrated the process of implementation of the UNIQUAC model into the open-source software PyCalphad. Benchmarking was performed to verify the calculations of Gibbs energy and its derivatives, as well as phase diagram calculations and plotting, demonstrating the successful implementation of the model. In addition, a custom model template generator was developed to facilitate the efficient implementation of additional thermodynamic models into PyCalphad. Chapter \ref{chap:models} also illustrates the use of the template generator with the Peng-Robinson Equation of State, which significantly enhanced the efficiency of integrating custom models into PyCalphad. The newly implemented model can directly take advantage of high-throughput modeling capabilities with uncertainty quantification within PyCalphad and ESPEI.

\section{Impact} \label{conclusion:sec:impact}
% Implementation of more models
A systematic framework for comparing and selecting thermodynamic models for phases has been developed. Bayesian model selection in CALPHAD modeling has been made available for the community by using ESPEI for parameter optimization with MCMC. A custom model template generator has been provided to the community to develop and implement any thermodynamic models into PyCalphad. Once models are implemented in PyCalphad, they can fully utilize all existing features in both PyCalphad and ESPEI, including Bayesian parameter estimation for optimization and uncertainty quantification for model evaluation. The open-source nature of all the provided codes ensures that users can easily customize, modify, and collaborate on the development and enhancement of thermodynamic models.

All thermodynamic modeling database files have been made publicly accessible. The application of the CALPHAD approach to investigating site occupancy has proven valuable in guiding the future design of intermetallic catalysts, including Cu-In-based catalysts. Additionally, the modeling work on the (LiF, NaF, KF, CrF${_2}$)-CrF${_3}$ and LiCl-KCl-LaCl${_3}$ systems can be integrated into the open-source database MSTDB-TC \cite{ard2022development}, facilitating the investigation of higher-order system equilibria.

\section{Future work} \label{conclusion:sec:future}
Several thermodynamic models widely used in the broader scientific community can be implemented into PyCalphad and ESPEI with the custom model template generator. The Peng-Robinson equation of state, discussed in Chapter \ref{chap:models}, is currently being implemented in PyCalphad. Given its popularity in the chemical engineering community, especially the petrochemical industry, the integration of PR EOS with ESPEI's powerful parameter optimization and uncertainty quantification capabilities will significantly enhance the accuracy and efficiency of multi-component fluid simulations using this model. 

Uncertainty quantification in CALPHAD modeling is anticipated to play an increasingly important role in materials design. The uncertainty associated with model parameters has been thoroughly examined \cite{paulson2019quantified} and can be propagated into property predictions. This work explored the use of the Bayes factor for statistical comparison between thermodynamic models. The marginal likelihood, which reflects the probability of the input data given the current model, serves as an indicator of how well the model describes the input data, thereby quantifying model performance. However, alternative methods for estimating marginal likelihood can provide a more nuanced understanding of model performance. Additionally, it is crucial to address the incorporation of input data uncertainty into CALPHAD modeling, which remains an area for further development. Greater attention and research are needed to thoroughly investigate the uncertainties associated with both the model itself and the input data.

CALPHAD modeling involves managing a substantial data flow. The process requires the collection of experimental data for parameter optimization, but such data—particularly for systems like molten salts—can be scarce due to the challenges of conducting experiments under extreme conditions. As machine learning (ML) and atomistic simulation methods including DFT and ML-based Molecular Dyanmiacs continue to advance, large volumes of data are being generated that can be integrated into CALPHAD modeling to enhance its accuracy and scope. Additionally, CALPHAD modeling itself produces extensive information, including property predictions and phase equilibrium data. This growing complexity underscores the need for a robust platform that facilitates the exchange and storage of these diverse data sets, enabling more efficient collaboration within the broad community. Building on this data foundation, a data-driven, automated CALPHAD modeling workflow can be developed. With the open-source CALPHAD modeling tools PyCalphad and ESPEI as computational engines, this workflow can maximize the robustness of data collection and enable smooth conversion of data formats. Moreover, it can minimize manual processes during parameterization and refinement, thereby significantly enhancing high-throughput CALPHAD modeling and accelerating progress in materials design and discovery. 

