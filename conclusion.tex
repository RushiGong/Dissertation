\chapter{Conclusion} \label{chap:conclusion}

\section{Conclusions} \label{conclusion:sec:conclusions}
The objectives of this research were to investigate atomic environments in the Pd-Zn-based intermetallic catalysts and fluoride and chloride molten salts using computational thermodynamics. A critical aspect of this work was selecting appropriate thermodynamic models to accurately describe the Gibbs energy of various phases. Within the CALPHAD community, the selection of models for solid phases is typically based on their structural symmetry and Wyckoff positions. For liquid phases, various models, such as the associate model, the two-sublattice ionic model, and MQMQA, are frequently employed to capture complex behaviors. The implementation of MQMQA in PyCalphad and ESPEI has facilitated the modeling of the same system using different models with uncertainty quantification and propagation into thermodynamic properties predictions. Chapter \ref{chap:method} discussed the application of Bayesian statistics in ESPEI, enabling robust statistical evaluation and comparison of liquid thermodynamic models for the first time. The use of Bayesian model selection in CALPHAD modeling enhances the reliability and accuracy in describing phases and predicting thermodynamic properties. 

Chapter \ref{chap:intermetallics} presented the thermodynamic modeling of the Pd-Zn system. The focus was on the $\gamma$-brass phase. 

Chapter \ref{chap:moltensalts}

In addition, a custom model template generator was developed for the efficient implementation of more thermodynamic models into PyCalphad. Chapter \ref{chap:models}



\section{Impact} \label{conclusion:sec:impact}



\section{Future work} \label{conclusion:sec:future}
% Implementation of more models
% Molten salts platform
% Automation of CALPHAD modeling, AI-driven
