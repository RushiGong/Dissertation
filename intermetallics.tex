\chapter{Thermodynamic modeling of the Pd-Zn-M system with uncertainty quantification and its implication to tailor catalysts} \label{chap:intermetallics}

\section{Introduction} \label{intermetallics:sec:intro}

\section{Thermodynamic modeling of the Pd-Zn system with uncertainty quantification} \label{intermetallics:sec:PdZn}
Pd–Zn intermetallic catalysts show encouraging combinations of activity and selectivity on well-defined active site ensembles. Thermodynamic description of the Pd–Zn system, delineating phase boundaries and enumerating site occupancies within intermediate alloy phases, is essential to determining the ensembles of Pd–Zn atoms as a function of composition and temperature. 

Combining the present extensive first-principles calculations based on density functional theory (DFT) and available experimental data, the Pd–Zn system was remodeled using the CALculation of PHAse Diagrams (CALPHAD) approach. High throughput modeling tools with uncertainty quantification, i.e., ESPEI and PyCalphad, were incorporated in the phase analysis. The site occupancies across the $\gamma$ phase composition region were given special attention. A four-sublattice model was used for the $\gamma$ phase owing to its four Wyckoff positions, i.e., the outer tetrahedral (OT) site 8c, the inner tetrahedral (IT) site 8c, the octahedral (OH) site 12e, and the cuboctahedral (CO) site 24g. The site fractions of Pd and Zn calculated from the present thermodynamic model show the occupancy preference of Pd in the OT and OH sublattices in agreement with experimental observations. The force constants obtained from DFT-based phonon calculations further supports the tendency of Pd occupying the OH sublattice compared with the IT and CO sublattice. The catalytic ensembles changing from Pd monomers (Pd1) to trimers (Pd3) on the surface of $\gamma$ phase are attributed to the increase of Pd occupancy in the OH sublattice.

\subsection{Modeling details} \label{intermetallics:ssec:PdZnmodel}


\subsection{Properties of Pd–Zn compounds by first-principles calculations} \label{intermetallics:ssec:PdZndft}

\subsection{Thermodynamic modeling and phase equilibria} \label{intermetallics:ssec:PdZneq}


\subsection{Site occupancy in the \texorpdfstring{$\gamma$}--phase and surface construction} \label{intermetallics:ssec:PdZnsite}


\section{Determination of site occupancy in the M-Pd-Zn (M = Cu, Ag, and Au) \texorpdfstring{$\gamma$}--brass phase} \label{intermetallics:sec:PdZnM}

\subsection{Literature review} \label{intermetallics:ssec:PdZnMlit}


\subsection{Modeling details} \label{intermetallics:ssec:PdZnMmodel}


\subsection{Results and discussion} \label{intermetallics:ssec:PdZnMresult}


\section{Summary} \label{intermetallics:sec:Summary}
