\chapter{Introduction} \label{sec:Introduction}

\section{CALPHAD modeling with model selection} \label{intro:sec:calphad}
The CALPHAD (Calculation of Phase Diagrams) approach \cite{liu2020computational, lukas2007computational} is a powerful computational thermodynamics methodology to predict the thermodynamic properties and phase behaviors of multicomponent systems. By leveraging a combination of experimental data and theoretical models, CALPHAD enables the construction of comprehensive thermodynamic databases that describe the relationships between temperature, composition, and thermodynamic properties. This systematic approach facilitates the development of phase diagrams and other critical thermodynamic information essential for materials design, processing, and optimization. CALPHAD modeling relies on the accurate representation of Gibbs energy functions for different phases. The atomic environment refers to the arrangement and interactions of atoms within a phase, which significantly influence its thermodynamic properties. Different phases exhibit varying degrees of atomic order and interaction complexities, requiring tailored modeling approaches to capture their unique behaviors accurately. The selection of appropriate models to describe Gibbs energy functions of these phases is crucial, as it directly impacts the accuracy and reliability of the predicted thermodynamic properties and phase equilibrium. 

The selection of a thermodynamic model depends on factors such as the nature of the phase, the presence of chemical ordering, and short-range interactions. Various models, including the compound energy formalism (CEF) \cite{hillert1970regular}, the associate model \cite{sommer1982association}, the two-sublattice ionic model \cite{hillert1985two}, and the modified quasichemical model (MQM) \cite{pelton2018phase}, are employed to capture the complexities of solid and liquid phases. However, systematically comparing different models remains challenging, as it is difficult to quantify the performance of the model in predicting both phase equilibria and thermochemical properties.

Recent advancements in computational tools and open-source software, such as PyCalphad \cite{otis2017pycalphad} and ESPEI \cite{bocklund2019espei}, have significantly enhanced the capabilities of CALPHAD modeling. The incorporation of Bayesian parameter estimation in thermodynamic modeling has enabled uncertainty quantification and statistical evaluation of model performance. These tools provide robust platforms for developing, validating, and implementing thermodynamic models, providing opportunities for high-throughput computational thermodynamics in the broad community.

\section{Challenges in intermetallic catalysts design} \label{intro:sec:catalysts}
Precise synthetic control of active site ensembles enables significant advancements in the design of selective heterogeneous catalysts. The active site can be thought of as the ensemble of atoms that directly catalyzes a reaction \cite{greeley2012active}. Intermetallic compounds (IMCs), characterized by their precise local atomic composition and structure (i.e., site occupancy), allow for systematic manipulation of the arrangement of multiple metals at active sites, provided the surface composition is consistent. The combination of active late transition metals, such as Pd, with a less catalytic second component, such as Zn, can be used to manipulate the active site ensemble, tuning the active site arrangement and electronic structure to facilitate the desired catalytic transformation while avoiding non-selective reactions. Several bimetallic IMCs have been identified that exhibit distinct catalytic properties compared to monometallic catalysts. For instance, Pd-Ga IMCs have been reported to show enhanced selectivity for acetylene semi-hydrogenation \cite{kovnir2007new, prinz2014adsorption}. Additionally, MgO-supported Ni-Ga IMCs have been investigated and demonstrated significantly higher selectivity for the semi-hydrogenation of phenylacetylene compared to pure Ni \cite{li2014nickel}.

Designing intermetallic catalysts involves addressing challenges related to thermodynamic stability and surface configuration of candidate IMCs. Ensuring the stability of these catalysts is crucial for both their design and processing, as variations in factors such as the composition of active metals can lead to phase transformations or decomposition, potentially undermining catalyst performance and longevity. Additionally, achieving a stable and well-defined surface configuration that retains the desired catalytic properties under operational conditions poses a significant challenge. A thorough understanding of the interplay between bulk thermodynamics and surface phenomena is essential for optimizing intermetallic catalysts. This requires comprehensive knowledge of phase diagrams and the ability to precisely control surface structures to ensure consistent performance and durability.

Determining phase stability and its variation with external conditions necessitates modeling the thermodynamic properties of all individual phases as functions of variables such as temperature and composition. The CALPHAD method is employed to model the Gibbs energies of both stable and metastable phases, using parameterized functions of temperature, composition, pressure, and internal degrees of freedom. This approach integrates experimental data with theoretical insights from density functional theory (DFT) calculations. By global minimization of Gibbs energy, this method allows for the determination of the distribution of active and non-active components, which in turn helps to identify stable bulk and surface configurations.

\section{Challenges in complex molten salts liquid modeling } \label{intro:sec:moltensalts}
Molten Salt Reactor (MSR) is one of the few game-changing concepts with rigorous safety standards while simultaneously achieving high levels of reliability and efficiency due to the use of molten salts as solvents for dissolving nuclear fuels \cite{blander1964molten, abram2008generation, cottrell1955operation}. The MSR utilizes a molten salt mixture, such as LiF-BeF$_2$-UF$_4$, in which fissile and fertile isotopes (e.g., $^{233}$U, $^{235}$U, $^{238}$U, and/or $^{239}$Pu) are dissolved. This mixture circulates continuously from the reactor core to the heat exchanger \cite{blander1964molten, leblanc2010molten}. A critical aspect of this system is its safety, which underscores the importance of carefully selecting appropriate molten salts \cite{benevs2013thermodynamic}.

Alkali and alkaline-earth metal fluorides, which can dissolve actinide fluorides like UF$_4$ and PuF$_3$, are central to MSR fuel salts. For instance, the $^7$LiF-BeF$_2$-ZrF$_4$-UF$_4$ fuel salt, with $^{235}$U, $^{233}$U, and/or $^{239}$Pu as fissile drivers, was used in the Molten Salt Reactor Experiment (MSRE) at Oak Ridge National Laboratory (ORNL) from 1965 to 1969 \cite{blander1964molten}. The coolant salt in the secondary loop was $^7$Li$_2$BeF$_4$. To date, substantial experimental data exist for simple coolant salt systems such as FLiNaK (the LiF-NaF-KF eutectics with its mole fraction around 0.465-0.115-0.420) and LiCl-NaCl-MgCl$_2$. However, data for chloride fuel salts—such as NaCl-UCl$_3$, NaCl-UCl$_3$-(Pu, TRU)Cl$_3$, and NaCl-MgCl$_2$-UCl$_3$—are limited \cite{mourogov2006potentialities}, positioning these chlorides as emerging model salts. Electrochemical pyroprocessing of used nuclear fuel with chloride melt matrices (e.g., LiCl-KCl) offers a promising option for the proliferation-resistant separation and recovery of fissile materials, particularly for high burn-up or fast reactor fuels where traditional solvent extraction methods may not be applicable \cite{blander1964molten}. The integration of pyroprocessing with reactor technology highlighted chloride-based molten salts as key materials for next-generation MSRs.

The CALPHAD method is extensively employed to predict the thermodynamic properties of molten salts. The primary thermodynamic databases for molten salts include the FactSage database \cite{bale2002factsage} and the open Molten Salt Thermodynamic Database (MSTDB-TC) \cite{ard2022development}. Despite these advancements, several challenges persist within the community: efficient high-throughput modeling of multicomponent molten salt systems, integrating modeling results from different database formats, ensuring the reliability and addressing the uncertainty of CALPHAD predictions, and selecting appropriate thermodynamic models for describing atomic environments in molten salts. The implementation of several thermodynamic models including the modified quasichemical model with quadrupled approximation (MQMQA) into the PyCalphad and ESPEI, has significantly enhanced high-throughput CALPHAD modeling by enabling uncertainty quantification and improved model selection. These developments are expected to provide more accurate and reliable predictions of critical molten salt properties.

\section{Executive Summary} \label{intro:sec:summary}
%%%%%%%%%%
First, Chapter \fullref{chap:method} introduces the first-principles calculations and CALPHAD method. First-principles calculations predict thermodynamic properties at both 0 K and finite temperatures, providing critical data to enhance the accuracy of CALPHAD modeling. Various thermodynamic models for the Gibbs energy function are presented, including the CEF model \cite{hillert1970regular}, the associate model \cite{sommer1982association}, the two-sublattice ionic model \cite{hillert1985two}, and the MQMQA model \cite{pelton2001modified}. The open-source software PyCalphad and ESPEI are utilized for computational thermodynamics. Additionally, this chapter discusses Bayesian parameter estimation used in the parameter optimization process, highlighting its role in uncertainty quantification and model selection.

Next, Chapter \fullref{chap:intermetallics} explores the application of CALPHAD modeling to intermetallic catalysts. It demonstrates how selecting the appropriate sublattice model for the $\gamma$-brass phase in the binary Pd-Zn system, as well as the ternary Pd-Zn-M (M=Cu, Ag, Au) systems, facilitates the investigation of site occupancy for active metals Pd, Cu, Ag, and Au in the $\gamma$-brass phase. The chapter also provides predictions regarding surface structure and active sites in intermetallic catalysts, aimed at optimizing the selectivity of hydrogenation reactions.

Chapter \fullref{chap:moltensalts} focuses on describing short-range ordering in complex molten salts using the CALPHAD method. The CALPHAD modeling aided by first-principles calculations are used in molten salts systems such as (LiF, NaF, KF, CrF$_2$)-CrF$_3$ and LiCl-KCl-LaCl$_3$. This chapter includes uncertainty quantification and propagation to assess the reliability of the modeling, as well as a detailed discussion of comparing various liquid models for the molten salts. Bayesian statistics are employed for model selection, providing insights into the predictive behavior of the models.

Chapter \fullref{chap:models} the enhancement of applicability of PyCalphad through the integration of additional thermodynamic models. The universal quasichemical model (UNIQUAC) \cite{abrams1975statistical} is introduced and successfully implemented in PyCalphad, accompanied by thorough validation and demonstration. Additionally, a custom model template generator is developed to facilitate the efficient implementation of various thermodynamic models. This chapter also includes a demonstration of the application of this template generator in implementing the Peng-Robinson equation of state (PR EOS) \cite{peng1976new}.

Lastly, Chapter \fullref{chap:conclusion} provides a comprehensive summary of the current research on atomic environments in intermetallic catalysts and molten salts. This chapter emphasizes the importance of selecting suitable thermodynamic models for an appropriate description of atomic environments in the phase and accurate predictions of thermodynamic properties. Bayesian statistics are employed for the modeling process, which enables model comparison and selection, ensuring robust and reliable results. The chapter also highlights the development of software features, including a custom model template generator, which is made available to the broader community to improve the efficiency and accessibility of computational thermodynamics.