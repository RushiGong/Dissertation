\chapter{Introduction} \label{sec:Introduction}

\section{CALPHAD modeling with model selection} \label{intro:sec:calphad}


\section{Key challenges in intermetallic catalysts design} \label{intro:sec:catalysts}


\section{Key challenges in complex molten salts liquid modeling } \label{intro:sec:moltensalts}
Molten Salt Reactor (MSR) is one of the few game-changing concepts with rigorous safety standards while simultaneously achieving high levels of reliability and efficiency due to the use of molten salts as solvents for dissolving nuclear fuels [1–3]. The MSR concept consists of a molten salt mixture (such as LiF-BeF2-UF4), in which the fissile and fertile isotopes (such as 233U, 235U, 238U, and/or 239Pu) are dissolved, circulating from the reactor core to the heat exchanger continuously [1,4]. A very important feature of this system is its safety, requiring more attention to select molten salts [5]. Alkali and alkaline-earth metal fluorides, in which actinide fluorides (such as UF4 and PuF3) can be dissolved, are considered as the key materials for MSR fuel salts [1]. For example, the fuel salt of 7LiF-BeF2-ZrF4-UF4 with 235U, 233U, and/or 239Pu as the fissile driver was used in the Molten Salt Reactor Experiment (MSRE) operated at Oak Ridge National Laboratory (ORNL) from 1965 to 1969 [1] with the coolant salt in the secondary loop being 7Li2BeF4. The Molten Salt Chemistry Workshop [1] suggested that fluoride salts, which are similar to the MSRE salts, are primarily for the thermal spectrum applications and for the primary and secondary coolant candidates, such as the FLiNaK, i.e., the LiF-NaF-KF eutectics with its mole fraction around 0.465-0.115-0.420 [6,7]. 



\section{Executive Summary } \label{intro:sec:summary}

%%%%%%%%%%
First, Chapter \fullref{chap:intermetallics} 

%\printbibliography[heading=subbibintoc]